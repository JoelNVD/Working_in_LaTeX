%====================================================================================
\section{Fallas de mercado}
%====================================================================================
\begin{frame}{Introducción}
En un sistema de libre mercado (competitivo) los recursos se asignan de manera óptima, en el sentido de Pareto. \\[0.3cm]

Es decir equilibrio eficiencia u óptimo paretiano (primer Teorema del Bienestar)
\end{frame}
%------------------------------------------------
\begin{frame}{¿Qué supuestos explican esta igualdad?}
Entre otros
	\begin{itemize}
		\item La existencia de mercados para todos los bienes.
		\item Todas las variables de las que dependen las F.O. de los agentes pertenecen a sus correspondientes conjuntos de elección individuales.
	\end{itemize}
\end{frame}
%------------------------------------------------
\begin{frame}{Fallas de mercado}
En general, una falla de mercado, es una situación en la cual el mercado, ausente la intervención gubernamental,
	\begin{itemize}
		\item lleva a ineficiencias,
		\item falla en su tarea de asignar los recursos eficientemente.
	\end{itemize}
Con fallas de mercado:
	$$\text{Equilibrio } \neq \text{ óptimo}$$
El funcionamiento de los mercados no permite aprovechar todas las posibles ganancias derivadas de la producción e intercambio de mercancías.
\end{frame}
%------------------------------------------------
\begin{frame}{Fallas de mercado}
Solución inviable:
	$$\text{coordinación entre los agentes}$$
Soluciones viables:
	$$\text{Mecanismos institucionales}$$
	$$\text{Intervención del gobierno}$$
Las principales son:
	\begin{itemize}
		\item Monopolio (poder de mercado)
		\item Desequilibrio del mercado
		\item Externalidades
		\item Bienes Públicos
		\item Información asimétrica
	\end{itemize}
	\pause
	\vspace{-1.7cm}
	\begin{tikzpicture}
		\draw [red] (0,3.2) -- (0,5) -- (5,5) -- (5,3.2) -- (0,3.2);
	\end{tikzpicture}
\end{frame}