\begin{tikzpicture}[domain=5.5:10,
					tangent/.style={
						decoration={
							markings,% switch on markings
							mark=
							at position #1
							with
							{
								\coordinate (tangent point-\pgfkeysvalueof{/pgf/decoration/mark info/sequence number}) at (0pt,0pt);
								\coordinate (tangent unit vector-\pgfkeysvalueof{/pgf/decoration/mark info/sequence number}) at (1,0pt);
								\coordinate (tangent orthogonal unit vector-\pgfkeysvalueof{/pgf/decoration/mark info/sequence number}) at (0pt,1);
							}
						},
						postaction=decorate
					},
					use tangent/.style={
						shift=(tangent point-#1),
						x=(tangent unit vector-#1),
						y=(tangent orthogonal unit vector-#1)
					},
					use tangent/.default=1,
					scale=0.8
					]
	
	% Relleno
		\fill [pattern=crosshatch dots,pattern color=orange!60!white] (5.69,4.15) -- plot (\x, {sqrt(10-\x)/0.5 }) -- (5.69,0);\textbf{}
	% Ejes
		\draw[<->] (5,0) node[align=center, left] {\footnotesize $-L$} -- (10,0)-- (10,4.5) node[align=center, above] {\footnotesize $q$};
	% Curva
		\draw[color=orange,samples=1000, tangent=0.4] plot (\x, {sqrt(10-\x)/0.5 });
	% Punto arbitario
		\filldraw[use tangent] (0,0) circle (2pt);
	% Tangente
		\draw[use tangent] (-2.2,0) -- (2.18,0);
	% Etiquetas
		\draw (10,1.4) node [scale=0.8, right] {$\frac{\pi(p,w)}{p}$};
		\draw (5.5,4.3) node [scale=0.8, left] {$q=f(L)$};
		\draw (5.6,0) node [scale=0.8, below] {$-\overline{L}$};
	% Linea punteadas
		\draw[dashed] (7.73,0) node [scale=0.8, below] {$-L(p,w)$} -- (7.73,3.03) -- (10,3.03) node [scale=0.8, right] {$q(p,w)$};
	% linea
		\draw (5.69,0)--(5.69,4.15);
\end{tikzpicture}