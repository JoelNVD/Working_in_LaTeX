\begin{enumerate}[a)]
	\item La fucnión de producción del bien 2, $q_2 = K_2 + L_2$, se puede expresar como
					\begin{gather}
						q_2 = (10 - K_1) + 2(10- L_1) = 30-(K_1 + 2L_2) \label{eq7}
					\end{gather}
			y dado que $q_1 = K_1 + 2L_1$, basta con reescribir (\ref{eq7}) para llegar al $CPP$ de la economía
					\begin{gather}
						\left\lbrace (q_1,q_2) \in \mathbb{R}_{+}^{2} \mid q_2(q_1) = 30 - q_1\right\rbrace \label{eq8}
					\end{gather}
			De mamenra análogo, el $CPU$ se obtiene teniendo en cuenta las funciones de utlidad de los consumidores, $u^1 = q_{1}^{1} + q_{2}^{1}$ y $u^2 = q_{2}^{2}$, y además el hecho de que la cantidad de bien 1 satisface la condición
					$$q_{1}^{1} + 0 = q_1$$
			mientras que la cantidad de bien 2 verifica
					$$q_{2}^{1} + q_{2}^{2} = q_{2}$$
			condición que teniendo en cuenta el $CPP$ dado en (\ref{eq8}) se convierte en
					$$q_{2}^{1} + q_{2}^{2} = 30 - q_{1}$$
			Sumando las utilidades, se tiene que $u^1 + u^2 = q_1 + q_2 = 30$. En definitiva, el $CPP$ de la economía descrita es
					$$\left\lbrace (u^1,u^2) \in \mathbb{R}_{+}^{2} \mid u^2(u^1) = 30 - u^1\right\rbrace$$
	\item Dada las asignaciones $\left[ \left( q_{1}^{1},q_{2}^{1} \right) ,\left( q_{1}^{2},q_{2}^{2} \right) \right] = \left[ \left( 10, 5\right) ,\left( 0, 15\right) \right]$, asignación representada por el punto $A$ de la caga de Edgeworth para el consumo de la siguiente figura:
			\begin{center}
				\begin{tikzpicture}[scale=1.065]
					% FPP
						\draw[orange] (0,3) node [black, left, scale = 0.6] {30} -- (3,0) node [black, below, scale = 0.6] {30};
					% Curva de indiferencia
						% Agente A
							\draw[blue] (0,2) node [black, above left, scale = 0.5] {$20$} -- (1,1);
							\draw[blue] (0,1.5) -- (1,0.5);
							\draw[blue] (0,1) -- (1,0) node [black, below right, scale = 0.5] {$10$};
						% Agente B
							\draw[red] (0,0.5) -- (1,0.5);
							\draw[red] (0,1) -- (1,1);
							\draw[red] (0,1.5) -- (1,1.5);
					% Caja en consumo
						\draw[<->] (0,2.5) node [black, left, scale = 0.6] {$q_{2}^{1}$}-- (0,0)  node [black, below left , scale = 0.6] {$O_1$}-- (1.5,0) node [black, below, scale = 0.6] {$q_{1}^{1}$};
						\draw[<->] (-0.5,2) node [black, left, scale = 0.6] {$q_{1}^{2}$} -- (1,2) node [black, above right, scale = 0.6] {$O_2$} -- (1,-0.5) node [black, below, scale = 0.6] {$q_{2}^{2}$};
					% Eje
						\draw[<->] (0,4) node[align=center, above] {$q_2$} -- (0,0)  -- (4,0) node[align=center, right] {$q_1$};
					% Punto
						\draw[black, fill=black] (1,0.5) circle[radius=0.05] node [right, scale=0.7] {$A$};
				\end{tikzpicture}
			\end{center}
		es claro que una posibilidad para aumentar $u^2$ es aumentar $q_{2}^{2}$, pero entonces disminuye $y^1$. Del mismo modo, para aumentar $u^1$ habría que incrementar $q_{1}^{1}$ o $q_{2}^{1}$, pero entonces disminuye $u^1$. Luego la asignación $\left[ \left( q_{1}^{1},q_{2}^{1} \right) ,\left( q_{1}^{2},q_{2}^{2} \right) \right] = \left[ \left( 10, 5\right) ,\left( 0, 15\right) \right]$ es eficiente en el consumo. y como el plan de producción $\left[ \left( q_1, K_1, L_1 \right) ,\left( q_2, K_2, L_2 \right) \right] = \left[ \left( 10, 10, 0\right), \left( 20, 0, 10\right) \right]$ que permite dichos consumos es técnicamente eficiente, se concluye que la asignación es eficiente en términos globales.\\
		
		Para descentralizar esta asignación de recursos, es necesario tener en cuenta, de acuerdo con la siguiente figura:
			\begin{center}
				\begin{tikzpicture}[scale=1.5]
					% FPP
						% Curva
							\draw[orange] (0,3) node [black, left, scale = 0.6] {30} -- (3,0) node [black, below, scale = 0.6] {30};
						% Eje
							\draw[<->] (0,4) node[align=center, above] {$q_2$} -- (0,0)  -- (4,0) node[align=center, right] {$q_1$};
					% Caja en consumo
						% Curva de indiferencia
							% Agente A
							\draw[blue] (0,1.5) -- (1,0.5);
							% Agente B
							\draw[red] (0,0.5) -- (1,0.5);
						% Ejes
							\draw[<->] (0,2.5) node [black, left, scale = 0.6] {$q_{2}^{1}$}-- (0,0)  node [black, below left , scale = 0.6] {$O_1$}-- (1.5,0) node [black, below, scale = 0.6] {$q_{1}^{1}$};
							\draw[<->] (-0.5,2) node [black, left, scale = 0.6] {$q_{1}^{2}$} -- (1,2) node [black, above right, scale = 0.6] {$O_2$} -- (1,-0.5) node [black, below, scale = 0.6] {$q_{2}^{2}$};
						% Punto
							\draw[black, fill=black] (1,0.5) circle[radius=0.035] node [right, scale=0.7] {$A$};
							\draw[black, fill=black] (0,1.5) circle[radius=0.035] node [left, scale=0.7] {$B$};
							\draw (1,0) node [black, below right, scale = 0.5] {$10$};
							\draw (0,2) node [black, above left, scale = 0.5] {$20$};
						% Recta de precios
						\draw (0.1,0.8) -- (1,0.5) -- (0.4,1.6);
					% Flechas
						\draw[->] (1.6,2.8) node [right, scale = 0.5] {Pendiente $= -1$}-- (0.05,1.45);
						\draw[->] (2,2.5) node  [right, scale = 0.5] {$\frac{p_1}{p_2} > 1$} -- (0.58,1.26);
						\draw[->] (2.4,2.2) node [right, scale = 0.5] {$\frac{p_1}{p_2} < 1$} --(0.6,0.63);
				\end{tikzpicture}
			\end{center}
		que si los precios de los bienes, $p_1$ y $p_2$, son tales que $\frac{p_1}{p_2} > 1$, no es posible descentralizar la citada asignación porque el individuo 1 consumiría únicamente el bien 2. Es por ello que los precios candidatos a descentralizar la asignación propuesta han de verificar la condición
			$$\frac{p_1}{p_2} \leq 1$$
		A la hora de calcular la dotación incial de recursos de los consumidores , $\left[W^1, W^2 \right] = \left[\left(w_{1}^{1},w_{2}^{1} \right), \left(w_{1}^{2},w_{2}^{2} \right) \right] $, en la que debes estas ``situada'' la economía para alcanzar la asignaciones propuestas, es necesario tener en cuenta que la utilidad del consumidor 1 debe cumplir la condición $u^{1}(W^1) = u^1(10,5)$. Con ello, las condiciones a satisfacer por $\left[W^1, W^2 \right]$ son:
			\begin{gather}
				10p_1 + 5p_2 = p_1w_{1}^{1} + p_2w_{2}^{1} \label{eq13}\\
				15p_2 = p_1w_{1}^{2}+p_2w_{2}^{2} \label{eq14} \\
				w_{1}^{1} + w_{1}^{2} = 10 \label{eq15} \\
				w_{2}^{1} + w_{2}^{2} = 20 \label{eq16} \\
				\frac{p_1}{p_2} < 1 \label{eq17}
			\end{gather}
		y, además, $\left[W^1, W^2 \right]$ ha de estar situada dentro del triángulo $ABC$ (ya que si está por encima, entonces $u^2(W^2) < u^2(0,5)$ y si está por debajo entonces $u^1(W^1) < u^1(10,5))$.\\
		
		Si fijamos $p_1 = 1$ y $p_2 = 2$, la condición (\ref{eq13}) se convierte en
			\begin{gather}
				10 + 10 = w_{1}^{1} + 2w_{2}^{1} \label{eq18}
			\end{gather}
		y la condición (\ref{eq14}) en
			\begin{gather}
				30 = w_{1}^{2} + 2w_{2}^{2} = (10 - w_{1}^{1}) + 2(20 - w_{2}^{1}) = 50 -w_{1}^{1} - 2w_{2}^{1} \label{eq19}
			\end{gather}
		una vez tenidas en cuenta (\ref{eq15}) y (\ref{eq16}). Finalmente, es vlaro que las condiciones (\ref{eq18}) y (\ref{eq19}) son idénticas, ya que ambas colapsan en $20 = w_{1}^{1} + 2w_{2}^{1}$, uqe es la condición de recursos inciales. En definitiva, con el sistema de precios
			$$(p_1,p_2) = (1,2)$$
		y el conjunto de rentas
			$$(m^1, m^2) = (20,30)$$
		se consigue descentralizar la asignación propuesta.
\end{enumerate}