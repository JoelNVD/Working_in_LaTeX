\begin{enumerate}[a)]
	\item Se requiere que las asignaciones factoriales sean factibles y no derrochadoras
			\begin{align*}
				RMT^A & = RMS^B\\
				\frac{PMgL^{A_x}}{PMgK^{A_x}} =\frac{K_{A_x}}{L_{A_x}} & = \frac{K_{B_y}}{2L_{B_y}} =\frac{PMgL^{B_y}}{PMgK^{B_y}}
			\end{align*}
		  Reemplazando $L_{A_x} + L_{B_y} = \overline{L}$ y $K_{A_x} + K_{B_y} = \overline{K}$, para hallar la curva de contrato
			$$\therefore \frac{K_{A_x}}{L_{A_x}} = \frac{\overline{K} - K_{A_x}}{2\overline{L} - L_{A_x}}$$
	\item Una asignación igualitaria de los factores entre las dos empresas significa que:
			\begin{gather*}
				L_{A_x} = \frac{\overline{L}}{2};\enskip K_{A_x}=\frac{\overline{K}}{2}\\
				L_{B_y} = \frac{\overline{L}}{2};\enskip K_{B_y}=\frac{\overline{K}}{2}
			\end{gather*}
		  Resulta evidente que con esta asignación de factores:
		  	$$\frac{K_{A_x}}{L_{A_x}} = \frac{K_{B_y}}{L_{B_y}}$$
		  por lo que no se verifica la condición $RMT^A = RMS^B$ de eficiencia
\end{enumerate}