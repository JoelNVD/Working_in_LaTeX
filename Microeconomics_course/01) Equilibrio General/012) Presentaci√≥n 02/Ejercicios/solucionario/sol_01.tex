\begin{center}
	\begingroup
		\setlength{\tabcolsep}{10pt} % Default value: 6pt
		\renewcommand{\arraystretch}{1.5} % Default value: 1
			\begin{tabular}{ccc}
					\hline
				Robinson consumidor & {} & Robinson empresario\\
				\hline
					Max U & {} & Max $\prod$\\
					s.a. RP & {} & s.a. RP\\
				Ofrece $L$ & {} &  Demanda $L$\\
				Demanda producto & {} &  Ofrece producto\\
					\hline
				\multicolumn{3}{c}{Economía descentralizada} \\
					\hline
				Consumidor 												& {} & Empresario\\
				Max $U = U\left( c, R\right)$ 							& {} & Max $\prod (w,p) = IT - CT$\\
				$\frac{w}{p} = RMS$ 									& {} & $\frac{w}{p} = PMg_L$\\
				$c^* \rightarrow$ Cantidad demandad de productos 		& {} & $q^* \rightarrow$ Cantidad ofrecida de producto\\
				$R^*$ o $L^* \rightarrow$ Cantidad demandad de trabajo	& {} & $L^* \rightarrow$ Cantidad demandad de trabajo \\
					\hline
			\end{tabular}
	\endgroup
\end{center}

\begin{itemize}
	\item Pregunta 1a: Robinson productor
			\begin{multicols}{2}
				\begin{itemize}
					\item $q = f(L)= L^{\frac{1}{2}}$
					\item Demanda de trabajo: $PMg_L = \frac{L^{-0.5}}{2} = \frac{w}{p} \Longrightarrow L^D = \frac{p^2}{4w^2}$
					\item Oferta de bienes: $q = \left( L^D\right) ^{0.5} \Longrightarrow q^S = \frac{p}{2w}$
					\item $\pi = pq(L) - wL= p \left( \frac{p}{2w}\right) - w\left( \frac{p^2}{4w^2}\right) $\\
					 	  $\pi = \frac{p^2}{4w}$
				\end{itemize}
			\end{multicols}
	\item Pregunta 1b: Robinson consumidor
			$$\text{Max } U = U \left( c,R\right) =cR \Longrightarrow \text{Ingreso} = \text{Gasto}$$
				\begin{center}
					\begingroup
						\setlength{\tabcolsep}{10pt} % Default value: 6pt
						\renewcommand{\arraystretch}{1.5} % Default value: 1
							\begin{tabular}{c|c|c}
								Demanda de ocio & Oferta de trabajo & Demanda de cocos\\
								$TMS = \frac{w}{p} \Rightarrow \frac{UMgR}{UMgC} = \frac{c}{R}  = \frac{w}{p}$ & $R + L  = 24$ & $\frac{c}{R} = \frac{w}{p}$ \\
								$pc  = wR$ & $24 - L  = R = L + \frac{p^2}{8w^2}$ & $\frac{c}{24-L} = \frac{w}{p}$\\
								$wL+\pi  = pc $ & $L^S = 12 - \frac{p^2}{8w^2}$ & $\frac{c}{24-\left( 12 - \frac{p^2}{8w^2}\right) }  = \frac{w}{p}$\\
								$wL + \pi  = wR $ & {} & $c^D = 12\frac{w}{p} + \frac{p}{8w}$\\
								$R = L + \frac{p^2}{4w^2}$ & {} & {}\\
							\end{tabular}
					\endgroup
				\end{center}
	\item Pregunta 1c: Equilibrio Walrasiano
			\vspace{-0.8cm}
				\begin{multicols}{2}
					\begin{gather*}
						L^D = L^S\\
						\frac{p^2}{4w^2} = 12 - \frac{p^2}{8w^2}\\
						\left( \frac{p_x}{p_y}\right)^* = \left( \frac{w}{p}\right)^* = \frac{1}{4\sqrt{2}}\\
						L^D=\frac{1}{4}\left( 4\sqrt{2}\right)^2=8=L^S\\
					\end{gather*}
					
					\begin{gather*}
						R = 24 - L \Longrightarrow R = 16 \\
						q^S = \sqrt{8} \approx 2.83\\
						c^D \approx 2.83 \\
						U = cR \approx 45.28\\
						\pi = IT - CT = pq - wL
					\end{gather*}
				\end{multicols}
			\vspace{-0.8cm}
\end{itemize}