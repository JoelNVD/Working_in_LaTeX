%====================================================================================
\section[Comercio internacional]{Aplicación al comercio internacional}
%====================================================================================

\begin{frame}{Actividad 2}
	La economía de un país produce únicamente dos bienes, cereal y vid, y tiene todos sus recursos plena y eficientemente empleados. En esta situación las opciones de producción que tiene son las siguientes:\\
		\begin{center}
			\begin{tabular}{ccc}
					\hline
				Opciones & Cerela & Vid \\
					\hline
				A & 0 & 40 \\
				B & 5 & 32 \\
				C & 15 & 15 \\
					\hline
			\end{tabular}
		\end{center}
	Se pide:
		\begin{itemize}
			\item Realiza la representación gráfica de la frontera de posibilidades de producción.
			\item Calcula los diferentes costes de oportunidad.
		\end{itemize}
\end{frame}
%-------------------------------------------------
\begin{frame}{Actividad 2}
	\begin{itemize}
		\item Pregunta A Gráfico
		\item Pregunta B: 
				El coste de oportunidad es la opción a la que se renuncia cuando se decide aumentar la producción de otro bien en una situación de escasez, es decir, cuando todos los recursos están eficientemente utilizados. Por ello, aumentar la producción de uno de los bienes supone la reducción de otro, esa reducción es el coste de oportunidad.
					\begin{center}
						\begin{tabular}{cccc}
							\hline
							Opciones & Cerela & Vid & Coste de oportunidad\\
							\hline
							A & 0  & 40 & - \\
							B & 5  & 32 & 8 \\
							C & 15 & 15 & 17 \\
							\hline
						\end{tabular}
					\end{center}
	\end{itemize}
\end{frame}