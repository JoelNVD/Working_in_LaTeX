%====================================================================================
\section{Primer Teorema del Bienestar}
%====================================================================================
\begin{frame}{Primer Teorema del Bienestar}
``Si los consumidores son codiciosos y no hay externalidades, las asignaciones de equilibrio walrasiano son óptimas de Pareto''
	\begin{itemize}
		\item En equilibrio walrasiano, se identifican asignaciones sobre la recta presupuestaria para las que las curvas de indiferencia son tangentes.
		\item Por tanto, en una asignación como esta, no podemos encontrar otra asignación factible que permita mejorar a ambos consumidores simultáneamente.
		\item Por tanto: cualquier asignación de equilibrio walrasiano, necesariamente es una asignación óptimo de Pareto.
	\end{itemize}
\end{frame}
%------------------------------------------------
\begin{frame}{Primer Teorema del Bienestar}
	\begin{center}
		\begin{tikzpicture}[transform canvas={scale=0.6}]
	% Formato de CAJA
		\draw[->] (0.5,0.5) node[align=center, below left] {\footnotesize $O_A$} -- (0.5,4.5) node[align=center, above] {\footnotesize $K^{A}$};
		\draw[->] (0.5,0.5) -- (8.5,0.5) node[align=center, right] {\footnotesize $L^{A}$};
	
		% Curvas de indiferencia1
			% Agente B
				\draw  [blue] (0.6,4) ..controls (1.4,1.4) and (1.74,1) .. (6,0.6);
				\draw  [blue] (1,4.3) ..controls (1.7,2.1) and (2,1.6) .. (6.3,1);
				\draw  [blue] (1.6,4.6) ..controls (2.1,3.2) and (1.74,2.2) .. (6.7,1.4);
			
			% Flechas
				\node[draw, single arrow,
						minimum height=30mm, minimum width=1mm,
						single arrow head extend=1.5mm,
						anchor=west, blue, fill=blue, scale=0.5, rotate=50] at (2.7,2.7) {};
	
	% Formato de CAJA rotado
		\draw[->] (18,4) node[align=center, above right] {\footnotesize $O_B$} -- (10,4) node[align=center, left] {\footnotesize $L^{B}$};
		\draw[->] (18,4) -- (18,0) node[align=center, below] {\footnotesize $K^{B}$};
	
		% Curvas de indiferencia1
			% Agente B
				\draw [red] (12,3.9) .. controls (16.76,3.5) and (17.1,3.1) .. (17.9,0.5);
				\draw [red] (11.7,3.5) .. controls (16.5,2.9) and (16.8,2.4) .. (17.3,0.2);
				\draw [red] (11.3,3.1) .. controls (16.76,2.3) and (16.4,1.3) .. (16.6,-0.1);
			
			% Flechas
				\node[draw, single arrow,
						minimum height=30mm, minimum width=1mm,
						single arrow head extend=1.5mm,
						anchor=west, red, fill=red, scale=0.5, rotate=-130] at (15.7,1.8) {};
\end{tikzpicture}
	\end{center}
\end{frame}
%------------------------------------------------
\begin{frame}{Demostración del primer teorema}
	\begin{multicols}{2}
		\begin{itemize}
			{\small
			\item Supongamos que $\left[p, \left( x^1, x^2, ... , x^I \right)  \right]$ es un equilibrio walrasiano.
			\item $\left(x^1, x^2, ... , x^I \right) $ no es eficiente a lo Pareto.
			\item Entonces existe $\left(\hat{x}^1, \hat{x}^2, ..., \hat{x}^I \right) $ tal que:
					\begin{align}
						&\sum \hat{x}^i = \sum \hat{w}^i \label{eq1}\\
						&\forall i, u^i\left(\hat{x}^i \right) \geq u^i\left(x^i \right) \label{eq2}
						\intertext{Para algún \textit{i}}
						& u^i\left(\hat{x}^i \right) \geq u^i\left(x^i \right) \label{eq3}
					\end{align}
			\item De (\ref{eq3}) se deduce que en equilibrio: $p\hat{x}^i \geq px^i$
			\item (\ref{eq2}) implica que para todo $i$: $p\hat{x}^i > px^i$
			\item Sumando para todos los agentes:
					$$\sum p\hat{x}^i > \sum px^i$$
			\item De donde se deduce que:
					$$p\sum \hat{x}^i > p \sum x^i = p\sum w$$
			\item Lo cual contradice (\ref{eq1})}
		\end{itemize}
	\end{multicols}
\end{frame}
%------------------------------------------------
\begin{frame}{Implicancia del Primer Teorema del Bienestar}
	Si una economía satisface los supuestos del modelo de equilibrio walrasiano y se encuentra en equilibrio, cualquier medida de política económica que pretenda mejorar a un agente, necesariamente perjudicará a otro.
\end{frame}