% Apartado de letras
\usepackage[utf8]{inputenc}
\usepackage[T1]{fontenc}
\usepackage[spanish]{babel}

% Aparatdo matemático
\usepackage{amsmath}
\usepackage{amsfonts}
\usepackage{amssymb}
\usepackage{mathtools}

% Apartado tablas, figuras, y otros
\usepackage{multirow,booktabs,setspace,caption,multicol,tabularx, tabulary}
\captionsetup{skip=0pt}

% Apartado Tablas y Figuras en APA
\DeclareCaptionLabelSeparator*{spaced}{\\[1ex]}
\captionsetup[table]{name = Tabla, textfont=it,format=plain,justification=justified,
	singlelinecheck=false,labelsep=spaced,skip=0pt}
\captionsetup[figure]{name = Figura, labelsep=period,labelfont=it,justification=justified,
	singlelinecheck=false,font=doublespacing}
	% Para hacer tablas APA más fácil
	\usepackage[flushleft]{threeparttable}

% Apartado hyperres
\usepackage{hyperref}

% Apartado Tikz
\usepackage{tikz, pgfplots}
\usetikzlibrary{positioning,calc,arrows}
\usetikzlibrary{shapes.arrows}
\usetikzlibrary{patterns}

	\tikzset{ % to make dots on the graph
		dot/.style = {circle, fill, minimum size=#1,
			inner sep=0pt, outer sep=0pt},
		dot/.default = 6pt % size of the circle diameter 
	}

% Apartado de colores
\usepackage{xcolor}

% Apartado tipò de letra
\usepackage{textcomp}
\usepackage{libertine}
\usepackage[libertine]{newtxmath}

% Apartado de íconos
\usepackage{fontawesome5}

%Creación de entornos matemáticas
\newtheorem{defi}{Definición}[section]
\newtheorem{lema}{Lema}[section]
\newtheorem{teo}{Teorema}[section]
\newtheorem{coro}{Corolario}[section]

% Diversos
\usefonttheme[onlymath]{serif}
\usepackage{remreset}
\usepackage{makecell}
\usepackage{float}
\usepackage{subfigure}

% Apartado justificación de textos
\usepackage{ragged2e}
\justifying
\renewcommand{\raggedright}{\leftskip=0pt \rightskip=0pt plus 0cm}