%====================================================================================
% Preamble
%------------------------------------------------------------------------------------
\documentclass[10pt,a4paper]{article}

% Apartado de texto
\usepackage[utf8]{inputenc}
\usepackage[spanish]{babel}
\usepackage[T1]{fontenc}

% Apartado matemático
\usepackage{amsmath}
\usepackage{amsfonts}
\usepackage{amssymb}
\usepackage{mathtools}
\usepackage{mathrsfs}
\usepackage[libertine]{newtxmath}

% Apartadode no sangría
\usepackage{parskip}

% Aparatod posición del texto
\usepackage[left=2cm,right=2cm,top=2cm,bottom=2cm]{geometry}
\usepackage{fancyhdr}

\usepackage{graphicx}
\usepackage{xcolor}
\definecolor{cardinal}{rgb}{0.77, 0.12, 0.23}

% Apartado de columnas y filas: juntarlas o separarlas
\usepackage{multicol, multirow}

% Apartado differents label en listado
\usepackage[shortlabels]{enumitem}

% Apartado hyperres
\usepackage[hidelinks]{hyperref}

\usepackage{tcolorbox}
\newtcolorbox{nota}[1]{
	colback=cardinal!5!white,
	colframe=cardinal!75!black,
	fonttitle=\bfseries,
	title=#1
}

% Apartado dibujos
\usepackage{tikz, pgfplots}
\pgfplotsset{compat=1.18}
\usetikzlibrary{positioning,calc,arrows}
\usetikzlibrary{shapes.arrows}
\usetikzlibrary{patterns}
\usetikzlibrary{babel} % Para que recono > y <, en inglés no se pone, pero en spañol sí
\usetikzlibrary{shadings,shadows}
\usetikzlibrary{matrix}
\usetikzlibrary{intersections}

% Apartado cambio como por punto
\decimalpoint

% \highlight[<colour>]{<stuff>}
\newcommand{\highlight}[2][yellow]{\mathchoice%
	{\colorbox{#1}{$\displaystyle#2$}}%
	{\colorbox{#1}{$\textstyle#2$}}%
	{\colorbox{#1}{$\scriptstyle#2$}}%
	{\colorbox{#1}{$\scriptscriptstyle#2$}}}%

% Definir oprerador
\DeclareMathOperator*{\M}{Max}

%====================================================================================

%====================================================================================
% Body
%====================================================================================
% Title Page
%-----------
\textwidth=450pt \textheight=620pt \oddsidemargin=0in
\topmargin=-10pt
\pagestyle{fancy} \rhead{\scriptsize{\textbf{Código del Curso:} xxXXXXX} \\
	\textbf{Fecha:} XX/XX/2021 \& 2021-I \hspace{0.04cm}}
\lhead{\scriptsize{\textbf{Profesor:} José A. Valderrama}  \\
	\textbf{Curso:} Teoría Microeconómica II}
\newcommand{\re}[1]{\smallskip\textsf{\textbf{Respuesta}} \begin{sf}\\ #1 \end{sf} \bigskip}
\newcommand{\ay}[1]{ \scriptsize{\textsl{Hint: #1}}\normalsize{}}
\newcommand{\pr}[2]{\frac{\partial #1}{\partial #2}}

%------------------------------------------------------------------------------------
% Title
%---------
\begin{document}
	\begin{center}
		{\Large {\textbf{Práctica Dirigida N$^{\circ}$3}}}

		\textsc{Funciones de bienestar}
		
	\end{center}
% EJERCICIOS---------------------------------------------------------------------------------
\begin{enumerate}
	\item Obtenga la Frontera de Posibilidades de Utilidad (\emph{FPU})para una economía de intercambio puro formada por dos individuos con la siguientes preferencias y dotaciones iniciales:
			$$U_A = x_{A}^{1/2}y_{A}^{1/2}\quad , \quad U_B = x_{B}^{1/2}y_{B}^{1/2}\quad , \quad W_A = (800, 175)\quad , \quad W_B = (400, 125)$$
		  Para cada caso:
		  	\begin{enumerate}[a)]
		  		\item Obtener las ecuaciones del conjunto paretiano y de la \emph{FPU}
		  		\item Realice el gráfico de la \emph{FPU}
		  	\end{enumerate}
			\textbf{\LARGE Solución}\\
				\vspace{-0.7cm}
\begin{multicols}{2}
	\begin{tikzpicture}
		\begin{axis}[scale=0.9,
					 xmin=0, xmax=20,
					 ymin=0, ymax=12,
					 title style={at={(0.5,0.2)},anchor=north,yshift=5cm},
					 title = Consumidor $A$,
					 axis lines = left,
					 xtick={2,4,6,8,10,12,14,16,18},
					 ytick={2,4,6,8,10},
					 grid=both,
					 grid style={line width=.1pt, draw=gray!20},
					 clip = false,
					]
			
			\node [above]  at (current axis.above origin) {$x_{2}^{A}$};
			\node [right]  at (current axis.right of origin) {$x_{1}^{A}$};
			
			\draw[fill=black] (60,80) circle (1.5) node[above right] {$w^{A}$};
		\end{axis}
	\end{tikzpicture}

	\begin{tikzpicture}
		\begin{axis}[scale=0.9,
					 xmin=0, xmax=20,
					 ymin=0, ymax=12,
					 title style={at={(0.5,0.2)},anchor=north,yshift=5cm},
					 title = Consumidor $B$,
					 axis lines = left,
					 xtick={2,4,6,8,10,12,14,16,18},
					 ytick={2,4,6,8,10},
					 grid=both,
					 grid style={line width=.1pt, draw=gray!20},
					 clip = false,
					]
			
			\node [above]  at (current axis.above origin) {$x_{2}^{B}$};
			\node [right]  at (current axis.right of origin) {$x_{1}^{B}$};
			
			\draw[fill=black] (120,40) circle (1.5) node[above right] {$w^{B}$};
		\end{axis}
	\end{tikzpicture}
\end{multicols}

Otras posibilidades pasan por redistribuir unidades entre ambos consumidores. Por ejemplo, si $A$ quiere consumir más de 6 unidades de $x_1$ habrá que quitarle a $B$.\\

Podrían proponerse otras posibles repartos, si representamos una asiganción como $\left(x_{1}^{A}, x_{2}^{A}, x_{1}^{B}, x_{2}^{B}\right)$ y posibles asignaciones como $\left( 4,8,14,4\right)$, $\left(6,6,12,6 \right)$ o $\left( 10,5,8,7\right)$; sin embargo la asignación $\left( 9,5,11,6\right)$ no es una asignación válida por lo siguiente:

$$\begin{array}{ccc}
	x_{1}^{A} + x_{2}^{A} = w_{1}^{A} + w_{2}^{A} & {} & x_{1}^{B} + x_{2}^{B} = w_{1}^{B} + w_{2}^{B} \\
	 4 + 14    = 6 + 12  & {} & 8 + 4    = 8 + 4 \\
	 6 + 12    = 6 + 12  & {} & 6 + 6    = 8 + 4 \\
	 9 + 11 \neq 6 + 12  & {} & 5 + 6 \neq 8 + 4 
\end{array}$$

	\item Obtenga la \emph{FPU} para una economía de intercambio puro formada por dos individuos con las siguientes preferencias y dotaciones iniciales:
			$$U_A = x_Ay_A \quad ; \quad U_B = x_By_B \quad ; \quad W_A = (1,1)\quad ; \quad W_B = (1,2)$$
		  Para cada caso:
		  	\begin{itemize}
		  		\item Obtener las ecuaciones del conjunto paretiano y de la \emph{FPU}
		  		\item Realice el gráfico de la \emph{FPU}
		  	\end{itemize}
	  		\textbf{\LARGE Solución}\\
	  			La pregunta anterior se resolvió de la forma general, pero se puede resolver de forma práctica igualando \emph{TMS}; si y solo si, las funciones de utilidad son tipo \emph{Cobb-Douglas} o transformaciones monotónicas de esta. Veamos.

	\begin{enumerate}[a)]
		\item \colorbox{yellow}{$TMS_A = TMS_B$}
				Teniendo en cuenta:
					$$TMS = \frac{UMg_x}{UMg_y} = \frac{\partial U/\partial x}{\partial U/\partial y}$$
				entonces,
					$$\frac{y_A}{x_A} = \frac{y_B}{x_B} \rightarrow \frac{y_A}{x_A} = \frac{3-y_A}{2-x_A} \rightarrow y_A = \frac{3x_A}{2} \text{ y } y_B = \frac{3x_B}{2}$$
		\item $$
				\begin{array}{c|c}
					U_A = x_A\left( \frac{3x_A}{2}\right)  & U_B = x_B\left( \frac{3x_B}{2}\right) \\[0.3cm]
					x_A = \frac{\sqrt{6}}{3}U_A^{1/2} & x_B = \frac{\sqrt{6}}{3}U_B^{1/2}
				\end{array}
			  $$
			  	$x_A + x_B = 2 \longrightarrow U_A^{1/2}+U_B^{1/2} = \sqrt{6} \longrightarrow \therefore U_B = \left( \sqrt{6} - U_A^{1/2}\right)^{2} $
			  	\begin{center}
			  		\begin{tikzpicture}[scale = 0.7, samples = 100]
			  			\draw[purple, domain = 0:6] plot({\x},{(sqrt(6)-(\x)^(0.5))^(2)});
			  			\draw[<->] (0,7) node [above] {$U_B$} -- (0,0) -- (7,0) node [right] {$U_A$};
			  			\draw[black, fill=black] (0,6) circle[radius=0.07] node [left] {6};
			  			\draw[black, fill=black] (6,0) circle[radius=0.07] node [below] {6};
			  		\end{tikzpicture}
			  	\end{center}
			  	
	\end{enumerate}
	\item Se tiene dos consumidores con idénticas preferencias, descritas por $u = \ln(XY)$. Supongamos, además, que las dotaciones iniicales de los bienes son $W_A = (3,2); W_B=(2,3)$
			\begin{itemize}
				\item Calcular la ecuación de la \emph{FPU} e indicar si el punto de la dotación inicial está por encima o por debajo de esa curva
				\item Identificar en la \emph{FPU} los niveles de utilidad correspondientes a los extremos del núcleo de esta economía y calcular sus correspondientes asignaciones
			\end{itemize}
			\textbf{\LARGE Solución}\\
				\begin{enumerate}[a)]
	\item Se requiere que las asignaciones factoriales sean factibles y no derrochadoras
			\begin{align*}
				RMT^A & = RMS^B\\
				\frac{PMgL^{A_x}}{PMgK^{A_x}} =\frac{K_{A_x}}{L_{A_x}} & = \frac{K_{B_y}}{2L_{B_y}} =\frac{PMgL^{B_y}}{PMgK^{B_y}}
			\end{align*}
		  Reemplazando $L_{A_x} + L_{B_y} = \overline{L}$ y $K_{A_x} + K_{B_y} = \overline{K}$, para hallar la curva de contrato
			$$\therefore \frac{K_{A_x}}{L_{A_x}} = \frac{\overline{K} - K_{A_x}}{2\overline{L} - L_{A_x}}$$
	\item Una asignación igualitaria de los factores entre las dos empresas significa que:
			\begin{gather*}
				L_{A_x} = \frac{\overline{L}}{2};\enskip K_{A_x}=\frac{\overline{K}}{2}\\
				L_{B_y} = \frac{\overline{L}}{2};\enskip K_{B_y}=\frac{\overline{K}}{2}
			\end{gather*}
		  Resulta evidente que con esta asignación de factores:
		  	$$\frac{K_{A_x}}{L_{A_x}} = \frac{K_{B_y}}{L_{B_y}}$$
		  por lo que no se verifica la condición $RMT^A = RMS^B$ de eficiencia
\end{enumerate}
	\item Supongamos la existencia de una economía de intercambio puro en la que dos consumidores (1, 2) intercambian dos bienes $(x,y)$. Las preferencias de ambos consumidores cons respecto a estos bienes vieien representadas por $U_1 = 2x_{1}^{3/4}y_{1}^{1/4}; U_2 = x_{2}^{3/4}y_{2}^{1/4}$. Supongamos, además, que las dataciones iniciales de los bienes son: $\overline{x}_1 = 20, \overline{y}_1 = 40, \overline{x}_2 = 80, \overline{y}_2 = 60$
			\begin{itemize}
				\item Calcular la ecuación de la \emph{FPU} e indicar si el punto de la dotación inicial está por encima o por debajo de esa curva
				\item Identificar en la \emph{FPU} los niveles de utilidad correspondientes a los extremos del núcleo de esta economía y calcular sus correspondientes asignaciones
			\end{itemize}
			\textbf{\LARGE Solución}\\
				La curva de contrato está dada por la siguiente relación:
	\begin{eqnarray*}
		RMS_{A} & = & RMS_{B}\\[0.3cm]
		\frac{x_{2}^{A}}{x_{1}^{A}} & = & \frac{x_{2}^{B}}{x_{1}^{B}}\\[0.3cm]
		\frac{x_{2}^{A}}{x_{1}^{A}} & = & \frac{12 - x_{2}^{A}}{15 - x_{1}^{A}}\\[0.3cm]
		\therefore x_{2}^{A} & = & \frac{12}{15}x_{1}^{A}
	\end{eqnarray*}
La curva de contrato será una \textbf{LÍNEA RECTA}
	\item Determinar cuanto de los bienes $x$ y $y$ debería producirse y cómo deberían distribuirse entre los consumidores $A$ y $B$, si se tiene la siguiente información:
			\begin{itemize}
				\item Curva de Transformación: $x^2 + y^2 = 500$
				\item Función de utilidad: $u^A = x^Ay^A, u^B = x^By^B$
				\item Función de Bienestar Social:$w=u_Au_B$
			\end{itemize}
	\item Sobre los Teoremas Fundamentales del Bienestar:
			\begin{enumerate}[a)]
				\item Enunciar y demostrar el \emph{Primer Teorema Fundamental de la Economía del Bienestar}.
				\item Discutir rl siguiente enunciado \textit{``Es necesario tener un gobierno benevolente que se encargue de darnos una buena dotación de recursos''}
			\end{enumerate}
			\textbf{\LARGE Solución}\\
				La condición de óptimo:
		$$\sum RMS = \frac{P_{X}}{P_{Y}}$$
Teniendo en cuenta que:
	\begin{align*}
		Y_{1} + Y_{2} &= Y\\
		X_{1} = X_{2} &= X
	\end{align*}
Resolviendo:
	\begin{align*}
		\frac{Y_{1}}{2X_{1}} + \frac{Y_{2}}{2X_{2}} &= \frac{100}{0.2}\\
		\frac{Y}{2X} &= 500\\
		\frac{Y}{X} &= 1000
	\end{align*}
Hallando el nivel óptimo de X, usando la restricción monetaria conjunta:
	\begin{align*}
		P_{X}X + P_{Y}Y_{1} + P_{Y}Y_{2} &= M_{1} + M_{2}\\
		1000X + 0.2Y &= 2 \times 3000\\
		1000X + 0.2\left(1000X\right) &= 6000\\
		1200X &= 6000\\
		X^* &= 5
	\end{align*}
Por tanto el nivel óptimo del bien público es: $X^* = 5$.
	\item Sobre el \emph{Primer Teorema Fundamental del Bienestar}, considere una economía con dos agentes ($A, B$) y dos bienes ($x, y$). Los dos agentes tienen las siguientes funciones de utilidad:
			$$U_A = x^2y \quad , \quad U_B =xy^2$$
	La cantidad total del bien $x$ es igual a 3 y la cantidad total del bien $y$ es igual a 3.\\
	¿Cuál de las siguientes locaciones pueden ser un equilibrio competitivo?
		\begin{enumerate}[a)]
			\item $A_A = (x=1; y=2), A_B = (x=2, y=1)$
			\item $A_A = (x=1.5; y=1.5), A_B = (x=1.5, y=1.5)$
			\item $A_A = (x=2; y=1), A_B = (x=1, y=2)$
		\end{enumerate}
		\textbf{\LARGE Solución}\\
			\begin{enumerate}[a)]
	\item Dadas las tecnología Leontief de las empresas, no es posible resolver el problema de forma analítica. Si los resolvemos aplicando directamente el concepto de eficiencia productiva ,ayudándonos para ello de las representaciones gráficas de esta economía, la cual se hace en la siguiente figura:
		\begin{center}
			\begin{tikzpicture}
				% Curva de contrato
					\draw[purple] (0,0) -- (6,4);
					\draw[purple] (4.5,0) -- (6,4);
					
				% Relleno
					\fill [pattern=crosshatch dots,pattern color=purple!20!white] (0,0) -- (6,4) -- (4.5,0) -- (0,0);
					
				% Caja
					\draw[<->] (0,4.5) node[align=center, above, scale=0.8] {$L_1$} -- (0,0) node[below left] {\footnotesize $O_1$} -- (6.5,0) node[align=center, right, scale=0.8] {$K_1$};
					\draw[<->] (-0.5,4) node[align=center, left, scale=0.8] {$K_2$} -- (6,4) node[align=center, above right, scale=0.8] {$O_2$} -- (6,-0.5) node[align=center, below, scale=0.8] {$L_2$};
					
				% Union de puntos
					\draw[dashed, black, fill=black] (4.5,0) circle[radius=0.05] node [below, scale=0.7] {$A$} -- (4.5,3) circle[radius=0.05] node [above, scale=0.7] {$B$};
					
				% Curvas de indiferencia
					% Agente A
						\draw[blue] (3,2.5) -- (3,2) -- (3.5,2);
						\draw[blue] (2,2.5) -- (2,1.33) -- (3.17,1.33); 
					% Agente B
						\draw[red] (4.75,2) -- (5.25,2) -- (5.25,1.5);
						\draw[red] (3.82,1.33) -- (5,1.33) -- (5,0.16);
						
				% Texto
					\draw[purple] (3,0.5) node[align=center, above, scale=0.8] {Conjunto Paretiano};
			\end{tikzpicture}
		\end{center}
	Es evidente que la asignaciones eficientes en la producción son la dadas por los puntos del conjunto cerrado $O_1AO_2$. Es decir,
		\begin{gather}
			\left\lbrace \left( q_1, K_1, L_1\right) \right\rbrace = \left\lbrace 0 \leq K_1 \leq 15 \text{ y } L_1 \leq 0.5K_1\right\rbrace \cup \left\lbrace 15 \leq K_1 \leq 20 \text{ y }  0.5K_1\geq L_1 \geq 2K_1 - 30\right\rbrace \label{eq1}
		\end{gather}
	y donde el subconjunto $\{0 \leq K_1 \leq 15$ y $L_1 \leq 0.5K_1 \}$ es el representado gráficamente por el triángulo $O_1AB$, en tanto que el sobconjunto $\{15 \leq K_1 \leq 20$ y $ 0.5K_1\geq L_1 \geq 2K_1 - 30 \}$ es el dado por el triangulo $ABO_2$.
	
	\item Para determinar el $CPP$, lo que hacemos es tomar algunos puntos del conjunto de contrato (de la producción) definido en (\ref{eq1}) y representarlo en el espacio de productos $\left\lbrace q_1,q_2\right\rbrace$. Por ejemplo, a lo largo de la recta $L_1 = 0.5 K_1$, la correspondencia es la siguiente:
		\begin{center}
			\begin{tabular}{ccc}
				\hline
					$\left( K_1, L_1\right)$ & {} & $\left( q_1, q_2\right)$ \\
				\hline
					(0,0) 	 & {} & (0,10)	\\
					(6,3) 	 & {} & (6,7)	\\
					(10,5)   & {} & (10,5)	\\
					(15,7.5) & {} & (15,2.5) \\
				\hline
			\end{tabular}
		\end{center}
	y es evidente que la FPP es\footnote{Tómense los punto (0,10) y (15,2.5) y recuérdese la ecuación de una recta que pasa por dos puntos.}
			$$q_2 = 10 - 0.5q_1, \text{ si } 0 \leq q_1 \leq 15$$
	Análogamente, si del $CCP$ definido en (\ref{eq1}) tomamos algunos puntos a lo largo de la recta $L_1 = 2K_1 - 30$ y los proyectamos en el espacio de productos $\left\lbrace q_1,q_2\right\rbrace $, se llega a
		\begin{center}
			\begin{tabular}{ccc}
				\hline
					$\left( K_2, L_2\right)$ & {} & $\left( q_1, q_2\right)$ \\
				\hline
					(15,0) 	 & {} & (0,10)	\\
					(18,6) 	 & {} & (12,4)	\\
					(19,8)   & {} & (16,2)	\\
					(20,10)  & {} & (20,0)  \\
				\hline
			\end{tabular}
		\end{center}
	pudiéndose inferir que
			$$q_2 = 10 - 0.5q_1, \text{ si } 0 \leq q_1 \leq 20$$
	En definitiva, la $FPP$ de esta economía es el conjunto de producciones
			$$\left\lbrace \left(q_1, q_2 \right) \in \mathbb{R}_{+}^{2} \mid q_2 \left( q_1\right) = 10 -0.5q_1, 0 \leq q_1 \leq 20 \right\rbrace $$
	\item A partir de la tecnología de la empresa 1, es evidente que las demandas de factores de esta empresa son las dadas por
			$$\left( K_1, L_1\right) = \left( \frac{2q_1}{2r + w}, \frac{q_1}{2r + w}\right) $$
	ya que los inputs $K$ y $L$ pueden ser interpretados como un ``input compuesto'' cuyp coste es $2r+w$. Análogamente, las demandas de inputs de la empresa 2 son
			$$\left( K_2, L_2\right) = \left( \frac{q_2}{r + 2w}, \frac{2q_2}{r + 2w}\right) $$
	Luego, el par de precios $(r, w)$ tales que
			$$\frac{2q_1}{2r + w}+  \frac{q_1}{2r + w} \leq 20$$
	y
			$$\frac{q_2}{r + 2w} + \frac{2q_2}{r + 2w} \leq 10$$
	define el equilibrio walrasiano de esta economía de inputs complementarios.
\end{enumerate}
	\item Considere una comunidad compuesta por dos individuos (1, 2). $A$, $B$ y $C$ son tres puntos que pertenecen a la \emph{FPU} de la comunidad. Las utilidad individuales ($U^A. U^B$) se describen como sigue:
				\begin{center}
					\begin{tabular}{ccc}
						\hline
							Points & $U_1$ & $U_2$ \\
						\hline
							$A$ & 2 & 7 \\
							$B$ & 5 & 5 \\
							$C$ & 7 & 2 \\
						\hline
					\end{tabular}
				\end{center}
		Considerando $D$:
				\begin{center}
					\begin{tabular}{ccc}
						\hline
							Points & $U_1$ & $U_2$ \\
						\hline
							$D$ & 2 & 5 \\
						\hline
					\end{tabular}
				\end{center}
		\begin{enumerate}[a)]
			\item ¿Pertenece $G$ a la \emph{FPU} de la comunidad?
			\item ¿Puedes encontrar punto en la \emph{FPU} que es Pareto-superior a $C$? Justifica tu respuesta
			\item ¿Cuál de los punto(s) $A$, $B$, $C$, $D$ miente(n) en la curva de indiferencia más alta de una Función de Bienestar Social (\emph{FBS}) utilitarista?
		\end{enumerate}
			\textbf{\LARGE Solución}\\
				\vspace{-0.3cm}
\begin{center}
	\begingroup
		\setlength{\tabcolsep}{10pt}
		\renewcommand{\arraystretch}{1.5} 
			\begin{tabular}{ccc}
					\hline
				\multicolumn{3}{c}{Datos} \\
					\hline
				2 consumidores: & {} & $A$ y $B$ \\
				2 bienes: & {} &$x$ e $y$ \\
				F.U: & {} &$u^{A} = \ln(x_A) + 2y_A$ y $u^{B} = \ln(x_B) + 2y_B$ \\
				Dotaciones: & {} &$w^A = (1, 4)$ y $w^B = (6, 3)$ \\
					\hline
			\end{tabular}
	\endgroup
\end{center}
La curva de contrato:
	$$RMS^A = RMS^B \Longrightarrow x_A = x_B$$
además
	$$x_A + x_B = \overline{x} \Longrightarrow x_A + x_A = \overline{x} \Longrightarrow x_A = \frac{\overline{x}}{2} = x_B$$
	\begin{center}
		\begin{tikzpicture}[scale=0.8]
				% Formato de CAJA
					\draw[->] (0,0) node[align=center, below left] {\footnotesize $O_A$} -- (0,8) node[align=center, above] {\footnotesize $x_{2}^{A}$};
				\draw[->] (0,0) -- (8,0) node[align=center, right] {\footnotesize $x_{1}^{A}$};
				
					\draw[->] (7,7) node[align=center, above right] {\footnotesize $O_B$} -- (-1,7) node[align=center, left] {\footnotesize $x_{2}^{B}$};
				\draw[->] (7,7) -- (7,-1) node[align=center, below] {\footnotesize $x_{2}^{B}$};
						
			% Curvas de indiferencia
				\draw [blue] (1.4,6.9) .. controls (2.84,5.7) and (3.8,5.2) .. (5.6,4.8);
				\draw [red] (1.4,6.15) .. controls (3.03,5.8) and (3.99,5.38) .. (5.6,4.05);
				
				\draw [blue] (1.4,5) .. controls (2.84,3.8) and (3.8,3.3) .. (5.6,2.9);
				\draw [red] (1.4,4.25) .. controls (3.03,3.9) and (3.99,3.48) .. (5.6,2.15);
				
				\draw [blue] (1.4,3.1) .. controls (2.84,1.9) and (3.8,1.4) .. (5.6,1);
				\draw [red] (1.4,2.35) .. controls (3.03,2) and (3.99,1.58) .. (5.6,0.25);
			% Curvas de contrato
				\draw [purple, very thick] (3.5,0) node [below, scale= 0.6mm] {$\frac{\overline{x}}{2}$} -- (3.5,7);
				
			% Flechas
				\node[draw, single arrow,
					minimum height=28mm, minimum width=1mm,
					single arrow head extend=1.5mm,
					anchor=west, blue, fill=blue, scale=0.5, rotate=33] at (1.5,5.1) {};
				\node[draw, single arrow,
					minimum height=28mm, minimum width=1mm,
					single arrow head extend=1.5mm,
					anchor=west, red, fill=red, rotate=-147, scale=0.5] at (5.5,2.1) {};
		\end{tikzpicture}
	\end{center}

Probamos si las dotaciones iniciales son ESP.
	$$x_{A} = x_{B} = \frac{\overline{x}}{2} \Longrightarrow	1 \neq 6 \neq \frac{7}{2}$$
Se comprueba que la dotación inicial
	\begin{center}
		\begin{tikzpicture}[scale=0.8]
			% Formato de CAJA
				\draw[->] (0,0) node[align=center, below left] {\footnotesize $O_A$} -- (0,8) node[align=center, above] {\footnotesize $x_{2}^{A}$};
				\draw[->] (0,0) -- (8,0) node[align=center, right] {\footnotesize $x_{1}^{A}$};
			
				\draw[->] (7,7) node[align=center, above right] {\footnotesize $O_B$} -- (-1,7) node[align=center, left] {\footnotesize $x_{2}^{B}$};
				\draw[->] (7,7) -- (7,-1) node[align=center, below] {\footnotesize $x_{2}^{B}$};
			
			% Curvas de indiferencia
				\draw [blue] (1.4,6.9) .. controls (2.84,5.7) and (3.8,5.2) .. (5.6,4.8);
				\draw [red] (1.4,6.15) .. controls (3.03,5.8) and (3.99,5.38) .. (5.6,4.05);
				
				\draw [blue] (1.4,5) .. controls (2.84,3.8) and (3.8,3.3) .. (5.6,2.9);
				\draw [red] (1.4,4.25) .. controls (3.03,3.9) and (3.99,3.48) .. (5.6,2.15);
				
				\draw [blue] (1.4,3.1) .. controls (2.84,1.9) and (3.8,1.4) .. (5.6,1);
				\draw [red] (1.4,2.35) .. controls (3.03,2) and (3.99,1.58) .. (5.6,0.25);
				
			% Curvas de contrato
				\draw [purple, very thick] (3.5,0) node [below, scale= 0.6mm] {$\frac{\overline{x}}{2}$} -- (3.5,7);
			
			% Flechas
				\node[draw, single arrow,
						minimum height=28mm, minimum width=1mm,
						single arrow head extend=1.5mm,
						anchor=west, blue, fill=blue, scale=0.5, rotate=33] at (1.5,5.1) {};
				\node[draw, single arrow,
						minimum height=28mm, minimum width=1mm,
						single arrow head extend=1.5mm,
						anchor=west, red, fill=red, rotate=-147, scale=0.5] at (5.5,2.1) {};
			
			% Intersección
				\draw[dashed] (1,7) node[above] {\footnotesize $x_{1}^{B}=6$} -- (1,0) node[below] {\footnotesize $x_{1}^{A}=1$};
				\draw[dashed] (0,4) node[left] {\footnotesize $x_{2}^{A}=4$} -- (7,4)node[right] {\footnotesize $x_{2}^{B}=3$};
			
			% Punto
				\draw[black, fill=black] (1,4) circle[radius=0.08] node[align=center, above left] {\footnotesize $w$};
		\end{tikzpicture}
	\end{center}

Con la inclusión de precios, la situación es la siguiente:
	$$RMS^A = RMS^B = \frac{p_x}{p_y}$$
	$$RMS^A=\frac{1}{2x_A}=\frac{p_x}{p_y} \quad , \quad RMS^B=\frac{1}{2x_B}=\frac{p_x}{p_y} \Longrightarrow x_A = x_B = \frac{p_y}{2p_x}$$
	
	$$
		\begin{array}{ccc}
			p_xx_A + p_yy_A = p_x + 4p_y & {} &p_xx_B + p_yy_B = 6p_x + 3p_y\\[0.4cm]
			p_x\frac{p_y}{2p_x} + p_yy_A = p_x + 4p_y & {} &p_x\frac{p_y}{2p_x} + p_yy_B = 6p_x + 3p_y\\[0.4cm]
			\frac{p_y}{2} + p_yy_A = p_x + 4p_y & {} & \frac{p_y}{2} + p_yy_B = 6p_x + 3p_y \\[0.4cm]
			y_{A}^* = \frac{2p_x + 7p_y}{2p_y} & {} & y_{B}^* = \frac{12p_x + 5p_y}{2p_y}
		\end{array}
	$$
	
	$$y_{A}^* + y_{B}^* = 4 + 3 = 7 \Longrightarrow \frac{2p_x + 7p_y}{2p_y} + \frac{12p_x + 5p_y}{2p_y} = 7  \Longrightarrow  \frac{p_x}{p_y} = \frac{1}{7}$$
	
	\begin{gather*}
		\therefore y_{A}^* = \frac{51}{14}  \Longrightarrow x_{B}^* = \frac{7}{2}\\[0.4cm]
		\therefore y_{b}^* = \frac{47}{14}  \Longrightarrow x_{B}^* = \frac{7}{2}
	\end{gather*}

	\begin{center}
		\begin{tikzpicture}[scale=0.8]
			% Formato de CAJA
				\draw[->] (0,0) node[align=center, below left] {\footnotesize $O_A$} -- (0,8) node[align=center, above] {\footnotesize $x_{2}^{A}$};
				\draw[->] (0,0) -- (8,0) node[align=center, right] {\footnotesize $x_{1}^{A}$};
				
				\draw[->] (7,7) node[align=center, above right] {\footnotesize $O_B$} -- (-1,7) node[align=center, left] {\footnotesize $x_{2}^{B}$};
				\draw[->] (7,7) -- (7,-1) node[align=center, below] {\footnotesize $x_{2}^{B}$};
			
			% Curvas de indiferencia
				\draw [blue] (1.4,6.9) .. controls (2.84,5.7) and (3.8,5.2) .. (5.6,4.8);
				\draw [red] (1.4,6.15) .. controls (3.03,5.8) and (3.99,5.38) .. (5.6,4.05);
				
				\draw [blue] (1.4,5) .. controls (2.84,3.8) and (3.8,3.3) .. (5.6,2.9);
				\draw [red] (1.4,4.25) .. controls (3.03,3.9) and (3.99,3.48) .. (5.6,2.15);
				
				\draw [blue] (1.4,3.1) .. controls (2.84,1.9) and (3.8,1.4) .. (5.6,1);
				\draw [red] (1.4,2.35) .. controls (3.03,2) and (3.99,1.58) .. (5.6,0.25);
			
			% Curvas de contrato
				\draw [purple, very thick] (3.5,0) -- (3.5,7);
			
			% Intersección
				\draw[dashed] (1,7) node[above] {\footnotesize $x_{1}^{B}=6$} -- (1,0) node[below] {\footnotesize $x_{1}^{A}=1$};
				\draw[dashed] (0,4) node[left] {\footnotesize $x_{2}^{A}=4$} -- (7,4)node[right] {\footnotesize $x_{2}^{B}=3$};
			
			% Punto
				\draw[black, fill=black] (1,4) circle[radius=0.08] node[align=center, above left] {\footnotesize $w$};
				\draw[black, fill=black] (3.5,3.58) circle[radius=0.08] node[align=center, below left] {\footnotesize $EGW$};
			
			% Recta presupuestaria
				\draw (0,5.26) -- (7,1.9) node[right] {\footnotesize $RP$};
			
			% Intersección
				\draw[dashed] (3.5,7) node[above] {\footnotesize $x_{1}^{B}=3.5$} -- (3.5,0) node[below] {\footnotesize $x_{1}^{A}=3.5$};
				\draw[dashed] (0,3.58) node[left] {\footnotesize $x_{2}^{A}=3.64$} -- (7,3.58)node[right] {\footnotesize $x_{2}^{B}=3.36$};
		\end{tikzpicture}
	\end{center}
	\item Considérese una sociedad compuesta por dos individuos -los individuos 1 y 2-  y cuya frontera de posibilidades de utilidad (\emph{FPU}) es la dada por la función $u^1+(u^2)^2 = 100$, donde $u^1$ denota la utilidad del agente 1 y $u^2$ la del agente 2. Además, ambos individuos creen que la asignación de recursos ideal en la que se obtiene maximizando una apropiada función de bienestar social (\emph{FBS}) utilitarista ponderada o Bergsoniana. En particular,
		\begin{enumerate}[a)]
			\item El individuo 1 cree que la mejor distribución social posible del bienestar es la dada por el vector de utilidad $\left( u^1, u^2\right) =(75, 5)$ y confronta esta solución con una \emph{FBS} constituida por la suma ponderada de la utilidades, viendo confirmado su aseveración. ¿Cuál es la \emph{FBS} según el criterio de este individuo 1?
			\item El individuo 2 cree, por otro parte, que la mejor distribución social posible del bienestar es $\left( u^1, u^2\right) =(19, 9)$. ¿Cuál es la \emph{FBS} según el criterio del individuo 2?
		\end{enumerate}
			\textbf{\LARGE Solución}\\
				\begin{itemize}
	\item Pregunta 9a:
			Para el agente $A$:
				\begin{align*}
					& \text{Max } \quad U_A = \alpha \ln(x_A) + (1-\alpha)\ln(x_A) \\[0.2cm]
					& \begin{array}{ll}
						\text{s.a: } & p_xx_A + p_yy_A = \overline{y}p_y + \overline{x}p_x \qquad (\overline{y}=1, \overline{x}=0)
					\end{array}
				\end{align*}
			
			Se forma el lagrange y se aplican las condiciones de primer orden:
			$$ \mathscr{L} =  \ln(x_A) + (1-\alpha)\ln(x_A) - \lambda\left[p_y - p_xx_A - p_yy_A  \right]$$
			
				$$\left.
					\begin{array}{l}
						\frac{\partial \mathscr{L}}{\partial x_{A}} = \frac{\alpha}{x_A} - \lambda p_x= 0\\[0.4cm]
						\frac{\partial \mathscr{L}}{\partial y_{A}} = \frac{(1-\alpha)}{y_A} - \lambda p_y=0
					\end{array}
				  \right\} \Longrightarrow 
				    \begin{array}{ccc}
						\frac{\alpha}{x_Ap_x} & = & \frac{(1-\alpha)}{y_Ap_y}  \\[0.3cm]
				  		y_A & = & \frac{(1-\alpha)}{\alpha p_y}x_Ap_x
				    \end{array}$$
			
				\begin{align*}
					p_xx_A + p_yy_A & = p_y\\[0.3cm]
					p_xx_A + p_y\left( \frac{(1-\alpha)}{\alpha p_y}x_Ap_x\right)  & = p_y\\[0.3cm]
					x_{A}^* & = \frac{\alpha p_y}{p_x}\\[0.3cm]
					y_{A}^* & = 1 - \alpha
				\end{align*}
			
			Para el agente $B$:
				\begin{align*}
					& \text{Max } \quad U_B = \text{Min} \left\lbrace x_B, y_B\right\rbrace \\[0.2cm]
					& \begin{array}{ll}
						\text{s.a: } & p_xx_A + p_yy_A = \overline{y}p_y + \overline{x}p_x  \qquad (\overline{y}=0, \overline{x}=1)
					  \end{array}
				\end{align*}
			
			Como $x$ e $y$ son complementarios perfectos
				$$x_B = y_B$$
			Y reemplazamos en la recta presupuestaria
				\begin{align*}
					p_xx_B + p_yy_B & = p_x\\[0.3cm]
					p_xx_B + p_yx_B  & = p_x\\[0.3cm]
					x_B\left( p_x + p_y\right) &= p_x\\[0.3cm]
					x_{B}^* & = \frac{p_x}{p_x+p_y} = y_{B}^*
				\end{align*}
			El criterio de la condición de factibilidad nos indica que los mercados se limpian; es decir, la demanda debe ser igual a la oferta (dotación)
				\begin{itemize}
					\item $x_a + x_B = \overline{x}_A + \overline{x}_B \Longrightarrow (1 - \alpha) + \frac{p_x}{p_x+p_y} = 0 + 1$
					\item $y_a + y_B = \overline{x}_A + \overline{x}_B \Longrightarrow \frac{\alpha p_y}{p_x} + \frac{p_x}{p_x+p_y} = 1 + 0$
				\end{itemize}
			Usando la expresión de $y$, los precios relativos son:
				$$\frac{p_y}{p_x} = \frac{1-\alpha}{\alpha}$$
	\item Pregunta 9b:
			$$\nexists \enskip p < 0 \enskip \Longrightarrow \enskip \frac{1 - \alpha}{\alpha} > 0 \enskip \Longrightarrow \enskip  \alpha \in <0,1>$$
\end{itemize}
	\item Consideremos una economía compuesta por dos empresas - las empresas 1 y 2-, dos consumidores - los individuos 1 y 2- y dos bienes -los bienes 1 y 2. Los consumidores tienen preferencias representadas por $u^1\left( q_{1}^{1},q_{2}^{1}\right) = \sqrt{q_{1}^{1}q_{2}^{1}}$ y $u^2\left( q_{1}^{2},q_{2}^{2}\right) = \frac{1}{2}\ln(q_{1}^{2})+\frac{1}{4}\ln(q_{2}^{2})$, respectivamente. A su vez, la empresa 1 produce el bien 1 según la tecnología $q_1=K_{1}^{1/4}L_{1}^{1/4}$, mientras que la empresa 2 produce el bien 2 a través de la tecnología $q_2=K_{2}^{1/3}L_{2}^{1/3}$. Las dotaciones de factores son las dadas por el vector $(K, L)\gg 0$, los precios de los bienes 1 y 2 son $p_1$ y $p_2$, respectivamente, y los de los factores $K$ y $L$, $r$ y $w$. La empresa 1 es propiedad del consumidor 1 y la 2 del consumidor 2. Supongamos, por último que el Estado representa sus preferencias mediante la función de utilidad o bienes social $W$. Teniendo en cuenta que en la asignación de quilibrio walrasiano y sobre la frontera de posibilidades de producción (\emph{FPP}) se puede definir un máximo de $W$ para un valor único de $q_1$, determínese este valor de $q_1$ en función de $K$ y $L$:
		\begin{enumerate}[a)]
			\item Si la función de bienestar social $(FBS)$ es $W\left( u^1,u^2\right) = u^1\left( q_{1}^{1},q_{2}^{1}\right)$; es decir, es una función que privilegia al consumidor 1.
			\item Si la $FBS$ es $W\left( u^1,u^2\right) = u^2\left( q_{1}^{2},q_{2}^{2}\right)$, en cuyo caso privilegia al consumidor 2.
		\end{enumerate}
			\textbf{\LARGE Solución}\\
				El punto $E$ indica la situación inicial de ambos consumidores. En este punto, el agente $A$ tiene una dotación de $(1,2)$; mientras que el $B$ una de $(2,1)$. La curva de contrato viene dada por la línea gruesa y el área morada señalada en el siguiente gráfico.

\begin{center}
	\begin{tikzpicture}[scale=1.7]
		% Formato de CAJA
		\draw[->] (0,0) node[align=center, below left] {\footnotesize $O_A$} -- (0,4) node[align=center, above] {\footnotesize $y_A$};
		\draw[->] (0,0) -- (4,0) node[align=center, right] {\footnotesize $x_{A}$};
		
		\draw[->] (3,3) node[align=center, above right] {\footnotesize $O_B$} -- (-1,3) node[align=center, left] {\footnotesize $x_{B}$};
		\draw[->] (3,3) -- (3,-1) node[align=center, below] {\footnotesize $y_{B}$};
		
		% 
		
		% Curvas de indiferencia
		\draw (0.5,3) -- (0.5,0.5) -- (3,0.5);
		\draw (2.5,3) -- (2.5,2.5) -- (3,2.5);
		
		\draw (2,3) -- (3,2);
		\draw (0,2.5) -- (2.5,0);
		\draw (0.7,3) -- (3,0.7);
		\draw (0,1) -- (1,0);
		
		% Curva de contrato       
		\draw [purple, very thick] (0,0)  -- (1,1);
		\draw [purple, very thick] (2,2)  -- (3,3);
		\draw [purple, very thick] (1,1)  -- (1,3) -- (2,3) -- (2,2) -- (3,2) -- (3,1) -- (1,1);
		\fill [color=purple!10](1,1)  -- (1,3) -- (2,3) -- (2,2) -- (3,2) -- (3,1) -- (1,1);
		
		% Punto
		\draw[black, fill=black] (1,2) circle[radius=0.04] node[align=center, right, scale = 0.25mm] {\footnotesize $E(1,2)$};
	\end{tikzpicture}
\end{center}
\end{enumerate}
%------------------------------------------------------------------------------------
\end{document}		
%====================================================================================