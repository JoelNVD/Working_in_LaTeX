\begin{enumerate}[a)]
	\item La pendiente de la \emph{FBS} $u^2(u^1) = \sqrt{100-u^2}$ es $\frac{\partial u^2}{\partial u^1} = -\frac{1}{2\sqrt{100-u^2}}$ y, evaluada en el punto $(u^1,u^2) = (75,5)$, es (el valor absoluto) igual a $\frac{1}{10}$. Por otra parte, dada la \emph{FBS} de Bergson que ``tiene in mente'' el individuo 1 y que es del tipo $W(u^1,u^2) = \theta^1u^1 + \theta^2u^2$, su pendiente es $-\frac{\theta^1}{\theta^2}$. Entonces ha de verificarse.
		$$\frac{\theta^1}{\theta^2}=\frac{1}{10}$$
	Por otra parte, el punto de máximo bienestar debe ser tal que el problema
		\begin{equation}
			\begin{aligned}
				& \M \limits_{(u^1,u^2)} \quad \theta^1u^1 + \theta^2u^2\\
				& \begin{array}{ll}
					\text{s.a. } & u^1 +(u^2)^2 = 100
				\end{array}
			\end{aligned} \label{eq1}
		\end{equation}
	dé como resultado el reparto de bienestar $(u^1,u^2) = (75,5)$. Las CPO de este problema (\ref{eq1}) son 
		\begin{gather}
			\theta^1 - \lambda = 0 \label{eq2}\\
			\theta^2 - 2 \lambda u_2 = 0 \label{eq3}\\
			\intertext{y}
			100 - u_1 - u_{2}^2 = 0
		\end{gather}
	De las CPO (\ref{eq2}) y (\ref{eq3}) resulta $\theta^1=\frac{\theta^2}{2u^2}$, y dado que $u^2=5$, se obtiene $\theta^2=10\theta^2$. En definitivo,
		$$W(u^1, u^2) = \theta^1u^1+10\theta^1u^2, \theta^1>0$$
	es la \emph{FBS} utilitarista ponderada a o de Bergson que implícitamente está utilizando el individuo 1.
	\item Operando de manera análoga a la del apartado 1, se obtiene
		$$W(u^1, u^2) = \theta^1u^1+18\theta^1u^2, \theta^1>0$$
	como \emph{FBS} según el criterio del individuo 2.
\end{enumerate}