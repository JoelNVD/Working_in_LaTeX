\begin{enumerate}[a)]
	\item Dadas las preferencias $Cobb-Douglas$ de los consumidores, las funciones de demanda marshalliana del consumidor 1 son las dadas por
				$$\left[ q_{1}^{1}\left( p,m^1\right), q_{2}^{1}\left( p,m^1\right) \right] = \left( \frac{m^1}{2p_1},\frac{m^1}{2p_2}\right) $$
		y la de los consumidores 2 por
				$$\left[ q_{1}^{2}\left( p,m^2\right), q_{2}^{2}\left( p,m^2\right) \right] = \left( \frac{2m^2}{3p_1},\frac{m^2}{3p_2}\right) $$
		Por otra parte, los rendimientos a escala de ambas empresas son decrecientes, con lo cual los beneficios obtenido en equilibrio serán positivos. Con respecto a la empresa 1, el problema
				\begin{align*}
					\M \limits_{(K_1,L_1)} \Pi_1 & = p_1q_1 -rK_1 - wL_1\\
												 & = p_1K_{1}^{1/4}L_{1}^{1/4} -rK_1 - wL_1
				\end{align*}
		requiere que
				\begin{gather}
					\frac{1}{4}p_1K_{1}^{-3/4}L_{1}^{1/4} -r = 0 \label{eq4}\\
					\intertext{y}
					\frac{1}{4}p_1K_{1}^{1/4}L_{1}^{-3/4} -w = 0 \label{eq5}
				\end{gather}
		Despejando $p_1$ en (\ref{eq4}) y (\ref{eq5}) e igualando, resulta $rK_1 = wL_1$ o, lo uqe es lo mismo,
				$$\frac{K_1}{L_1} = \frac{w}{r}$$
		Análogamente, la maximización del beneficio de la empresa 2,
				$$\M \limits_{(K_1,L_1)} \Pi_1 = p_2K_{2}^{1/3}L_{2}^{1/3} -rK_2 - wL_2$$
		exige que
				$$\frac{1}{3}p_2K_{2}^{-2/3}L_{2}^{1/3} - r =0$$
		y
				$$\frac{1}{3}p_2K_{2}^{1/3}L_{2}^{2-/3} - w =0$$
		lo cual da lugar a $rK_2 = wL_2$; es decir, a 
				$$\frac{K_2}{L_2} = \frac{w}{r}$$
		La $FPP$ de la economía es la relación que determina $q_2$ en función de $q_1$ (o viceversa), dadas las cantidades de recursos $k$ y $L$ con las que cuenta dicha economía. Se obtiene, por tanto,a partir de
			\begin{gather*}
				K = K_1 + K_2\\
				L = L_1 + L_2
				\intertext{y}
				\frac{K_1}{L_1} = \frac{K_2}{L_2} = \frac{w}{r}
			\end{gather*}
		Dado que
			$$\frac{K_1}{L_1} = \frac{K - K_1}{L - L_1} \text{ o } \frac{K_2}{L_2}= \frac{K - K_2}{L - L_2}$$
		Entonces
			\begin{gather*}
				K_1 = \frac{K}{L}L_1 \text{ o } K_2 = \frac{K}{L}L_2 \label{eq6}
			\end{gather*}
		A partir de (\ref{eq6}) se tiene que 
			$$
				\begin{array}{c|c}
					q_1 = K_{1}^{1/4}L_{1}^{1/4} = \left( \frac{K}{L}L_1\right)^{1/4} L_{1}^{1/4} = \left( \frac{K}{L}\right)^{1/4}L_{1}^{1/2} & q_2 = K_{2}^{1/3}L_{2}^{1/3} = \left( \frac{K}{L}L_2\right)^{1/3} L_{2}^{1/3} = \left( \frac{K}{L}\right)^{1/3}L_{3}^{2/3}\\[0.3cm]
					L_1 = \left( \frac{L}{K}\right)^{1/2}q_{1}^{2} & L_2 = \left( \frac{L}{K}\right) ^{1/2}q_{2}^{3/2}
				\end{array}
			$$
		Finalmente, la condición $L = L_1+L_2 = \left( \frac{L}{K}\right) ^{1/2}\left( q_{1}^{2}+q_{2}^{3/2}\right) $ se puede reescribir como
			$$q_{1}^{2}+q_{2}^{3/2} = \frac{L}{\left( \frac{L}{K}\right) ^{1/2}} = \left( \frac{K}{L}L_2\right)^{1/2}L = (KL)^{1/2}$$
		es decir, como
			\begin{gather}
				\left\lbrace \left(q_1,q_2 \right) \in \mathbb{R}_{+}^{2} \mid \sqrt{KL} - q_{1}^{2} - q_{2}^{3/2} = 0 \right\rbrace \label{eq7}
			\end{gather}
		que es justamente la $FPP$ de la economía\\
		
		Si el dueño de la empresa 1 es el consumidor 1 y el de la empresa 2 el consumidor 2, entonces la renta del consumidor 1 es $m^1 = wL_1+rK_1+\pi_1=p_1q_1$, mientras que la del consumidor 2 es $m^2 = wL_2+rK_2+\pi_2=p_2q_2$. Con ello, las demandas del individuos 1 y 2 pasan a ser
			$$
				\begin{array}{c|c}
					\left( q_{1}^{1},q_{2}^{1}\right) = \left( \frac{m^1}{2p_1},\frac{m^1}{2p_2}\right) = \left( \frac{q_1}{2}, \frac{p_1q_1}{2p_2}\right) & \left( q_{1}^{2},q_{2}^{2}\right) = \left( \frac{2m^2}{3p_1},\frac{m^2}{3p_2}\right) = \left( \frac{2p_2q_2}{3p_1}, \frac{q_2}{3}\right)
				\end{array}
			$$
		Por lo tanto, y de acuerdo con la ley de Walras, la condición de equilibrio en el mercado de uno de los bienes (por ejemplo, el del bien 1),
			$$q_{1}^{1} + q_{2}^{1} = q_{1} \Rightarrow \frac{q_1}{2}+ \frac{p_1q_1}{2p_2} = q_{1}$$
		da lugar a $\frac{p_2}{p_1} = \frac{3}{4}\frac{q_1}{q_2}$ como precio de equilibrio, con lo cual la correspondiente asignación de equilibrio es la dada por 
			$$\left[ \left( q_{1}^{1},q_{2}^{1}\right) , \left( q_{1}^{2},q_{2}^{2}\right) \right] = \left[ \left(\frac{q_1}{2}, \frac{p_1q_1}{2p_2} \right) , \left( \frac{2p_2q_2}{3p_1}, \frac{q_2}{3}\right) \right] $$
		En definitiva, el equilibrio walrasiano es el conjunto de precios, el plan de producción y los consumos de los individuos dados por el conjunto
			$$
			\begin{aligned}
				&\left\{\left(p_{1}, p_{2}\right),\left[\left(q_{1}, K_{1}, L_{1}\right),\left(q_{2}, K_{2}, L_{2}\right)\right],\left[\left(q_{1}^{1}, q_{2}^{1}\right),\left(q_{1}^{2}, q_{2}^{2}\right)\right]\right\}=\\
				=&\left\{\left(p_{1}, \frac{3}{4} \frac{q_{1}}{q_{2}} p_{1}\right),\left[\left(q_{1},\left(\frac{K}{L}\right)^{\frac{1}{2}}\left(q_{1}\right)^{2},\left(\frac{L}{K}\right)^{\frac{1}{2}}\left(q_{1}\right)^{2}\right),\right.\right.\\
				&\left.\left.\left(q_{2},\left(\frac{K}{L}\right)^{\frac{1}{2}}\left(q_{2}\right)^{\frac{3}{2}},\left(\frac{L}{K}\right)^{\frac{1}{2}}\left(q_{2}\right)^{\frac{3}{2}}\right)\right],\left[\left(\frac{q_{1}}{2}, \frac{2 q_{2}}{3}\right),\left(\frac{q_{1}}{2}, \frac{q_{2}}{3}\right)\right]\right\} .
			\end{aligned}
			$$
		A partir de aquí, si la FBS es
			$$
			W\left(u^{1}, u^{2}\right) \equiv u^{1}\left(q_{1}^{1}, q_{2}^{1}\right)=\left(\frac{q_{1}}{2}\right)^{\frac{1}{2}}\left(\frac{2 q_{2}}{3}\right)^{\frac{1}{2}}
			$$
		El problema
			$$\M \limits_{(q_1,q_2)} W\left(u^{1}, u^{2}\right)=\sqrt{\frac{q_{1} q_{2}}{3}}$$
		es equivalente al problema 
			\begin{gather}
				\M \limits_{q_1} \left(\frac{1}{3}\right)^{\frac{1}{2}}\left(q_{1}\right)^{\frac{1}{2}}\left[(K L)^{\frac{1}{2}}-\left(q_{1}\right)^{2}\right]^{\frac{1}{3}} \label{eq8}
			\end{gather}
		una vez tenida en cuenta la FPP (\ref{eq7}). Y la CPO del problema (\ref{eq8})
			$$
			\left(\frac{1}{3}\right)^{\frac{1}{2}}\left\{\frac{1}{2}\left(q_{1}\right)^{-\frac{1}{2}}\left[(K L)^{\frac{1}{2}}-\left(q_{1}\right)^{2}\right]^{\frac{1}{3}}-\frac{2}{3}\left(q_{1}\right)^{\frac{3}{2}}\left[(K \mid L)^{\frac{1}{2}}-\left(q_{1}\right)^{2}\right]^{-\frac{2}{3}}\right\}=0
			$$
		da lugar a 
			\begin{gather}
				q_{1}=\left(\frac{3}{5}\right)^{\frac{1}{2}}(K L)^{\frac{1}{4}} \label{eq9}
			\end{gather}
		Finalmente, insertando (\ref{eq9}) en la FPP (\ref{eq7}), se obtiene
			$$q_{2}=\left[(K L)^{\frac{1}{2}}-\left(q_{1}\right)^{2}\right]^{\frac{2}{3}}=\left(\frac{2}{5}\right)^{\frac{2}{3}}(K L)^{\frac{1}{3}}$$
	\item En este caso, la FBS es
			$$ W\left(u^{1}, u^{2}\right) \equiv u^{2}\left(q_{1}^{2}, q_{2}^{2}\right)=\left(\frac{q_{1}}{2}\right)^{\frac{1}{2}}\left(\frac{q_{2}}{3}\right)^{\frac{1}{4}}$$
		y el problema consiste en
			\begin{gather}
				\M \limits_{q_1} \left(\frac{1}{2^{\frac{1}{2}} \cdot 3^{\frac{1}{4}}}\right)\left(q_{1}\right)^{\frac{1}{2}}\left[(K L)^{\frac{1}{2}}-\left(q_{1}\right)^{2}\right]^{\frac{1}{6}} \label{eq10}
			\end{gather}
		La CPO del problema (\ref{eq10}) es
			$$ \frac{1}{2^{\frac{1}{2}} \cdot 3^{\frac{1}{4}}}\left\{\frac{1}{2}\left(q_{1}\right)^{-\frac{1}{2}}\left[(K L)^{\frac{1}{2}}-\left(q_{1}\right)^{2}\right]^{\frac{1}{6}}-\frac{1}{3}\left(q_{1}\right)^{\frac{3}{2}}\left[(K L)^{\frac{1}{2}}-\left(q_{1}\right)^{2}\right]^{-\frac{5}{6}}\right\}=0 $$
		y da lugar a
			$$ q_{1}=\left(\frac{3}{5}\right)^{\frac{1}{2}}(K L)^{\frac{1}{4}} $$
		Por lo tanto,
			$$ q_{2}=\left[(K L)^{\frac{1}{2}}-\left(q_{1}\right)^{2}\right]^{\frac{2}{3}}=\left(\frac{2}{5}\right)^{\frac{2}{3}}(K L)^{\frac{1}{3}} $$
		Es decir, el nivel socialmente óptimo de $q_{1}\left(\mathrm{y} q_{2}\right)$ es independiente de quien sea el individuo que resulte privilegiado en la FBS. La diferencia entre el contexto definido por la FBS del apartado 1 y la FBS del apartado 2 radica en que, en el primero de ellos los recursos van a parar al individuo 1 , mientras que en el segundo van a manos del agente 2.
\end{enumerate}

