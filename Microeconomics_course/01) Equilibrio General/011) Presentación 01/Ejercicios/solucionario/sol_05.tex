Analizando al agente $A$, se debe cumplir la siguiente relación:
	\begin{itemize}
		\item $RMS_{A}\left(x_{1}^{A},\enskip x_{2}^{A} \right) = \frac{p_1}{p_2}\Longrightarrow \frac{x_{2}^{A}}{2x_{1}^{A}} = \frac{1}{1}$
		\item $p_{1}x_{1}^{A}+p_{2}x_{2}^{A} = p_{1}w_{1}^{A} + p_{2}w_{2}^{A} \Longrightarrow x_{1}^{A} + x_{2}^{A} = 16 + 4$
	\end{itemize}
Entonces 
	\begin{itemize}
		\item $x_{2}^{A}=2x_{1}^{A} \Longrightarrow  x_{1}^{A} + 2x_{1}^{A} = 3x_{1}^{A} = 16 + 4 = 20 \Longrightarrow x_{1}^{*A} = \frac{20}{3}$
		\item $x_{2}^{A}=2x_{1}^{A} = 2x_{1}^{*A} = \frac{20}{3} \Longrightarrow x_{2}^{*A} = \frac{40}{3}$
	\end{itemize}
La decisión optima de $A$:\\
	$$\therefore \left(x_{1}^{*A},\enskip x_{2}^{*A} \right) = \left(\frac{20}{3}, \enskip \frac{40}{3}\right)$$
Dada la dotación incial, para consumir lo anterior el condumiro $A$ tendrá que vender $\frac{28}{3}$ del bien 1 (una demanda neta de $\frac{20}{3} - 16$, negativa), y comprar $\frac{28}{3}$ del bien 2 (demanda neta $\frac{40}{3}-4$)\\

Analizando al agente $B$
			$$	\left.
					\begin{array}{l}
						\frac{2x_{2}^{B}}{x_{1}^{B}} = \frac{1}{1} \\ [.5cm]
						x_{1}^{B} + x_{2}^{B} = 4 + 10
					\end{array}
				\right\} \Longrightarrow \therefore \left(x_{1}^{*B},\enskip x_{2}^{*B} \right) = \left(\frac{28}{3}, \enskip \frac{14}{3}\right) $$
Comparando la deciión de $A$ y $B$. El consumidor $A$ necesita vender bien 1 y el $B$ necesita comprar, pero no en la misma cantidad. La demanda neta de bien 1 por parte de $B$ es de $\frac{14}{3}$, frente a los $\frac{28}{3}$ que $A$ quiere vender. Lo mismo, en sentido contrario ocurre en el bien 2.