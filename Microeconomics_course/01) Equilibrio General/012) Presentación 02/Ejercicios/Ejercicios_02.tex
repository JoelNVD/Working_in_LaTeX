%====================================================================================
% Preamble
%------------------------------------------------------------------------------------
\documentclass[10pt,a4paper]{article}

% Apartado de texto
\usepackage[utf8]{inputenc}
\usepackage[spanish]{babel}
\usepackage[T1]{fontenc}

% Apartado matemático
\usepackage{amsmath}
\usepackage{amsfonts}
\usepackage{amssymb}
\usepackage{mathtools}
\usepackage{mathrsfs}
\usepackage[libertine]{newtxmath}

% Apartadode no sangría
\usepackage{parskip}

% Aparatod posición del texto
\usepackage[left=2cm,right=2cm,top=2cm,bottom=2cm]{geometry}
\usepackage{fancyhdr}

\usepackage{graphicx}
\usepackage{xcolor}
\definecolor{cardinal}{rgb}{0.77, 0.12, 0.23}

% Apartado de columnas y filas: juntarlas o separarlas
\usepackage{multicol, multirow}

% Apartado differents label en listado
\usepackage[shortlabels]{enumitem}

% Apartado hyperres
\usepackage[hidelinks]{hyperref}

\usepackage{tcolorbox}
\newtcolorbox{nota}[1]{
	colback=cardinal!5!white,
	colframe=cardinal!75!black,
	fonttitle=\bfseries,
	title=#1
}

% Apartado dibujos
\usepackage{tikz, pgfplots}
\usetikzlibrary{positioning,calc,arrows}
\usetikzlibrary{shapes.arrows}
\usetikzlibrary{patterns}
\usetikzlibrary{babel} % Para que recono > y <, en inglés no se pone, pero en spañol sí
\usetikzlibrary{shadings,shadows}
\usetikzlibrary{matrix}

% Apartado cambio como por punto
\decimalpoint

%====================================================================================

%====================================================================================
% Body
%====================================================================================
% Title Page
%-----------
\textwidth=450pt \textheight=620pt \oddsidemargin=0in
\topmargin=-10pt
\pagestyle{fancy} \rhead{\scriptsize{\textbf{Código del Curso:} xxXXXXX} \\
	\textbf{Fecha:} XX/XX/2021 \& 2021-I \hspace{0.04cm}}
\lhead{\scriptsize{\textbf{Profesor:} José A. Valderrama}  \\
	\textbf{Curso:} Teoría Microeconómica II}
\newcommand{\re}[1]{\smallskip\textsf{\textbf{Respuesta}} \begin{sf}\\ #1 \end{sf} \bigskip}
\newcommand{\ay}[1]{ \scriptsize{\textsl{Hint: #1}}\normalsize{}}
\newcommand{\pr}[2]{\frac{\partial #1}{\partial #2}}

%------------------------------------------------------------------------------------
% Title
%---------
\begin{document}
	\begin{center}
		{\Large {\textbf{Práctica Dirigida N$^{\circ}$2}}}

		\textsc{Equilibrio General con Producción\\
				Frontera de Posibilidades de Producción}
		
	\end{center}
% EJERCICIOS---------------------------------------------------------------------------------
\begin{enumerate}
	\item Sea una economía Robinson Crusoe, donde la función de producción de cocos es $q=f(L)=\sqrt{L}$.
			\begin{enumerate}[a)]
				\item Determinar la función de demanda de trabajo, de oferta de cocos y el beneficio máximo de Robinson.
				\item Si la función de utilidad de Robinson es $c=u\left( R,c\right) =RC$, determina la función de demanda de cocos y de oferta de trabajo.
				\item Halle la relación de precios de equilibrio walrasiano, ¿cuáles son la cantidades de bien t de factor?¿cuál es el beneficio que obtiene Robinson?
			\end{enumerate}
			%\\[0.5cm]
			\textbf{\LARGE Solución}\\
				Se tiene la función inversa de demanda:  
		$$P = 30-q$$
La función inversa de oferta es:
		$$CM = 3 + q$$
Por tanto la cantidad de equilibrio se da cuando:
		$$CM = P$$
Por tanto:
	\begin{align*}
		3+q &= 30-q\\
		2q &= 27\\
		q^* &= 13.5
	\end{align*}
Hallando el precio de equilibrio:
	\begin{align*}
		p &= 30-q^*\\
		p &= 30-13.5\\
		p^* &= 16.5
	\end{align*}
	\item Obtenga la Frontera de Posibilidades de Producción (FPP) en una economía con dos outputs (1 y 2), obtenidos a partir de dos factores $(L y K)$, sabiendo que las funciones de producción de los dos outputs son las siguientes:
			$$q_x = L_x + K_x \quad q_y = L_y + K_y$$
	y que las dotaciones factoriales totales en la economía son:y que las dotaciones factoriales totales en la economía son:  $\overline{L}=2,\enskip \overline{K}=2$.\\[0.5cm]
		\textbf{\LARGE Solución}\\
			El tamaño de la caja será de $15$ unidades del bien 1 por 12 del bien 2. Se calcula las utilidades del punto inicial y de los propuestos:

	\begin{center}
		\begingroup
			\setlength{\tabcolsep}{10pt} % Default value: 6pt
			\renewcommand{\arraystretch}{1.5} % Default value: 1
				\begin{tabular}{ccccc}
						\hline
					Consumidor $A$ & {} & Consumidor $B$ & {} & Situación \\
						\hline
					$U_A(12,3) = 36$ & {} & $U_B(3,9) = 27$ & {} & Inicial\\
					$U_A(9,5)  = 45$ & {} & $U_B(6,7) = 42$ & {} & Ganan\\
					$U_A(8,9)  = 72$ & {} & $U_B(7,3) = 21$ & {} & A gana y B pierde\\
						\hline
				\end{tabular}
		\endgroup
	\end{center}

En general cualquier punto perteneciente al área de intercambio cumple con lo siguiente:
	\begin{gather*}
		U^{A}\left(x_{1}^{A}, x_{2}^{A}\right) \geq U^{A}\left(w_{1}^{A}, w_{2}^{A}\right) \\
		U^{B}\left(x_{1}^{B}, x_{2}^{B}\right) \geq U^{B}\left(w_{1}^{B}, w_{2}^{B}\right)
	\end{gather*}

En este ejercicio sería así:
	$$x_{1}^{A} \cdot x_{2}^{A} \geq 36 \qquad x_{1}^{B} \cdot x_{2}^{B} \geq 27$$

y sabiendo que 
	$$x_{1}^{B} = 36 - x_{1}^{A} \qquad x_{2}^{B} = 27 - x_{2}^{A}$$
	\vspace{-0.8cm}
gráficamente
\begin{center}
	\begin{tikzpicture}[scale=1.1]
		% Formación de la caja
			% Consumidor A
				\draw[->] (0.5,0.5) node[align=center, below left] {\footnotesize $O_A$} -- (0.5,4.5) node[align=center, above] {\footnotesize $x_{1}^{A}$};
				\draw[->] (0.5,0.5) -- (8.5,0.5) node[align=center, right] {\footnotesize $x_{2}^{A}$};
			
			%Consumidor B
				\draw[->] (8,4) node[align=center, above right] {\footnotesize $O_B$} -- (0,4) node[align=center, left] {\footnotesize $x_{1}^{B}$};
				\draw[->] (8,4) -- (8,0) node[align=center, below] {\footnotesize $x_{2}^{B}$};
		
		% Curvas de indiferencia
			\begin{axis}[
						hide axis,
						xmin=0, xmax=10, 
						ymin=0, ymax=10,
						ytick=\empty,
						]
				% Área sombreada
						\fill [pattern=crosshatch dots,pattern color=green!60!white] (axis cs:4.07,6.462) to [bend right=34] coordinate[pos=0.5] (l_i) (axis cs:8.43,1.57) to (axis cs:8.42,1.56) to [bend right=34] coordinate[pos=0] (l_i) (axis cs:4,6.461);
				
				% Curvas de indiferencia
					% Agente A
						\draw [blue] (axis cs:4,6.9) to [bend right=40] coordinate[pos=1] (l_i) (axis cs:9,1.5);
					% Agente B
						\draw [red] (axis cs:8.5,1) to [bend right=40] coordinate[pos=0.5] (l_i) (axis cs:3.6,6.5);
				
			\end{axis}
		
		% Punto
			\draw[black, fill=black] (5.775,0.89) circle[radius=0.05] node [above right] {$w$};
			\draw[black, fill=black] (4.775,1.95) circle[radius=0.05] node [above] {\footnotesize $(9,5)$};
			\draw[black, fill=black] (4.4,3.4) circle[radius=0.05] node [above] {\footnotesize (8,9)};
			
			% Intersección
				\draw [dashed] (5.77,0.5) node [below]{\footnotesize $x_{1}^{A}=12$} -- (5.77,4)node [above] {\footnotesize $x_{1}^{B}=3$};
				\draw [dashed] (0.5,0.89) node [left] {\footnotesize $3=x_{2}^{A}$}-- (8,0.89) node [right] {\footnotesize $x_{2}^{B}=9$};
	
		% Etiqueta de función de utilidad
			\node [above, blue] at (3.1,1.5) {$36=u_{A}$};
			\node [below, red]  at (5.5,3.1) {$u_{B}=27$};
	\end{tikzpicture}
\end{center}
	\item Suponga una economía en la que se producen dos mercancías (q1 y q2) a partir de dos inputs z1 y z2 de acuerdo
	con las siguientes funciones de producción:
			\begin{gather*}
				q_x = L_{A_x}^{\frac{1}{2}}K_{A_x}^{\frac{1}{2}}\\
				q_y = L_{B_y}^{\frac{1}{3}}K_{B_y}^{\frac{2}{3}}
			\end{gather*}	
	La cantidad de ambos inputs en la economía está limitada, de modo que sólo se dispone de $\overline{L}$ del primer
	factor y de $\overline{K}$ del segundo.
			\begin{enumerate}[a)]
				\item Halle el conjunto de asignaciones Pareto-eficientes en la producción.
				\item ¿Sería productivamente eficiente una asignación igualitaria de los factores entre las dos empresas?
			\end{enumerate}
		\textbf{\LARGE Solución}\\
			En el óptimo social, se cumple lo siguiente:
	$$CM + CEM = P$$
Entonces:
	\begin{align*}
		CM + CEM &= P\\
		\left(5+2q\right) + \left(0.5q\right) &= 30\\
		5 + 2.5q &= 30\\
		2.5q &= 25\\
		q^* &= 10
	\end{align*}
	\item En una economía se producen dos bienes $x$ e $y$, de acuerdo a las siguientes funciones de producción: $x = L_{x}^{\frac{1}{4}}K_{x}^{\frac{1}{4}}; y = L_{y}^{\frac{1}{2}}K_{y}^{\frac{1}{2}}$. La dotación inicial de factores está limitada, disponiéndose de $25$ unidades de trabajo y $25$ unidades de capital. El único consumidor que opera en esta economía tiene unas preferencias representadas por la función de utilidad $u = xy$. Determine:
			\begin{enumerate}[a)]
				\item La expresión del conjunto paretiano en producción
				\item La expresión de la frontera de posibilidades de producción.
				\item Los niveles de producción y los precios correspondientes al equilibrio competitivo.
				\item La asignación de factores correspondientes al óptimo de pareto.
			\end{enumerate}
		\textbf{\LARGE Solución}\\
			\begin{enumerate}[a)]
	\item Dado que las funciones son tipo \emph{Cobb-Douglas}, se igualan \emph{TMS}
			$$\frac{2\frac{3}{4}x_{1}^{-1/4}y_{1}^{1/4}}{2\frac{1}{4}x_{1}^{3/4}y_{1}^{-3/4}} = \frac{\frac{3}{4}x_{2}^{-1/4}y_{2}^{1/4}}{\frac{1}{4}x_{1}^{3/4}y_{2}^{-3/4}} \rightarrow \frac{y_1}{x_1} = \frac{y_2}{x_2} \rightarrow \frac{y_1}{x_1} = \frac{100- y_1}{100-x_1} \rightarrow y_1 = x_1 \text{ y } y_2 = x_2$$
		  Reemplazando en las F.U.
			$$
			\begin{array}{c|c}
				U_1 = 2x_{1}^{3/4}\left( x_1\right)^{1/4} & U_2 = x_{2}^{3/4}\left( x_{2}\right)^{1/4}\\[0.3cm]
				x_1=\frac{U_1}{2} & x_2 = U_2
			\end{array}
			$$
		  Reemplazando en la dotación de $x$ (o de $y$)
			$$x_1 + x_2 = 100 \longrightarrow \frac{U_1}{2} + U_2 = 100 \longrightarrow \therefore U_2 = 100 - \frac{U_1}{2}$$
		  Gráficamente
			\begin{center}
				\begin{tikzpicture}[samples = 100, scale=0.8]
					\draw[purple, domain = 0:10] plot({\x},{10-\x/2});
					\draw[<->] (0,10.5) node [above] {$U_2$} -- (0,5) -- (10.5,5) node [right] {$U_1$};
					\draw (4,9) node[right, purple] {$U_2 = 100 - \frac{U_1}{2}$};
				\end{tikzpicture}
			\end{center}
	\item Utilizando la siguiente expresión que se desprende de una \emph{Cobb-Douglas}\\
			$$U(x,y) = x^\alpha y^\beta$$
			$$
				\begin{array}{c|c}
					\highlight{x^M=\frac{\alpha I}{(\alpha + \beta)p_x}} & \highlight{y^M=\frac{\beta I}{(\alpha + \beta)p_y}}
				\end{array}
			$$
		  Entonces
		  	$$
		  		\begin{array}{c|c}
					x_{1}^{M} = \frac{\frac{3}{4}(20p_x+40p_y)}{\left( \frac{3}{4} + \frac{1}{4}\right)p_x} & x_{2}^{M} = \frac{\frac{3}{4}(80p_x+60p_y)}{\left( \frac{3}{4} + \frac{1}{4}\right)p_x}\\[0.3cm]
					\frac{15p_x+30p_y}{p_x} & \frac{60p_x+45p_y}{p_x}
		  		\end{array}
		  	$$
		  Reemplazando en la dotación de $x$ (o de $y$)
		  	$$x_A + x_B = 100 \longrightarrow \frac{15p_x+30p_y}{p_x} + \frac{60p_x+45p_y}{p_x} = 100 \longrightarrow \frac{p_y}{p_x} = \frac{1}{3}$$
		  Con el mismo procedimiento para $y$, se obtendrán los siguientes resultados:
		  	$$
		  		\begin{array}{c|c}
		  			x_{1}^* = 25 & y_{1}^*=25\\[0.3cm]
		  			x_{1}^* + x_2 = 100 & y_{1}^* + y_2 = 100\\[0.3cm]
		  			x_{2}^* = 75 & y_{2}^*=75
		  		\end{array}
		  	$$
		  Finalmente, reemplazando las dotaciones en las F.U.
		  		\begin{center}
		  			\begingroup
			  			\setlength{\tabcolsep}{10pt} % Default value: 6pt
			  			\renewcommand{\arraystretch}{1.5} % Default value: 1
				  			\begin{tabular}{ccc}
				  					\hline
				  				{} & $U_1$ & $U_2$ \\
					  				\hline
				  				Dotación inicial & $U_1 = 2(20)^{3/4}(40)^{1/4} \approx 47.57$ & $U_2 = (80)^{3/4}(60)^{1/4} \approx 74.45$\\
				  				Pareto & $U_1 = 2(25)^{3/4}(25)^{1/4} = 50$ & $U_2 = (75)^{3/4}(75)^{1/4} = 75$\\
					  				\hline
				  			\end{tabular}
		  			\endgroup
		  		\end{center}
	  	Podemos identificar 2 puntos, $A = (47.57, 74.45)$ y $B = (50,75)$. Gráficamente, ubicamos estos puntos en la \emph{FPU}.
	  		\begin{center}
	  			\begin{tikzpicture}[samples = 100, scale=0.8]
	  				\draw[purple, domain = 0:10] plot({\x},{10-\x/2});
	  				\draw[<->] (0,10.5) node [above] {$U_2$} -- (0,5) -- (10.5,5) node [right] {$U_1$};
	  				\draw (4,9) node[right, purple] {$U_2 = 100 - \frac{U_1}{2}$};
	  				\draw[dashed] (0,6.9) node [left] {74.45}-- (4.1,6.9) -- (4.1,5) node [below] {47.57};
	  				\draw[dashed] (0,7.5) node [left] {75}-- (5,7.5) -- (5,5) node [below] {50};
	  				\draw[black, fill=black] (4.1,6.9) circle[radius=0.05] node [above right] {$A$};
	  				\draw[black, fill=black] (5,7.5) circle[radius=0.05] node [above right] {$B$};
	  			\end{tikzpicture}
	  		\end{center}
\end{enumerate}
	\item Sea una economía con dos empresas. La empresa $1$ produce el bien $x$ de acuerdo con la función de producción $f_x(L_x)=L_{x}^{0.5}$, y la empresa $2$ produce el bien $y$ de acuerdo con la función de producción $f_y(l_y, q_y)=\left( \frac{L_y}{1+0.06q_{x}^{2}}\right)^{0.5}$, donde $L_x$ y $L_y$ son, respectivamente, las cantidades utilizadas en la producción de los bienes $x$ e $y$ del único factor existente en la economía ($L$), del que hay una dotación inicial de 800 unidades. El único consumidor de esta economía tiene unas preferencias representadas por la función de utilidad $u\left( c_x, c_y\right) = \ln(c_x) + \ln(c_y)$
			\begin{itemize}
				\item Obtenga la expresión de la Frontera de Posibilidades de Producción (FPP) de esta economía.
				\item Calcule las cantidades de producción óptimo paretianas de esta economía.
			\end{itemize}
		\textbf{\LARGE Solución}\\
			Para hallar el consumo óptimo de cuadros y cervezas de cada estudiante cuando vive solo. Se aplica lo siguiente:
	$$\frac{Umg_{X}^{i}}{Umg_{Y}^{i}} = \frac{P_{X}}{P_{Y}}$$
Resolviendo:
	\begin{align*}
		\frac{\frac{1}{3}X^{\frac{-2}{3}}_{i}Y^{\frac{2}{3}}_{i} }{\frac{2}{3}X^{\frac{1}{3}}_{i}Y^{\frac{-1}{3}}_{i}} &= \frac{P_{X}}{P_{Y}}\\
		\frac{Y_{i}}{2X_{i}} &= \frac{P_{X}}{P_{Y}}
	\end{align*}
Tenemos la restricción presupuestaria:
	$$X_{i}P_{X} + Y_{i}P_{Y} = M$$
Hallando $X_{i}$:
	\begin{align*}
		X_{i}P_{X} + 2X_{i}\frac{P_{X}}{P_{Y}}P_{Y} &= M\\
		3X_{i}P_{X} &= M\\
		X_{i}^* &= \frac{M}{3P_{X}}
	\end{align*}
Hallando $Y_{i}$:    
	\begin{align*}
		Y^*_{i} &= 2X_{i}^*\frac{P_{X}}{P_{Y}}\\
		Y^*_{i} &= 2\frac{M}{3P_{X}}\frac{P_{X}}{P_{Y}}\\
		Y^*_{i} &= \frac{2M}{3P_{Y}}
	\end{align*}
Reemplazando los datos: $M=3000$, $P_{X}=100$, $P_{Y}=0.2$, se tiene los consumos óptimos.
	\begin{align*}
		X^*_{i} &= 10\\
		Y^*_{i} &= 10000
	\end{align*}
	\item En una economía se producen dos bienes $x$ y $y$, de acuerdo con las siguientes funciones de producción $x=\frac{L_x}{2}$, y $y=L_{y}^{\frac{1}{2}}$. La dotación total de mano de obra es 100 unidades. Determine la ecuación de la FPP y la relación marginal de transformación.\\
		\textbf{\LARGE Solución}\\
			\begin{enumerate}[a)]
	\item El primer \emph{TFB} dice que un equilibrio general competitivo (\emph{EGC}) es Pareto-eficiente. Para tener un \emph{EGC}; sin embargo, es necesario que no haya presencia de bienes públicos, externalidades, información asimétrica, mercados incompletos y que la competencia perfecta esté asegurada.\\
	
		Se puede esbozar una prueba del primer \emph{TFB} de la siguiente manera
			\begin{center}
				\begingroup
					\setlength{\tabcolsep}{10pt} % Default value: 6pt
					\renewcommand{\arraystretch}{1.5} % Default value: 1
						\begin{tabular}{ccc}
							\hline
								\textbf{Optimización del consumidor} & $\rightarrow$ & \textbf{Eficiencia en el intercambio}\\
							\hline
								$TMS_{x,y}^A = p_x/p_y$ &\multirow{2}{*}{$\rightarrow$}	& \multirow{2}{*}{$TMS_{x,y}^A = TMS_{x,y}^B$}\\
								$TMS_{x,y}^B = p_x/p_y$	\\
							\hline
								\textbf{Minimización del costo} & $\rightarrow$ & \textbf{Eficiencia en al producción}\\
							\hline
								$TMST_{L,K}^x = w/r$ &\multirow{2}{*}{$\rightarrow$}	& \multirow{2}{*}{$TMST_{L,K}^x = TMST_{L,K}^y$}\\
								$TMST_{L,K}^y = w/r$ \\
							\hline
								\textbf{Maximización del beneficio} & $\rightarrow$ & \textbf{Eficiencia de alto nivel}\\
							\hline
								$p_x = CMg_x$ &\multirow{2}{*}{$\rightarrow$}	& \multirow{2}{*}{$TMT_{x,y} = \frac{CMg_x}{CMg_y} = \frac{p_x}{p_y}=TMS_{x,y}^A = TMS_{x,y}^B$}\\
								$p_y = CMg_y$\\
							\hline
								$\downarrow$ && $\downarrow$\\
							\hline
								\textbf{Equilibrio General Competitivo} & $\rightarrow$ & \textbf{Pareto-eficiente}\\
							\hline
						\end{tabular}
				\endgroup
			\end{center}
	\item La afirmación no es correcta. Según el segundo \emph{TFB}, siempre que las preferencias y la tecnología sean convexas y se cumplan todos los demás supuestos del primer \emph{TFB}, se puede obtener cualquier asignación Pareto-eficiente dada una adecuada redistribución de los recursos iniciales. A lo más, el gobierno debería encargarse de la correcta asignación inicial de recursos, pero el mercado podría llevarlo a una solución Pareto-eficiente. Sin embargo, hay que decir que este resultado es de difícil aplicación en el mundo real.
\end{enumerate}
	\item Considérese una determinada economía cuya dotación de los factores de producción capital $K$, y trabajo, $L$, es la dada por el vector $(K, L) = (20, 10)$. Estos inputs son utilizados en la producción de los bienes 1 y 2 de acuerdo con las tecnologías representadas por las funciones de producción $q_1 = min \{K_1, 2L_1\}$ y $q_2 = min \{2K_2, L_2\}$, respectivamente. En estas condiciones determinar:
		\begin{enumerate}[a)]
			\item Las asignaciones de factores o planes de producción $\left[\left( q_1, K_1, L_1\right) , \left(q_2, K_2, L_2 \right)  \right] $ eficientes en la producción.
			\item La frontera de posibilidades de producción $(FPP)$ de esta economía.
			\item El equilibrio walrasiano de esta economía con (solo) producción
		\end{enumerate}
		\textbf{\LARGE Solución}\\
			\begin{enumerate}[a)]
	\item El problema de principal para el agente $M$ es maximizar su beneficio neto: 
	\begin{align*}
		& \M \quad k e-w^{M}\\
		& \begin{array}{ll}
			\text{s.a. } & U^{M} \geq 0\\
			& w^{M}-2 e^{M^{2}} \geq 0
		\end{array}
	\end{align*}
	De la restricción de participación se obtiene: $w^{M}=2 e^{M^{2}}$.\\
	
	Por tanto el beneficio neto es: $k e-2e^{2}$. Derivando respecto al esfuerzo e igualando a cero: $k-4e=0$ 
	$$e^{M*}=\frac{k}{4}$$
	Reemplazando en la $R.P.$: $w^{M*} =\frac{k^{2}}{16}$\\
	
	Similarmente, el problema de principal para el agente B es:
	\begin{align*}
		& \M \quad k e-w^{B}\\
		& \begin{array}{ll}
			\text{s.a. } & U^{B} \geq 0\\
			& w^{B}-2e^{B^{2}} \geq 0
		\end{array}
	\end{align*}
	De la restricción de participación se obtiene: $w^{B}=e^{B^{2}}$.\\
	
	Por tanto el beneficio neto es: $ke-e^{2}$. Derivando respecto al esfuerzo e igualando a cero:$k-2e=0$
	$$e^{B*}=\frac{k}{2}$$
	Reemplazando en la $R.P.$: $w^{B*}=\frac{k^{2}}{4}$.
	\item El principal ofrece un menú de contratos. El problema de optimización es: 
	\begin{align*}
		& \M \quad q\left(ke^{B}-w^{B}\right)+(1-q)\left(ke^{M}-w^{M}\right)\\
		& \begin{array}{llc}
			\text{s.a. } & \quad w^{B}-e^{B^{2}} \geq 0 & (R.P.1)\\
			& \quad w^{M}-2 e^{M^{2}} \geq 0 & (R.P.2) \\
			& w^{B}-e^{B^{2}} \geq w^{M}-e^{M^{2}} & (R.I.1) \\
			& w^{M}-2 e^{M^{2}} \geq w^{B}-2 e^{B^{2}} & (R.I.2)
		\end{array}
	\end{align*}
	La restricción de participación del agente $B (R.P.1)$ y la restricción de incentivos del agente $M (R.I.2)$ están incluidas en las demás restricciones.\\
	
	El problema se reduce entonces a :
	\begin{align*}
		& q\left(ke^{B}-w^{B}\right)+(1-q)\left(ke^{M}-w^{M}\right)\\
		& \begin{array}{ll}
			\text{s.a. } & w^{B}-e^{B^{2}} \geq w^{M}-e^{M^{2}} \\
			& w^{M}-2 e^{M^{2}} \geq 0
		\end{array}
	\end{align*}
	La ecuación de Lagrange será:
	$$ L=q\left(k e^{B}-w^{B}\right)+(1-q)\left(k e^{M}-w^{M}\right)+\lambda\left(w^{B}-e^{B^{2}}-w^{M}+e^{M^{2}}\right)+\mu\left(w^{M}-2e^{M^{2}}\right) $$
	Las condiciones de primer orden serán:
	\begin{align} 
		&\frac{\partial L}{\partial w^{B}}=0 \Leftrightarrow -q+\lambda=0 \label{eq1}\\
		&\frac{\partial L}{\partial w^{M}}=0 \Leftrightarrow -1+q-\lambda+\mu=0 \label{eq2}\\
		&\frac{\partial L}{\partial e^{B}}=0 \Leftrightarrow qk-2 \lambda e^{B}=0 \label{eq3}\\
		&\frac{\partial L}{\partial e^{M}}=0 \Leftrightarrow (1-q) k+2 \lambda e^{M}-4 \mu e^{M} \label{eq4}
	\end{align}
	De (\ref{eq1})
	\begin{gather}
		\lambda=q>0 \label{eq5}
	\end{gather}
	luego
	\begin{align}
		&\text{(\ref{eq5}) en (\ref{eq2})}: -1+q-q+\mu=0 \Rightarrow \mu=1>0 \notag\\
		&\text{(\ref{eq5}) en (\ref{eq3})}: qk-2qe^{B}=0 \Rightarrow e^{B*}=\frac{k}{2} \label{eq6}\\
		&\text{(\ref{eq5}) y (\ref{eq6}) en (\ref{eq4})}: (1-q)k+2qe^{M}-4e^{M} \Rightarrow e^{M*}=\frac{(1-q)k}{4-2q} \notag
	\end{align}
	De la restricción de participación del agente $M$:
	$$w^{M*}=2e^{M^{2}}$$
	De la restricción de incentivos del agente $B$:
	$$w^{B*}=e^{B^{2}}+w^{M}-e^{M^{2}}=e^{B^{2}}+2 e^{M^{2}}-e^{M^{2}}=e^{B^{2}}+e^{M^{2}}$$
	El esfuerzo que se pide al agente B es el mismo que con información simétrica $(\mathrm{k} / 2)$.\\
	
	El agente $M$ realiza un esfuerzo menor $\left(\frac{(1-q)k}{4-2q}<\frac{k}{4}\right)$\\
	
	El agente $M$ obtiene su utilidad de reserva (la restricción de participación se cumple con igualdad).\\
	
	El agente $B$ obtiene una renta informacional $\left(e^{B^{2}}+e^{M^{2}}>e^{B^{2}}\right)$
\end{enumerate}
	\item Consideremos una economía compuesta por un único input - el trabajo - y cuya dotación es $L$, dos bienes finales - los bienes 1 y 2 - producidos por sendas empresas mediante las tecnologías descritas por la funciones de producción $q_1 = L_{1}^{\alpha}$ y $q_2 = L_{2}^{\beta}$, siendo $\alpha, \beta  > 0$.
		\begin{enumerate}[a)]
			\item Determinar la frontera de posibilidades de producción $(FPP)$ o curva de transformación de esta economúa y las propiedades de dicha $FPP$
			\item Analizar la curvatura de la $FPP$ según los valores uqe pueden adoptar los parámetros $\alpha$ y $\beta$.
		\end{enumerate}
		\textbf{\LARGE Solución}\\
			\vspace{-0.3cm}
\begin{center}
	\begingroup
		\setlength{\tabcolsep}{10pt}
		\renewcommand{\arraystretch}{1.5} 
			\begin{tabular}{ccc}
					\hline
				\multicolumn{3}{c}{Datos} \\
					\hline
				2 consumidores: & {} & $A$ y $B$ \\
				2 bienes: & {} &$x$ e $y$ \\
				F.U: & {} &$u^{A} = \ln(x_A) + 2y_A$ y $u^{B} = \ln(x_B) + 2y_B$ \\
				Dotaciones: & {} &$w^A = (1, 4)$ y $w^B = (6, 3)$ \\
					\hline
			\end{tabular}
	\endgroup
\end{center}
La curva de contrato:
	$$RMS^A = RMS^B \Longrightarrow x_A = x_B$$
además
	$$x_A + x_B = \overline{x} \Longrightarrow x_A + x_A = \overline{x} \Longrightarrow x_A = \frac{\overline{x}}{2} = x_B$$
	\begin{center}
		\begin{tikzpicture}[scale=0.8]
				% Formato de CAJA
					\draw[->] (0,0) node[align=center, below left] {\footnotesize $O_A$} -- (0,8) node[align=center, above] {\footnotesize $x_{2}^{A}$};
				\draw[->] (0,0) -- (8,0) node[align=center, right] {\footnotesize $x_{1}^{A}$};
				
					\draw[->] (7,7) node[align=center, above right] {\footnotesize $O_B$} -- (-1,7) node[align=center, left] {\footnotesize $x_{2}^{B}$};
				\draw[->] (7,7) -- (7,-1) node[align=center, below] {\footnotesize $x_{2}^{B}$};
						
			% Curvas de indiferencia
				\draw [blue] (1.4,6.9) .. controls (2.84,5.7) and (3.8,5.2) .. (5.6,4.8);
				\draw [red] (1.4,6.15) .. controls (3.03,5.8) and (3.99,5.38) .. (5.6,4.05);
				
				\draw [blue] (1.4,5) .. controls (2.84,3.8) and (3.8,3.3) .. (5.6,2.9);
				\draw [red] (1.4,4.25) .. controls (3.03,3.9) and (3.99,3.48) .. (5.6,2.15);
				
				\draw [blue] (1.4,3.1) .. controls (2.84,1.9) and (3.8,1.4) .. (5.6,1);
				\draw [red] (1.4,2.35) .. controls (3.03,2) and (3.99,1.58) .. (5.6,0.25);
			% Curvas de contrato
				\draw [purple, very thick] (3.5,0) node [below, scale= 0.6mm] {$\frac{\overline{x}}{2}$} -- (3.5,7);
				
			% Flechas
				\node[draw, single arrow,
					minimum height=28mm, minimum width=1mm,
					single arrow head extend=1.5mm,
					anchor=west, blue, fill=blue, scale=0.5, rotate=33] at (1.5,5.1) {};
				\node[draw, single arrow,
					minimum height=28mm, minimum width=1mm,
					single arrow head extend=1.5mm,
					anchor=west, red, fill=red, rotate=-147, scale=0.5] at (5.5,2.1) {};
		\end{tikzpicture}
	\end{center}

Probamos si las dotaciones iniciales son ESP.
	$$x_{A} = x_{B} = \frac{\overline{x}}{2} \Longrightarrow	1 \neq 6 \neq \frac{7}{2}$$
Se comprueba que la dotación inicial
	\begin{center}
		\begin{tikzpicture}[scale=0.8]
			% Formato de CAJA
				\draw[->] (0,0) node[align=center, below left] {\footnotesize $O_A$} -- (0,8) node[align=center, above] {\footnotesize $x_{2}^{A}$};
				\draw[->] (0,0) -- (8,0) node[align=center, right] {\footnotesize $x_{1}^{A}$};
			
				\draw[->] (7,7) node[align=center, above right] {\footnotesize $O_B$} -- (-1,7) node[align=center, left] {\footnotesize $x_{2}^{B}$};
				\draw[->] (7,7) -- (7,-1) node[align=center, below] {\footnotesize $x_{2}^{B}$};
			
			% Curvas de indiferencia
				\draw [blue] (1.4,6.9) .. controls (2.84,5.7) and (3.8,5.2) .. (5.6,4.8);
				\draw [red] (1.4,6.15) .. controls (3.03,5.8) and (3.99,5.38) .. (5.6,4.05);
				
				\draw [blue] (1.4,5) .. controls (2.84,3.8) and (3.8,3.3) .. (5.6,2.9);
				\draw [red] (1.4,4.25) .. controls (3.03,3.9) and (3.99,3.48) .. (5.6,2.15);
				
				\draw [blue] (1.4,3.1) .. controls (2.84,1.9) and (3.8,1.4) .. (5.6,1);
				\draw [red] (1.4,2.35) .. controls (3.03,2) and (3.99,1.58) .. (5.6,0.25);
				
			% Curvas de contrato
				\draw [purple, very thick] (3.5,0) node [below, scale= 0.6mm] {$\frac{\overline{x}}{2}$} -- (3.5,7);
			
			% Flechas
				\node[draw, single arrow,
						minimum height=28mm, minimum width=1mm,
						single arrow head extend=1.5mm,
						anchor=west, blue, fill=blue, scale=0.5, rotate=33] at (1.5,5.1) {};
				\node[draw, single arrow,
						minimum height=28mm, minimum width=1mm,
						single arrow head extend=1.5mm,
						anchor=west, red, fill=red, rotate=-147, scale=0.5] at (5.5,2.1) {};
			
			% Intersección
				\draw[dashed] (1,7) node[above] {\footnotesize $x_{1}^{B}=6$} -- (1,0) node[below] {\footnotesize $x_{1}^{A}=1$};
				\draw[dashed] (0,4) node[left] {\footnotesize $x_{2}^{A}=4$} -- (7,4)node[right] {\footnotesize $x_{2}^{B}=3$};
			
			% Punto
				\draw[black, fill=black] (1,4) circle[radius=0.08] node[align=center, above left] {\footnotesize $w$};
		\end{tikzpicture}
	\end{center}

Con la inclusión de precios, la situación es la siguiente:
	$$RMS^A = RMS^B = \frac{p_x}{p_y}$$
	$$RMS^A=\frac{1}{2x_A}=\frac{p_x}{p_y} \quad , \quad RMS^B=\frac{1}{2x_B}=\frac{p_x}{p_y} \Longrightarrow x_A = x_B = \frac{p_y}{2p_x}$$
	
	$$
		\begin{array}{ccc}
			p_xx_A + p_yy_A = p_x + 4p_y & {} &p_xx_B + p_yy_B = 6p_x + 3p_y\\[0.4cm]
			p_x\frac{p_y}{2p_x} + p_yy_A = p_x + 4p_y & {} &p_x\frac{p_y}{2p_x} + p_yy_B = 6p_x + 3p_y\\[0.4cm]
			\frac{p_y}{2} + p_yy_A = p_x + 4p_y & {} & \frac{p_y}{2} + p_yy_B = 6p_x + 3p_y \\[0.4cm]
			y_{A}^* = \frac{2p_x + 7p_y}{2p_y} & {} & y_{B}^* = \frac{12p_x + 5p_y}{2p_y}
		\end{array}
	$$
	
	$$y_{A}^* + y_{B}^* = 4 + 3 = 7 \Longrightarrow \frac{2p_x + 7p_y}{2p_y} + \frac{12p_x + 5p_y}{2p_y} = 7  \Longrightarrow  \frac{p_x}{p_y} = \frac{1}{7}$$
	
	\begin{gather*}
		\therefore y_{A}^* = \frac{51}{14}  \Longrightarrow x_{B}^* = \frac{7}{2}\\[0.4cm]
		\therefore y_{b}^* = \frac{47}{14}  \Longrightarrow x_{B}^* = \frac{7}{2}
	\end{gather*}

	\begin{center}
		\begin{tikzpicture}[scale=0.8]
			% Formato de CAJA
				\draw[->] (0,0) node[align=center, below left] {\footnotesize $O_A$} -- (0,8) node[align=center, above] {\footnotesize $x_{2}^{A}$};
				\draw[->] (0,0) -- (8,0) node[align=center, right] {\footnotesize $x_{1}^{A}$};
				
				\draw[->] (7,7) node[align=center, above right] {\footnotesize $O_B$} -- (-1,7) node[align=center, left] {\footnotesize $x_{2}^{B}$};
				\draw[->] (7,7) -- (7,-1) node[align=center, below] {\footnotesize $x_{2}^{B}$};
			
			% Curvas de indiferencia
				\draw [blue] (1.4,6.9) .. controls (2.84,5.7) and (3.8,5.2) .. (5.6,4.8);
				\draw [red] (1.4,6.15) .. controls (3.03,5.8) and (3.99,5.38) .. (5.6,4.05);
				
				\draw [blue] (1.4,5) .. controls (2.84,3.8) and (3.8,3.3) .. (5.6,2.9);
				\draw [red] (1.4,4.25) .. controls (3.03,3.9) and (3.99,3.48) .. (5.6,2.15);
				
				\draw [blue] (1.4,3.1) .. controls (2.84,1.9) and (3.8,1.4) .. (5.6,1);
				\draw [red] (1.4,2.35) .. controls (3.03,2) and (3.99,1.58) .. (5.6,0.25);
			
			% Curvas de contrato
				\draw [purple, very thick] (3.5,0) -- (3.5,7);
			
			% Intersección
				\draw[dashed] (1,7) node[above] {\footnotesize $x_{1}^{B}=6$} -- (1,0) node[below] {\footnotesize $x_{1}^{A}=1$};
				\draw[dashed] (0,4) node[left] {\footnotesize $x_{2}^{A}=4$} -- (7,4)node[right] {\footnotesize $x_{2}^{B}=3$};
			
			% Punto
				\draw[black, fill=black] (1,4) circle[radius=0.08] node[align=center, above left] {\footnotesize $w$};
				\draw[black, fill=black] (3.5,3.58) circle[radius=0.08] node[align=center, below left] {\footnotesize $EGW$};
			
			% Recta presupuestaria
				\draw (0,5.26) -- (7,1.9) node[right] {\footnotesize $RP$};
			
			% Intersección
				\draw[dashed] (3.5,7) node[above] {\footnotesize $x_{1}^{B}=3.5$} -- (3.5,0) node[below] {\footnotesize $x_{1}^{A}=3.5$};
				\draw[dashed] (0,3.58) node[left] {\footnotesize $x_{2}^{A}=3.64$} -- (7,3.58)node[right] {\footnotesize $x_{2}^{B}=3.36$};
		\end{tikzpicture}
	\end{center}
	\item Considérese una economía en la que, existen dos factores productivos - los inputs $K$ y $L$- cuyas dotaciones iniciales son las dadas por el vector $(K,L) = (30,10)$, y dos bienes finales - los bienes 1 y 2- que se obtienen a partir de los inputs mencionados de acuerdo con las tecnología descritas por la sfunciones de producción de inputs sustitutivos $q_1 = 2K_1 + L_!$ y $q_2 = K_2 + 2L_2$, respectivamente. En estas condiciones, determinar:
		\begin{enumerate}[a)]
			\item Los usos eficientes de factores en la producción.
			\item El conjunto de posibilidades de producción $(CPP)$ de la economía.
			\item El equilibrio Walrasiano.
		\end{enumerate}
		\textbf{\LARGE Solución}\\
			\begin{enumerate}[a)]
	\item La pendiente de la \emph{FBS} $u^2(u^1) = \sqrt{100-u^2}$ es $\frac{\partial u^2}{\partial u^1} = -\frac{1}{2\sqrt{100-u^2}}$ y, evaluada en el punto $(u^1,u^2) = (75,5)$, es (el valor absoluto) igual a $\frac{1}{10}$. Por otra parte, dada la \emph{FBS} de Bergson que ``tiene in mente'' el individuo 1 y que es del tipo $W(u^1,u^2) = \theta^1u^1 + \theta^2u^2$, su pendiente es $-\frac{\theta^1}{\theta^2}$. Entonces ha de verificarse.
		$$\frac{\theta^1}{\theta^2}=\frac{1}{10}$$
	Por otra parte, el punto de máximo bienestar debe ser tal que el problema
		\begin{equation}
			\begin{aligned}
				& \M \limits_{(u^1,u^2)} \quad \theta^1u^1 + \theta^2u^2\\
				& \begin{array}{ll}
					\text{s.a. } & u^1 +(u^2)^2 = 100
				\end{array}
			\end{aligned} \label{eq1}
		\end{equation}
	dé como resultado el reparto de bienestar $(u^1,u^2) = (75,5)$. Las CPO de este problema (\ref{eq1}) son 
		\begin{gather}
			\theta^1 - \lambda = 0 \label{eq2}\\
			\theta^2 - 2 \lambda u_2 = 0 \label{eq3}\\
			\intertext{y}
			100 - u_1 - u_{2}^2 = 0
		\end{gather}
	De las CPO (\ref{eq2}) y (\ref{eq3}) resulta $\theta^1=\frac{\theta^2}{2u^2}$, y dado que $u^2=5$, se obtiene $\theta^2=10\theta^2$. En definitivo,
		$$W(u^1, u^2) = \theta^1u^1+10\theta^1u^2, \theta^1>0$$
	es la \emph{FBS} utilitarista ponderada a o de Bergson que implícitamente está utilizando el individuo 1.
	\item Operando de manera análoga a la del apartado 1, se obtiene
		$$W(u^1, u^2) = \theta^1u^1+18\theta^1u^2, \theta^1>0$$
	como \emph{FBS} según el criterio del individuo 2.
\end{enumerate}
	\item Considérese una economía $2 \times 4 \times 2 $ en la que existen dos inputs productivos básicos - los inputs $K$ y $L$- en cantidades $(K,L) = (10,10)$. En esta economía se pueden obtener dos vienes finales - los bienes 1 y 2- mediante las tecnología descritas por las funciones de producción $q_1 = K_1 + 2L2$ y $q_2 = K_2 + 2L_2$, respectivamente. Finalmente, en la economía también existen dos consumidores 1 y 2 - cuyas preferencias sobre los bienes finales esta´n dadas por las funciones de utilidad $u^1\left( q_{1}^{1},q_{2}^{1}\right) = q_{1}^{1} + q_{2}^{1}$ y $u^2\left( q_{1}^{2},q_{2}^{2}\right) = q_{2}^{2}$, respectivamente. En estas condiciones,
		\begin{enumerate}[a)]
			\item Calcular el conjunto de posibilidades de producción $(CPP)$ y el conjuntos de posibilidades de producción $(CPU)$ de esta economía.
			\item Comprobar que la asignación de consumo $\left[\left( q_{1}^{1},q_{2}^{1}\right), \left( q_{1}^{2},q_{2}^{2}\right)\right] = \left[\left( 10,5 \right), \left( 0,15\right)\right]$ es eficiente. Determinar, asimismo, los precios y la distribución incial de recursos que hacen que dichos consumos, así como las decisiones de producción que los hacen posibles, puedan descentralizarse como equilibrios walrasianos.
		\end{enumerate}
		\textbf{\LARGE Solución}\\
			\begin{enumerate}[a)]
	\item La fucnión de producción del bien 2, $q_2 = K_2 + L_2$, se puede expresar como
					\begin{gather}
						q_2 = (10 - K_1) + 2(10- L_1) = 30-(K_1 + 2L_2) \label{eq7}
					\end{gather}
			y dado que $q_1 = K_1 + 2L_1$, basta con reescribir (\ref{eq7}) para llegar al $CPP$ de la economía
					\begin{gather}
						\left\lbrace (q_1,q_2) \in \mathbb{R}_{+}^{2} \mid q_2(q_1) = 30 - q_1\right\rbrace \label{eq8}
					\end{gather}
			De mamenra análogo, el $CPU$ se obtiene teniendo en cuenta las funciones de utlidad de los consumidores, $u^1 = q_{1}^{1} + q_{2}^{1}$ y $u^2 = q_{2}^{2}$, y además el hecho de que la cantidad de bien 1 satisface la condición
					$$q_{1}^{1} + 0 = q_1$$
			mientras que la cantidad de bien 2 verifica
					$$q_{2}^{1} + q_{2}^{2} = q_{2}$$
			condición que teniendo en cuenta el $CPP$ dado en (\ref{eq8}) se convierte en
					$$q_{2}^{1} + q_{2}^{2} = 30 - q_{1}$$
			Sumando las utilidades, se tiene que $u^1 + u^2 = q_1 + q_2 = 30$. En definitiva, el $CPP$ de la economía descrita es
					$$\left\lbrace (u^1,u^2) \in \mathbb{R}_{+}^{2} \mid u^2(u^1) = 30 - u^1\right\rbrace$$
	\item Dada las asignaciones $\left[ \left( q_{1}^{1},q_{2}^{1} \right) ,\left( q_{1}^{2},q_{2}^{2} \right) \right] = \left[ \left( 10, 5\right) ,\left( 0, 15\right) \right]$, asignación representada por el punto $A$ de la caga de Edgeworth para el consumo de la siguiente figura:
			\begin{center}
				\begin{tikzpicture}[scale=1.065]
					% FPP
						\draw[orange] (0,3) node [black, left, scale = 0.6] {30} -- (3,0) node [black, below, scale = 0.6] {30};
					% Curva de indiferencia
						% Agente A
							\draw[blue] (0,2) node [black, above left, scale = 0.5] {$20$} -- (1,1);
							\draw[blue] (0,1.5) -- (1,0.5);
							\draw[blue] (0,1) -- (1,0) node [black, below right, scale = 0.5] {$10$};
						% Agente B
							\draw[red] (0,0.5) -- (1,0.5);
							\draw[red] (0,1) -- (1,1);
							\draw[red] (0,1.5) -- (1,1.5);
					% Caja en consumo
						\draw[<->] (0,2.5) node [black, left, scale = 0.6] {$q_{2}^{1}$}-- (0,0)  node [black, below left , scale = 0.6] {$O_1$}-- (1.5,0) node [black, below, scale = 0.6] {$q_{1}^{1}$};
						\draw[<->] (-0.5,2) node [black, left, scale = 0.6] {$q_{1}^{2}$} -- (1,2) node [black, above right, scale = 0.6] {$O_2$} -- (1,-0.5) node [black, below, scale = 0.6] {$q_{2}^{2}$};
					% Eje
						\draw[<->] (0,4) node[align=center, above] {$q_2$} -- (0,0)  -- (4,0) node[align=center, right] {$q_1$};
					% Punto
						\draw[black, fill=black] (1,0.5) circle[radius=0.05] node [right, scale=0.7] {$A$};
				\end{tikzpicture}
			\end{center}
		es claro que una posibilidad para aumentar $u^2$ es aumentar $q_{2}^{2}$, pero entonces disminuye $y^1$. Del mismo modo, para aumentar $u^1$ habría que incrementar $q_{1}^{1}$ o $q_{2}^{1}$, pero entonces disminuye $u^1$. Luego la asignación $\left[ \left( q_{1}^{1},q_{2}^{1} \right) ,\left( q_{1}^{2},q_{2}^{2} \right) \right] = \left[ \left( 10, 5\right) ,\left( 0, 15\right) \right]$ es eficiente en el consumo. y como el plan de producción $\left[ \left( q_1, K_1, L_1 \right) ,\left( q_2, K_2, L_2 \right) \right] = \left[ \left( 10, 10, 0\right), \left( 20, 0, 10\right) \right]$ que permite dichos consumos es técnicamente eficiente, se concluye que la asignación es eficiente en términos globales.\\
		
		Para descentralizar esta asignación de recursos, es necesario tener en cuenta, de acuerdo con la siguiente figura:
			\begin{center}
				\begin{tikzpicture}[scale=1.5]
					% FPP
						% Curva
							\draw[orange] (0,3) node [black, left, scale = 0.6] {30} -- (3,0) node [black, below, scale = 0.6] {30};
						% Eje
							\draw[<->] (0,4) node[align=center, above] {$q_2$} -- (0,0)  -- (4,0) node[align=center, right] {$q_1$};
					% Caja en consumo
						% Curva de indiferencia
							% Agente A
							\draw[blue] (0,1.5) -- (1,0.5);
							% Agente B
							\draw[red] (0,0.5) -- (1,0.5);
						% Ejes
							\draw[<->] (0,2.5) node [black, left, scale = 0.6] {$q_{2}^{1}$}-- (0,0)  node [black, below left , scale = 0.6] {$O_1$}-- (1.5,0) node [black, below, scale = 0.6] {$q_{1}^{1}$};
							\draw[<->] (-0.5,2) node [black, left, scale = 0.6] {$q_{1}^{2}$} -- (1,2) node [black, above right, scale = 0.6] {$O_2$} -- (1,-0.5) node [black, below, scale = 0.6] {$q_{2}^{2}$};
						% Punto
							\draw[black, fill=black] (1,0.5) circle[radius=0.035] node [right, scale=0.7] {$A$};
							\draw[black, fill=black] (0,1.5) circle[radius=0.035] node [left, scale=0.7] {$B$};
							\draw (1,0) node [black, below right, scale = 0.5] {$10$};
							\draw (0,2) node [black, above left, scale = 0.5] {$20$};
						% Recta de precios
						\draw (0.1,0.8) -- (1,0.5) -- (0.4,1.6);
					% Flechas
						\draw[->] (1.6,2.8) node [right, scale = 0.5] {Pendiente $= -1$}-- (0.05,1.45);
						\draw[->] (2,2.5) node  [right, scale = 0.5] {$\frac{p_1}{p_2} > 1$} -- (0.58,1.26);
						\draw[->] (2.4,2.2) node [right, scale = 0.5] {$\frac{p_1}{p_2} < 1$} --(0.6,0.63);
				\end{tikzpicture}
			\end{center}
		que si los precios de los bienes, $p_1$ y $p_2$, son tales que $\frac{p_1}{p_2} > 1$, no es posible descentralizar la citada asignación porque el individuo 1 consumiría únicamente el bien 2. Es por ello que los precios candidatos a descentralizar la asignación propuesta han de verificar la condición
			$$\frac{p_1}{p_2} \leq 1$$
		A la hora de calcular la dotación incial de recursos de los consumidores , $\left[W^1, W^2 \right] = \left[\left(w_{1}^{1},w_{2}^{1} \right), \left(w_{1}^{2},w_{2}^{2} \right) \right] $, en la que debes estas ``situada'' la economía para alcanzar la asignaciones propuestas, es necesario tener en cuenta que la utilidad del consumidor 1 debe cumplir la condición $u^{1}(W^1) = u^1(10,5)$. Con ello, las condiciones a satisfacer por $\left[W^1, W^2 \right]$ son:
			\begin{gather}
				10p_1 + 5p_2 = p_1w_{1}^{1} + p_2w_{2}^{1} \label{eq13}\\
				15p_2 = p_1w_{1}^{2}+p_2w_{2}^{2} \label{eq14} \\
				w_{1}^{1} + w_{1}^{2} = 10 \label{eq15} \\
				w_{2}^{1} + w_{2}^{2} = 20 \label{eq16} \\
				\frac{p_1}{p_2} < 1 \label{eq17}
			\end{gather}
		y, además, $\left[W^1, W^2 \right]$ ha de estar situada dentro del triángulo $ABC$ (ya que si está por encima, entonces $u^2(W^2) < u^2(0,5)$ y si está por debajo entonces $u^1(W^1) < u^1(10,5))$.\\
		
		Si fijamos $p_1 = 1$ y $p_2 = 2$, la condición (\ref{eq13}) se convierte en
			\begin{gather}
				10 + 10 = w_{1}^{1} + 2w_{2}^{1} \label{eq18}
			\end{gather}
		y la condición (\ref{eq14}) en
			\begin{gather}
				30 = w_{1}^{2} + 2w_{2}^{2} = (10 - w_{1}^{1}) + 2(20 - w_{2}^{1}) = 50 -w_{1}^{1} - 2w_{2}^{1} \label{eq19}
			\end{gather}
		una vez tenidas en cuenta (\ref{eq15}) y (\ref{eq16}). Finalmente, es vlaro que las condiciones (\ref{eq18}) y (\ref{eq19}) son idénticas, ya que ambas colapsan en $20 = w_{1}^{1} + 2w_{2}^{1}$, uqe es la condición de recursos inciales. En definitiva, con el sistema de precios
			$$(p_1,p_2) = (1,2)$$
		y el conjunto de rentas
			$$(m^1, m^2) = (20,30)$$
		se consigue descentralizar la asignación propuesta.
\end{enumerate}
\end{enumerate}
%------------------------------------------------------------------------------------
\end{document}		
%====================================================================================