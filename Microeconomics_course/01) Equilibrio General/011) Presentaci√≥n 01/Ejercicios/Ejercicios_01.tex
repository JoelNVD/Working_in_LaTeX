%====================================================================================
% Preamble
%------------------------------------------------------------------------------------
\documentclass[10pt,a4paper]{article}

% Apartado de texto
\usepackage[utf8]{inputenc}
\usepackage[spanish]{babel}
\usepackage[T1]{fontenc}

% Apartado matemático
\usepackage{amsmath}
\usepackage{amsfonts}
\usepackage{amssymb}
\usepackage{mathtools}
\usepackage[libertine]{newtxmath}

% Apartadode no sangría
\usepackage{parskip}

% Aparatod posición del texto
\usepackage[left=2cm,right=2cm,top=2cm,bottom=2cm]{geometry}
\usepackage{fancyhdr}

\usepackage{graphicx}
\usepackage{xcolor}
\definecolor{cardinal}{rgb}{0.77, 0.12, 0.23}

% Apartado de columnas y filas: juntarlas o separarlas
\usepackage{multicol, multirow}

% Apartado differents label en listado
\usepackage[shortlabels]{enumitem}

% Apartado hyperres
\usepackage[hidelinks]{hyperref}

\usepackage{tcolorbox}
\newtcolorbox{nota}[1]{
	colback=cardinal!5!white,
	colframe=cardinal!75!black,
	fonttitle=\bfseries,
	title=#1
}

% Apartado dibujos
\usepackage{tikz, pgfplots}
\usetikzlibrary{positioning,calc,arrows}
\usetikzlibrary{shapes.arrows}
\usetikzlibrary{patterns}
\usetikzlibrary{babel} % Para que recono > y <, en inglés no se pone, pero en spañol sí
\usetikzlibrary{shadings,shadows}
\usetikzlibrary{matrix}

% Apartado cambio como por punto
\decimalpoint

% Definir oprerador
\DeclareMathOperator*{\M}{Max}

%====================================================================================

%====================================================================================
% Body
%====================================================================================
% Title Page
%-----------
\textwidth=450pt \textheight=620pt \oddsidemargin=0in
\topmargin=-10pt
\pagestyle{fancy} \rhead{\scriptsize{\textbf{Código del Curso:} xxXXXXX} \\
	\textbf{Fecha:} XX/XX/2021 \& 2021-I \hspace{0.04cm}}
\lhead{\scriptsize{\textbf{Profesor:} José A. Valderrama}  \\
	\textbf{Curso:} Teoría Microeconómica II}
\newcommand{\re}[1]{\smallskip\textsf{\textbf{Respuesta}} \begin{sf}\\ #1 \end{sf} \bigskip}
\newcommand{\ay}[1]{ \scriptsize{\textsl{Hint: #1}}\normalsize{}}
\newcommand{\pr}[2]{\frac{\partial #1}{\partial #2}}

%------------------------------------------------------------------------------------
% Title
%---------
\begin{document}
	\begin{center}
		{\Large {\textbf{Práctica Dirigida N$^{\circ}$1}}}

		\textsc{Equilibrio General}
		
	\end{center}
	
% EJERCICIOS---------------------------------------------------------------------------------
\begin{enumerate}
	\item Una economía esta formado por 2 unidades $A$ y $B$, cuyas dotaciones iniciales son
			$$\left(w_{1}^{A}+w_{2}^{A}\right) = \left( 6,8\right) \qquad \left(w_{1}^{B}+w_{2}^{B}\right) = \left(12,4\right)$$
	 	  ¿Cómo podemos representar los distintos posible de consumo final para $A$ y $B$?\\[0.5cm]
			\textbf{\LARGE Solución}\\
				\vspace{-0.7cm}
\begin{multicols}{2}
	\begin{tikzpicture}
		\begin{axis}[scale=0.9,
					 xmin=0, xmax=20,
					 ymin=0, ymax=12,
					 title style={at={(0.5,0.2)},anchor=north,yshift=5cm},
					 title = Consumidor $A$,
					 axis lines = left,
					 xtick={2,4,6,8,10,12,14,16,18},
					 ytick={2,4,6,8,10},
					 grid=both,
					 grid style={line width=.1pt, draw=gray!20},
					 clip = false,
					]
			
			\node [above]  at (current axis.above origin) {$x_{2}^{A}$};
			\node [right]  at (current axis.right of origin) {$x_{1}^{A}$};
			
			\draw[fill=black] (60,80) circle (1.5) node[above right] {$w^{A}$};
		\end{axis}
	\end{tikzpicture}

	\begin{tikzpicture}
		\begin{axis}[scale=0.9,
					 xmin=0, xmax=20,
					 ymin=0, ymax=12,
					 title style={at={(0.5,0.2)},anchor=north,yshift=5cm},
					 title = Consumidor $B$,
					 axis lines = left,
					 xtick={2,4,6,8,10,12,14,16,18},
					 ytick={2,4,6,8,10},
					 grid=both,
					 grid style={line width=.1pt, draw=gray!20},
					 clip = false,
					]
			
			\node [above]  at (current axis.above origin) {$x_{2}^{B}$};
			\node [right]  at (current axis.right of origin) {$x_{1}^{B}$};
			
			\draw[fill=black] (120,40) circle (1.5) node[above right] {$w^{B}$};
		\end{axis}
	\end{tikzpicture}
\end{multicols}

Otras posibilidades pasan por redistribuir unidades entre ambos consumidores. Por ejemplo, si $A$ quiere consumir más de 6 unidades de $x_1$ habrá que quitarle a $B$.\\

Podrían proponerse otras posibles repartos, si representamos una asiganción como $\left(x_{1}^{A}, x_{2}^{A}, x_{1}^{B}, x_{2}^{B}\right)$ y posibles asignaciones como $\left( 4,8,14,4\right)$, $\left(6,6,12,6 \right)$ o $\left( 10,5,8,7\right)$; sin embargo la asignación $\left( 9,5,11,6\right)$ no es una asignación válida por lo siguiente:

$$\begin{array}{ccc}
	x_{1}^{A} + x_{2}^{A} = w_{1}^{A} + w_{2}^{A} & {} & x_{1}^{B} + x_{2}^{B} = w_{1}^{B} + w_{2}^{B} \\
	 4 + 14    = 6 + 12  & {} & 8 + 4    = 8 + 4 \\
	 6 + 12    = 6 + 12  & {} & 6 + 6    = 8 + 4 \\
	 9 + 11 \neq 6 + 12  & {} & 5 + 6 \neq 8 + 4 
\end{array}$$

	\item Los consumidores tienen las siguientes dotaciones iniciales 
		  $\left(w_{1}^{A}+w_{2}^{A}\right) = \left( 12,3\right)$ y 
		  $\left(w_{1}^{B}+w_{2}^{B}\right) = \left( 3,4\right)$ y sus funciones de utilidad son:
			\begin{gather*}
				U^{A}\left(x_{1}^{A},x_{2}^{A}\right)  = x_{1}^{A}x_{2}^{A}\\
				U^{B}\left(x_{1}^{B},x_{2}^{B}\right)  = x_{1}^{B}x_{2}^{B}\\
			\end{gather*}
		 ¿Pertenecen los puntos $\left(9,5 \right)$ y $\left(8,9 \right)$ al área de intercambio voluntario?\\[0.5cm]
			\textbf{\LARGE Solución}\\
				La pregunta anterior se resolvió de la forma general, pero se puede resolver de forma práctica igualando \emph{TMS}; si y solo si, las funciones de utilidad son tipo \emph{Cobb-Douglas} o transformaciones monotónicas de esta. Veamos.

	\begin{enumerate}[a)]
		\item \colorbox{yellow}{$TMS_A = TMS_B$}
				Teniendo en cuenta:
					$$TMS = \frac{UMg_x}{UMg_y} = \frac{\partial U/\partial x}{\partial U/\partial y}$$
				entonces,
					$$\frac{y_A}{x_A} = \frac{y_B}{x_B} \rightarrow \frac{y_A}{x_A} = \frac{3-y_A}{2-x_A} \rightarrow y_A = \frac{3x_A}{2} \text{ y } y_B = \frac{3x_B}{2}$$
		\item $$
				\begin{array}{c|c}
					U_A = x_A\left( \frac{3x_A}{2}\right)  & U_B = x_B\left( \frac{3x_B}{2}\right) \\[0.3cm]
					x_A = \frac{\sqrt{6}}{3}U_A^{1/2} & x_B = \frac{\sqrt{6}}{3}U_B^{1/2}
				\end{array}
			  $$
			  	$x_A + x_B = 2 \longrightarrow U_A^{1/2}+U_B^{1/2} = \sqrt{6} \longrightarrow \therefore U_B = \left( \sqrt{6} - U_A^{1/2}\right)^{2} $
			  	\begin{center}
			  		\begin{tikzpicture}[scale = 0.7, samples = 100]
			  			\draw[purple, domain = 0:6] plot({\x},{(sqrt(6)-(\x)^(0.5))^(2)});
			  			\draw[<->] (0,7) node [above] {$U_B$} -- (0,0) -- (7,0) node [right] {$U_A$};
			  			\draw[black, fill=black] (0,6) circle[radius=0.07] node [left] {6};
			  			\draw[black, fill=black] (6,0) circle[radius=0.07] node [below] {6};
			  		\end{tikzpicture}
			  	\end{center}
			  	
	\end{enumerate}
	\item ¿Cómo calcular los puntos $p_1$ y $p_2$ para el caso en el que venimos  	   trabajando?\\[0.5cm]
			\textbf{\LARGE Solución}\\
				\begin{enumerate}[a)]
	\item Se requiere que las asignaciones factoriales sean factibles y no derrochadoras
			\begin{align*}
				RMT^A & = RMS^B\\
				\frac{PMgL^{A_x}}{PMgK^{A_x}} =\frac{K_{A_x}}{L_{A_x}} & = \frac{K_{B_y}}{2L_{B_y}} =\frac{PMgL^{B_y}}{PMgK^{B_y}}
			\end{align*}
		  Reemplazando $L_{A_x} + L_{B_y} = \overline{L}$ y $K_{A_x} + K_{B_y} = \overline{K}$, para hallar la curva de contrato
			$$\therefore \frac{K_{A_x}}{L_{A_x}} = \frac{\overline{K} - K_{A_x}}{2\overline{L} - L_{A_x}}$$
	\item Una asignación igualitaria de los factores entre las dos empresas significa que:
			\begin{gather*}
				L_{A_x} = \frac{\overline{L}}{2};\enskip K_{A_x}=\frac{\overline{K}}{2}\\
				L_{B_y} = \frac{\overline{L}}{2};\enskip K_{B_y}=\frac{\overline{K}}{2}
			\end{gather*}
		  Resulta evidente que con esta asignación de factores:
		  	$$\frac{K_{A_x}}{L_{A_x}} = \frac{K_{B_y}}{L_{B_y}}$$
		  por lo que no se verifica la condición $RMT^A = RMS^B$ de eficiencia
\end{enumerate}
	\item ¿Cuál es la curva de contrato en el ejercicio anterior?\\[0.5cm]
			\textbf{\LARGE Solución}\\
				La curva de contrato está dada por la siguiente relación:
	\begin{eqnarray*}
		RMS_{A} & = & RMS_{B}\\[0.3cm]
		\frac{x_{2}^{A}}{x_{1}^{A}} & = & \frac{x_{2}^{B}}{x_{1}^{B}}\\[0.3cm]
		\frac{x_{2}^{A}}{x_{1}^{A}} & = & \frac{12 - x_{2}^{A}}{15 - x_{1}^{A}}\\[0.3cm]
		\therefore x_{2}^{A} & = & \frac{12}{15}x_{1}^{A}
	\end{eqnarray*}
La curva de contrato será una \textbf{LÍNEA RECTA}
	\item Las preferencias del consumidor $A$ vienen dados por la función de utilidad. \label{Ejercicio_5}
	 	  		\vspace{-0.8cm}
			\begin{multicols}{2}
				\begin{eqnarray*}
					U_{A} \left(x_{1}^{A},\enskip x_{2}^{A} \right) & = & x_{1}^{A}\cdot \left(x_{2}^{A} \right)^{2}\\
					\left( w_{1}^{A},\enskip w_{2}^{A} \right) & = & \left(16,\enskip 4 \right) \\
					\left(p_1,\enskip p_2 \right) & = &\left(1,\enskip 1 \right)
				\end{eqnarray*}
			
				\begin{eqnarray*}
					U_{B} \left(x_{1}^{B},\enskip x_{2}^{B} \right) & = &  \left(x_{1}^{B} \right)^{2}\cdot x_{2}^{B}\\
					\left( w_{1}^{B},\enskip w_{2}^{B} \right) & = & \left(4,\enskip 10 \right) \\
					\left(p_1,\enskip p_2 \right) & = &\left(1,\enskip 1 \right)
				\end{eqnarray*}
			\end{multicols}
		  ¿Cuál será la decisión óptima de $A$ en esta situación?\\[0.5cm]
			\textbf{\LARGE Solución}\\
				Analizando al agente $A$, se debe cumplir la siguiente relación:
	\begin{itemize}
		\item $RMS_{A}\left(x_{1}^{A},\enskip x_{2}^{A} \right) = \frac{p_1}{p_2}\Longrightarrow \frac{x_{2}^{A}}{2x_{1}^{A}} = \frac{1}{1}$
		\item $p_{1}x_{1}^{A}+p_{2}x_{2}^{A} = p_{1}w_{1}^{A} + p_{2}w_{2}^{A} \Longrightarrow x_{1}^{A} + x_{2}^{A} = 16 + 4$
	\end{itemize}
Entonces 
	\begin{itemize}
		\item $x_{2}^{A}=2x_{1}^{A} \Longrightarrow  x_{1}^{A} + 2x_{1}^{A} = 3x_{1}^{A} = 16 + 4 = 20 \Longrightarrow x_{1}^{*A} = \frac{20}{3}$
		\item $x_{2}^{A}=2x_{1}^{A} = 2x_{1}^{*A} = \frac{20}{3} \Longrightarrow x_{2}^{*A} = \frac{40}{3}$
	\end{itemize}
La decisión optima de $A$:\\
	$$\therefore \left(x_{1}^{*A},\enskip x_{2}^{*A} \right) = \left(\frac{20}{3}, \enskip \frac{40}{3}\right)$$
Dada la dotación incial, para consumir lo anterior el condumiro $A$ tendrá que vender $\frac{28}{3}$ del bien 1 (una demanda neta de $\frac{20}{3} - 16$, negativa), y comprar $\frac{28}{3}$ del bien 2 (demanda neta $\frac{40}{3}-4$)\\

Analizando al agente $B$
			$$	\left.
					\begin{array}{l}
						\frac{2x_{2}^{B}}{x_{1}^{B}} = \frac{1}{1} \\ [.5cm]
						x_{1}^{B} + x_{2}^{B} = 4 + 10
					\end{array}
				\right\} \Longrightarrow \therefore \left(x_{1}^{*B},\enskip x_{2}^{*B} \right) = \left(\frac{28}{3}, \enskip \frac{14}{3}\right) $$
Comparando la deciión de $A$ y $B$. El consumidor $A$ necesita vender bien 1 y el $B$ necesita comprar, pero no en la misma cantidad. La demanda neta de bien 1 por parte de $B$ es de $\frac{14}{3}$, frente a los $\frac{28}{3}$ que $A$ quiere vender. Lo mismo, en sentido contrario ocurre en el bien 2.
	\item Precio de equilibrio. Con los datos del ejercicio (\ref{Ejercicio_5}), sin considerar los precios, hallar los precios de equilibrio.\\[0.5cm]
		\textbf{\LARGE Solución}\\
			La condición de óptimo:
		$$\sum RMS = \frac{P_{X}}{P_{Y}}$$
Teniendo en cuenta que:
	\begin{align*}
		Y_{1} + Y_{2} &= Y\\
		X_{1} = X_{2} &= X
	\end{align*}
Resolviendo:
	\begin{align*}
		\frac{Y_{1}}{2X_{1}} + \frac{Y_{2}}{2X_{2}} &= \frac{100}{0.2}\\
		\frac{Y}{2X} &= 500\\
		\frac{Y}{X} &= 1000
	\end{align*}
Hallando el nivel óptimo de X, usando la restricción monetaria conjunta:
	\begin{align*}
		P_{X}X + P_{Y}Y_{1} + P_{Y}Y_{2} &= M_{1} + M_{2}\\
		1000X + 0.2Y &= 2 \times 3000\\
		1000X + 0.2\left(1000X\right) &= 6000\\
		1200X &= 6000\\
		X^* &= 5
	\end{align*}
Por tanto el nivel óptimo del bien público es: $X^* = 5$.
	\item Sean dos consumidores $A$ y $B$ que tienen preferencias por los bienes $x$ e $y$ representadas por las funciones de utilidad $U_{A} = 2x_{A}y_{A}$, $U_{B} = 4x_{B}^{2}y_{B}$. Si las dotaciones existentes en la economía son $\overline{x} = 30,\enskip \overline{y} = 30$, que están repartidas inicialmente entre los consumidores como $w_{A} = (10, 10)$, $w_{B}=(20,20)$
		\begin{enumerate}[a)]
			\item Obtener las expresiones de la curva de contrato.
			\item ¿Es la dotación inicial eficiente en el sentido de Pâreto?
			\item Determinar los precios de equilibrio competitivo $p_{x}, p_{y}$, de esta economía de intercambio puro.
			\item Compruebe que la asignación de equilibrio verifica la \emph{Ley de Walras}.
		\end{enumerate}
				\vspace{0.5cm}
					\textbf{\LARGE Solución}\\
						\begin{enumerate}[a)]
	\item Dadas las tecnología Leontief de las empresas, no es posible resolver el problema de forma analítica. Si los resolvemos aplicando directamente el concepto de eficiencia productiva ,ayudándonos para ello de las representaciones gráficas de esta economía, la cual se hace en la siguiente figura:
		\begin{center}
			\begin{tikzpicture}
				% Curva de contrato
					\draw[purple] (0,0) -- (6,4);
					\draw[purple] (4.5,0) -- (6,4);
					
				% Relleno
					\fill [pattern=crosshatch dots,pattern color=purple!20!white] (0,0) -- (6,4) -- (4.5,0) -- (0,0);
					
				% Caja
					\draw[<->] (0,4.5) node[align=center, above, scale=0.8] {$L_1$} -- (0,0) node[below left] {\footnotesize $O_1$} -- (6.5,0) node[align=center, right, scale=0.8] {$K_1$};
					\draw[<->] (-0.5,4) node[align=center, left, scale=0.8] {$K_2$} -- (6,4) node[align=center, above right, scale=0.8] {$O_2$} -- (6,-0.5) node[align=center, below, scale=0.8] {$L_2$};
					
				% Union de puntos
					\draw[dashed, black, fill=black] (4.5,0) circle[radius=0.05] node [below, scale=0.7] {$A$} -- (4.5,3) circle[radius=0.05] node [above, scale=0.7] {$B$};
					
				% Curvas de indiferencia
					% Agente A
						\draw[blue] (3,2.5) -- (3,2) -- (3.5,2);
						\draw[blue] (2,2.5) -- (2,1.33) -- (3.17,1.33); 
					% Agente B
						\draw[red] (4.75,2) -- (5.25,2) -- (5.25,1.5);
						\draw[red] (3.82,1.33) -- (5,1.33) -- (5,0.16);
						
				% Texto
					\draw[purple] (3,0.5) node[align=center, above, scale=0.8] {Conjunto Paretiano};
			\end{tikzpicture}
		\end{center}
	Es evidente que la asignaciones eficientes en la producción son la dadas por los puntos del conjunto cerrado $O_1AO_2$. Es decir,
		\begin{gather}
			\left\lbrace \left( q_1, K_1, L_1\right) \right\rbrace = \left\lbrace 0 \leq K_1 \leq 15 \text{ y } L_1 \leq 0.5K_1\right\rbrace \cup \left\lbrace 15 \leq K_1 \leq 20 \text{ y }  0.5K_1\geq L_1 \geq 2K_1 - 30\right\rbrace \label{eq1}
		\end{gather}
	y donde el subconjunto $\{0 \leq K_1 \leq 15$ y $L_1 \leq 0.5K_1 \}$ es el representado gráficamente por el triángulo $O_1AB$, en tanto que el sobconjunto $\{15 \leq K_1 \leq 20$ y $ 0.5K_1\geq L_1 \geq 2K_1 - 30 \}$ es el dado por el triangulo $ABO_2$.
	
	\item Para determinar el $CPP$, lo que hacemos es tomar algunos puntos del conjunto de contrato (de la producción) definido en (\ref{eq1}) y representarlo en el espacio de productos $\left\lbrace q_1,q_2\right\rbrace$. Por ejemplo, a lo largo de la recta $L_1 = 0.5 K_1$, la correspondencia es la siguiente:
		\begin{center}
			\begin{tabular}{ccc}
				\hline
					$\left( K_1, L_1\right)$ & {} & $\left( q_1, q_2\right)$ \\
				\hline
					(0,0) 	 & {} & (0,10)	\\
					(6,3) 	 & {} & (6,7)	\\
					(10,5)   & {} & (10,5)	\\
					(15,7.5) & {} & (15,2.5) \\
				\hline
			\end{tabular}
		\end{center}
	y es evidente que la FPP es\footnote{Tómense los punto (0,10) y (15,2.5) y recuérdese la ecuación de una recta que pasa por dos puntos.}
			$$q_2 = 10 - 0.5q_1, \text{ si } 0 \leq q_1 \leq 15$$
	Análogamente, si del $CCP$ definido en (\ref{eq1}) tomamos algunos puntos a lo largo de la recta $L_1 = 2K_1 - 30$ y los proyectamos en el espacio de productos $\left\lbrace q_1,q_2\right\rbrace $, se llega a
		\begin{center}
			\begin{tabular}{ccc}
				\hline
					$\left( K_2, L_2\right)$ & {} & $\left( q_1, q_2\right)$ \\
				\hline
					(15,0) 	 & {} & (0,10)	\\
					(18,6) 	 & {} & (12,4)	\\
					(19,8)   & {} & (16,2)	\\
					(20,10)  & {} & (20,0)  \\
				\hline
			\end{tabular}
		\end{center}
	pudiéndose inferir que
			$$q_2 = 10 - 0.5q_1, \text{ si } 0 \leq q_1 \leq 20$$
	En definitiva, la $FPP$ de esta economía es el conjunto de producciones
			$$\left\lbrace \left(q_1, q_2 \right) \in \mathbb{R}_{+}^{2} \mid q_2 \left( q_1\right) = 10 -0.5q_1, 0 \leq q_1 \leq 20 \right\rbrace $$
	\item A partir de la tecnología de la empresa 1, es evidente que las demandas de factores de esta empresa son las dadas por
			$$\left( K_1, L_1\right) = \left( \frac{2q_1}{2r + w}, \frac{q_1}{2r + w}\right) $$
	ya que los inputs $K$ y $L$ pueden ser interpretados como un ``input compuesto'' cuyp coste es $2r+w$. Análogamente, las demandas de inputs de la empresa 2 son
			$$\left( K_2, L_2\right) = \left( \frac{q_2}{r + 2w}, \frac{2q_2}{r + 2w}\right) $$
	Luego, el par de precios $(r, w)$ tales que
			$$\frac{2q_1}{2r + w}+  \frac{q_1}{2r + w} \leq 20$$
	y
			$$\frac{q_2}{r + 2w} + \frac{2q_2}{r + 2w} \leq 10$$
	define el equilibrio walrasiano de esta economía de inputs complementarios.
\end{enumerate}
	\item considere una economía de intercambio puro con dos bienes $(x, y)$ y dos agente $(A, B)$, cuyas funciones de utilidad son:
			\begin{gather*}
				u^{A} = \ln(x_A) + 2y_A\\
				u^{B} = \ln(x_B) + 2y_B
			\end{gather*}
		  Además, se sabe que las dotaciones iniciales  $A$ y $B$ son, respectivamente $(1, 4)$ y $(6, 3)$. Determine la relación de precios walrasianos, la asignación de equilibrio e identifique el sentido del intercambio; es decir, las cantidades demandadas y ofertadas de cada bien por caga agente.\\[0.5cm]
		  	\textbf{\LARGE Solución}\\
		  		\vspace{-0.3cm}
\begin{center}
	\begingroup
		\setlength{\tabcolsep}{10pt}
		\renewcommand{\arraystretch}{1.5} 
			\begin{tabular}{ccc}
					\hline
				\multicolumn{3}{c}{Datos} \\
					\hline
				2 consumidores: & {} & $A$ y $B$ \\
				2 bienes: & {} &$x$ e $y$ \\
				F.U: & {} &$u^{A} = \ln(x_A) + 2y_A$ y $u^{B} = \ln(x_B) + 2y_B$ \\
				Dotaciones: & {} &$w^A = (1, 4)$ y $w^B = (6, 3)$ \\
					\hline
			\end{tabular}
	\endgroup
\end{center}
La curva de contrato:
	$$RMS^A = RMS^B \Longrightarrow x_A = x_B$$
además
	$$x_A + x_B = \overline{x} \Longrightarrow x_A + x_A = \overline{x} \Longrightarrow x_A = \frac{\overline{x}}{2} = x_B$$
	\begin{center}
		\begin{tikzpicture}[scale=0.8]
				% Formato de CAJA
					\draw[->] (0,0) node[align=center, below left] {\footnotesize $O_A$} -- (0,8) node[align=center, above] {\footnotesize $x_{2}^{A}$};
				\draw[->] (0,0) -- (8,0) node[align=center, right] {\footnotesize $x_{1}^{A}$};
				
					\draw[->] (7,7) node[align=center, above right] {\footnotesize $O_B$} -- (-1,7) node[align=center, left] {\footnotesize $x_{2}^{B}$};
				\draw[->] (7,7) -- (7,-1) node[align=center, below] {\footnotesize $x_{2}^{B}$};
						
			% Curvas de indiferencia
				\draw [blue] (1.4,6.9) .. controls (2.84,5.7) and (3.8,5.2) .. (5.6,4.8);
				\draw [red] (1.4,6.15) .. controls (3.03,5.8) and (3.99,5.38) .. (5.6,4.05);
				
				\draw [blue] (1.4,5) .. controls (2.84,3.8) and (3.8,3.3) .. (5.6,2.9);
				\draw [red] (1.4,4.25) .. controls (3.03,3.9) and (3.99,3.48) .. (5.6,2.15);
				
				\draw [blue] (1.4,3.1) .. controls (2.84,1.9) and (3.8,1.4) .. (5.6,1);
				\draw [red] (1.4,2.35) .. controls (3.03,2) and (3.99,1.58) .. (5.6,0.25);
			% Curvas de contrato
				\draw [purple, very thick] (3.5,0) node [below, scale= 0.6mm] {$\frac{\overline{x}}{2}$} -- (3.5,7);
				
			% Flechas
				\node[draw, single arrow,
					minimum height=28mm, minimum width=1mm,
					single arrow head extend=1.5mm,
					anchor=west, blue, fill=blue, scale=0.5, rotate=33] at (1.5,5.1) {};
				\node[draw, single arrow,
					minimum height=28mm, minimum width=1mm,
					single arrow head extend=1.5mm,
					anchor=west, red, fill=red, rotate=-147, scale=0.5] at (5.5,2.1) {};
		\end{tikzpicture}
	\end{center}

Probamos si las dotaciones iniciales son ESP.
	$$x_{A} = x_{B} = \frac{\overline{x}}{2} \Longrightarrow	1 \neq 6 \neq \frac{7}{2}$$
Se comprueba que la dotación inicial
	\begin{center}
		\begin{tikzpicture}[scale=0.8]
			% Formato de CAJA
				\draw[->] (0,0) node[align=center, below left] {\footnotesize $O_A$} -- (0,8) node[align=center, above] {\footnotesize $x_{2}^{A}$};
				\draw[->] (0,0) -- (8,0) node[align=center, right] {\footnotesize $x_{1}^{A}$};
			
				\draw[->] (7,7) node[align=center, above right] {\footnotesize $O_B$} -- (-1,7) node[align=center, left] {\footnotesize $x_{2}^{B}$};
				\draw[->] (7,7) -- (7,-1) node[align=center, below] {\footnotesize $x_{2}^{B}$};
			
			% Curvas de indiferencia
				\draw [blue] (1.4,6.9) .. controls (2.84,5.7) and (3.8,5.2) .. (5.6,4.8);
				\draw [red] (1.4,6.15) .. controls (3.03,5.8) and (3.99,5.38) .. (5.6,4.05);
				
				\draw [blue] (1.4,5) .. controls (2.84,3.8) and (3.8,3.3) .. (5.6,2.9);
				\draw [red] (1.4,4.25) .. controls (3.03,3.9) and (3.99,3.48) .. (5.6,2.15);
				
				\draw [blue] (1.4,3.1) .. controls (2.84,1.9) and (3.8,1.4) .. (5.6,1);
				\draw [red] (1.4,2.35) .. controls (3.03,2) and (3.99,1.58) .. (5.6,0.25);
				
			% Curvas de contrato
				\draw [purple, very thick] (3.5,0) node [below, scale= 0.6mm] {$\frac{\overline{x}}{2}$} -- (3.5,7);
			
			% Flechas
				\node[draw, single arrow,
						minimum height=28mm, minimum width=1mm,
						single arrow head extend=1.5mm,
						anchor=west, blue, fill=blue, scale=0.5, rotate=33] at (1.5,5.1) {};
				\node[draw, single arrow,
						minimum height=28mm, minimum width=1mm,
						single arrow head extend=1.5mm,
						anchor=west, red, fill=red, rotate=-147, scale=0.5] at (5.5,2.1) {};
			
			% Intersección
				\draw[dashed] (1,7) node[above] {\footnotesize $x_{1}^{B}=6$} -- (1,0) node[below] {\footnotesize $x_{1}^{A}=1$};
				\draw[dashed] (0,4) node[left] {\footnotesize $x_{2}^{A}=4$} -- (7,4)node[right] {\footnotesize $x_{2}^{B}=3$};
			
			% Punto
				\draw[black, fill=black] (1,4) circle[radius=0.08] node[align=center, above left] {\footnotesize $w$};
		\end{tikzpicture}
	\end{center}

Con la inclusión de precios, la situación es la siguiente:
	$$RMS^A = RMS^B = \frac{p_x}{p_y}$$
	$$RMS^A=\frac{1}{2x_A}=\frac{p_x}{p_y} \quad , \quad RMS^B=\frac{1}{2x_B}=\frac{p_x}{p_y} \Longrightarrow x_A = x_B = \frac{p_y}{2p_x}$$
	
	$$
		\begin{array}{ccc}
			p_xx_A + p_yy_A = p_x + 4p_y & {} &p_xx_B + p_yy_B = 6p_x + 3p_y\\[0.4cm]
			p_x\frac{p_y}{2p_x} + p_yy_A = p_x + 4p_y & {} &p_x\frac{p_y}{2p_x} + p_yy_B = 6p_x + 3p_y\\[0.4cm]
			\frac{p_y}{2} + p_yy_A = p_x + 4p_y & {} & \frac{p_y}{2} + p_yy_B = 6p_x + 3p_y \\[0.4cm]
			y_{A}^* = \frac{2p_x + 7p_y}{2p_y} & {} & y_{B}^* = \frac{12p_x + 5p_y}{2p_y}
		\end{array}
	$$
	
	$$y_{A}^* + y_{B}^* = 4 + 3 = 7 \Longrightarrow \frac{2p_x + 7p_y}{2p_y} + \frac{12p_x + 5p_y}{2p_y} = 7  \Longrightarrow  \frac{p_x}{p_y} = \frac{1}{7}$$
	
	\begin{gather*}
		\therefore y_{A}^* = \frac{51}{14}  \Longrightarrow x_{B}^* = \frac{7}{2}\\[0.4cm]
		\therefore y_{b}^* = \frac{47}{14}  \Longrightarrow x_{B}^* = \frac{7}{2}
	\end{gather*}

	\begin{center}
		\begin{tikzpicture}[scale=0.8]
			% Formato de CAJA
				\draw[->] (0,0) node[align=center, below left] {\footnotesize $O_A$} -- (0,8) node[align=center, above] {\footnotesize $x_{2}^{A}$};
				\draw[->] (0,0) -- (8,0) node[align=center, right] {\footnotesize $x_{1}^{A}$};
				
				\draw[->] (7,7) node[align=center, above right] {\footnotesize $O_B$} -- (-1,7) node[align=center, left] {\footnotesize $x_{2}^{B}$};
				\draw[->] (7,7) -- (7,-1) node[align=center, below] {\footnotesize $x_{2}^{B}$};
			
			% Curvas de indiferencia
				\draw [blue] (1.4,6.9) .. controls (2.84,5.7) and (3.8,5.2) .. (5.6,4.8);
				\draw [red] (1.4,6.15) .. controls (3.03,5.8) and (3.99,5.38) .. (5.6,4.05);
				
				\draw [blue] (1.4,5) .. controls (2.84,3.8) and (3.8,3.3) .. (5.6,2.9);
				\draw [red] (1.4,4.25) .. controls (3.03,3.9) and (3.99,3.48) .. (5.6,2.15);
				
				\draw [blue] (1.4,3.1) .. controls (2.84,1.9) and (3.8,1.4) .. (5.6,1);
				\draw [red] (1.4,2.35) .. controls (3.03,2) and (3.99,1.58) .. (5.6,0.25);
			
			% Curvas de contrato
				\draw [purple, very thick] (3.5,0) -- (3.5,7);
			
			% Intersección
				\draw[dashed] (1,7) node[above] {\footnotesize $x_{1}^{B}=6$} -- (1,0) node[below] {\footnotesize $x_{1}^{A}=1$};
				\draw[dashed] (0,4) node[left] {\footnotesize $x_{2}^{A}=4$} -- (7,4)node[right] {\footnotesize $x_{2}^{B}=3$};
			
			% Punto
				\draw[black, fill=black] (1,4) circle[radius=0.08] node[align=center, above left] {\footnotesize $w$};
				\draw[black, fill=black] (3.5,3.58) circle[radius=0.08] node[align=center, below left] {\footnotesize $EGW$};
			
			% Recta presupuestaria
				\draw (0,5.26) -- (7,1.9) node[right] {\footnotesize $RP$};
			
			% Intersección
				\draw[dashed] (3.5,7) node[above] {\footnotesize $x_{1}^{B}=3.5$} -- (3.5,0) node[below] {\footnotesize $x_{1}^{A}=3.5$};
				\draw[dashed] (0,3.58) node[left] {\footnotesize $x_{2}^{A}=3.64$} -- (7,3.58)node[right] {\footnotesize $x_{2}^{B}=3.36$};
		\end{tikzpicture}
	\end{center}
	\item En una economía de intercambio puro, en donde existe dos bienes $(x, y)$ y dos consumidores $A$ y $B$, las funciones de utilidad son:
				\begin{gather*}
					U_A (x,y) = \alpha \ln(x_A) + (1-\alpha)\ln(x_A)\\
					U_B (x,y) = \text{Min} \left\lbrace x_B, y_B\right\rbrace 
				\end{gather*}
			El consumidor $A$ posee una dotación inicial de una unidad de $y$ y el consumidor $B$, de una unidad de $x$.
				\begin{enumerate}[a)]
					\item Halle los precios relativos y las demandas de equilibrio
					\item ¿Entre qué valores puede encontrarse la constante $a$?
				\end{enumerate}
						\vspace{0.5cm}
					\textbf{\LARGE Solución}\\
						\begin{itemize}
	\item Pregunta 9a:
			Para el agente $A$:
				\begin{align*}
					& \text{Max } \quad U_A = \alpha \ln(x_A) + (1-\alpha)\ln(x_A) \\[0.2cm]
					& \begin{array}{ll}
						\text{s.a: } & p_xx_A + p_yy_A = \overline{y}p_y + \overline{x}p_x \qquad (\overline{y}=1, \overline{x}=0)
					\end{array}
				\end{align*}
			
			Se forma el lagrange y se aplican las condiciones de primer orden:
			$$ \mathscr{L} =  \ln(x_A) + (1-\alpha)\ln(x_A) - \lambda\left[p_y - p_xx_A - p_yy_A  \right]$$
			
				$$\left.
					\begin{array}{l}
						\frac{\partial \mathscr{L}}{\partial x_{A}} = \frac{\alpha}{x_A} - \lambda p_x= 0\\[0.4cm]
						\frac{\partial \mathscr{L}}{\partial y_{A}} = \frac{(1-\alpha)}{y_A} - \lambda p_y=0
					\end{array}
				  \right\} \Longrightarrow 
				    \begin{array}{ccc}
						\frac{\alpha}{x_Ap_x} & = & \frac{(1-\alpha)}{y_Ap_y}  \\[0.3cm]
				  		y_A & = & \frac{(1-\alpha)}{\alpha p_y}x_Ap_x
				    \end{array}$$
			
				\begin{align*}
					p_xx_A + p_yy_A & = p_y\\[0.3cm]
					p_xx_A + p_y\left( \frac{(1-\alpha)}{\alpha p_y}x_Ap_x\right)  & = p_y\\[0.3cm]
					x_{A}^* & = \frac{\alpha p_y}{p_x}\\[0.3cm]
					y_{A}^* & = 1 - \alpha
				\end{align*}
			
			Para el agente $B$:
				\begin{align*}
					& \text{Max } \quad U_B = \text{Min} \left\lbrace x_B, y_B\right\rbrace \\[0.2cm]
					& \begin{array}{ll}
						\text{s.a: } & p_xx_A + p_yy_A = \overline{y}p_y + \overline{x}p_x  \qquad (\overline{y}=0, \overline{x}=1)
					  \end{array}
				\end{align*}
			
			Como $x$ e $y$ son complementarios perfectos
				$$x_B = y_B$$
			Y reemplazamos en la recta presupuestaria
				\begin{align*}
					p_xx_B + p_yy_B & = p_x\\[0.3cm]
					p_xx_B + p_yx_B  & = p_x\\[0.3cm]
					x_B\left( p_x + p_y\right) &= p_x\\[0.3cm]
					x_{B}^* & = \frac{p_x}{p_x+p_y} = y_{B}^*
				\end{align*}
			El criterio de la condición de factibilidad nos indica que los mercados se limpian; es decir, la demanda debe ser igual a la oferta (dotación)
				\begin{itemize}
					\item $x_a + x_B = \overline{x}_A + \overline{x}_B \Longrightarrow (1 - \alpha) + \frac{p_x}{p_x+p_y} = 0 + 1$
					\item $y_a + y_B = \overline{x}_A + \overline{x}_B \Longrightarrow \frac{\alpha p_y}{p_x} + \frac{p_x}{p_x+p_y} = 1 + 0$
				\end{itemize}
			Usando la expresión de $y$, los precios relativos son:
				$$\frac{p_y}{p_x} = \frac{1-\alpha}{\alpha}$$
	\item Pregunta 9b:
			$$\nexists \enskip p < 0 \enskip \Longrightarrow \enskip \frac{1 - \alpha}{\alpha} > 0 \enskip \Longrightarrow \enskip  \alpha \in <0,1>$$
\end{itemize}
	\item Una economía de intercambio puro de dos agentes $A$ e $B$ y dos bienes $x$ e $y$ presentan las siguientes características:
				\begin{align*}
					& U_A = \left\lbrace 
								\begin{array}{ccccccc}
									\text{Min}\{x,y\} & , & 0 & \leq & \text{Min}\{x,y\} & \leq & 1 \\
											1         & , & 1 &  <   & \text{Min}\{x,y\} &   <  & 2 \\
									\text{Min}\{x,y\} & , & 1 &  <   & \text{Min}\{x,y\} &   {} & {}\\
								\end{array}
							\right. ; \text{Dotaciones: } x = 1 , y = 2\\[0.3cm]
					& U_B = x + y ;\text{Dotaciones: } x = 2 , y = 1 
				\end{align*}
			Grafique la situación inicial de los agentes en la caja de Edgeworth (las dimensiones de la caja, la forma de la curva de indifenrecia y la posción del nivel de dotaciones iniciales). Indique la curva de contrato.\\[0.5cm]
				\textbf{\LARGE Solución}\\
					El punto $E$ indica la situación inicial de ambos consumidores. En este punto, el agente $A$ tiene una dotación de $(1,2)$; mientras que el $B$ una de $(2,1)$. La curva de contrato viene dada por la línea gruesa y el área morada señalada en el siguiente gráfico.

\begin{center}
	\begin{tikzpicture}[scale=1.7]
		% Formato de CAJA
		\draw[->] (0,0) node[align=center, below left] {\footnotesize $O_A$} -- (0,4) node[align=center, above] {\footnotesize $y_A$};
		\draw[->] (0,0) -- (4,0) node[align=center, right] {\footnotesize $x_{A}$};
		
		\draw[->] (3,3) node[align=center, above right] {\footnotesize $O_B$} -- (-1,3) node[align=center, left] {\footnotesize $x_{B}$};
		\draw[->] (3,3) -- (3,-1) node[align=center, below] {\footnotesize $y_{B}$};
		
		% 
		
		% Curvas de indiferencia
		\draw (0.5,3) -- (0.5,0.5) -- (3,0.5);
		\draw (2.5,3) -- (2.5,2.5) -- (3,2.5);
		
		\draw (2,3) -- (3,2);
		\draw (0,2.5) -- (2.5,0);
		\draw (0.7,3) -- (3,0.7);
		\draw (0,1) -- (1,0);
		
		% Curva de contrato       
		\draw [purple, very thick] (0,0)  -- (1,1);
		\draw [purple, very thick] (2,2)  -- (3,3);
		\draw [purple, very thick] (1,1)  -- (1,3) -- (2,3) -- (2,2) -- (3,2) -- (3,1) -- (1,1);
		\fill [color=purple!10](1,1)  -- (1,3) -- (2,3) -- (2,2) -- (3,2) -- (3,1) -- (1,1);
		
		% Punto
		\draw[black, fill=black] (1,2) circle[radius=0.04] node[align=center, right, scale = 0.25mm] {\footnotesize $E(1,2)$};
	\end{tikzpicture}
\end{center}
\end{enumerate}

%------------------------------------------------------------------------------------
\end{document}		
%====================================================================================