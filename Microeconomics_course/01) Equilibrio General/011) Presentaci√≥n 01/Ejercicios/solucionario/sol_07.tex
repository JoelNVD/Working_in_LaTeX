\vspace{-0.3cm}
\begin{center}
	\begingroup
		\setlength{\tabcolsep}{10pt} % Default value: 6pt
		\renewcommand{\arraystretch}{1.5} % Default value: 1
			\begin{tabular}{ccc}
					\hline
				\multicolumn{3}{c}{Datos} \\
					\hline
				2 consumidores: & {} & $A$ y $B$ \\
				2 bienes: & {} &$x$ e $y$ \\
				F.U: & {} &$U_{A} = 2x_{A}y_{A}$ y $U_{B} = 4x_{B}^{2}y_{B}$ \\
				Dotaciones: & {} &$\overline{x} = 30$ y $\overline{y} = 30$ \\
				Reparto: & {} &$w_{A} = (10, 10)$ y $w_{B}=(20,20)$\\
					\hline
			\end{tabular}
	\endgroup
\end{center}

\begin{itemize}
	\item Pregunta 7a:
			\begin{gather*}
				RMS^{A} = RMS^{B} \Rightarrow \frac{y_A}{x_A}	= \frac{2y_B}{x_B} \\
				x_A + x_B = \overline{x} = \overline{x}_A + \overline{x}_B = 10 + 20 = 30 \Rightarrow x_B = 30 - x_A\\
				y_A + y_B = \overline{y} = \overline{y}_A + \overline{y}_B = 10 + 20 = 30 \Rightarrow y_B = 30 - y_A\\
				\frac{y_A}{x_A}	= \frac{2\left(30 - y_A \right) }{30 -x_A}\\
				y_A = \frac{60x_A}{30+x_A}
			\end{gather*}
			Gráfica de la función:
				$$ \frac{\partial y_A}{\partial x_A} = \frac{1800}{\left( 30 - x_A\right)^2 } > 0 \qquad \frac{\partial^2 y_A}{\partial x_{A}^{2}} = - \frac{3600}{(30-x_A)^3} < 0$$
				
				\begin{center}
					\begin{tikzpicture}[scale=0.8]
						% Curvas de contrato
							\draw [purple] (0,0) .. controls (0.2,3.8) and (2.2,5.8) .. (6,6);
						% Formación de la caja
							% Consumidor A
								\draw (0,0) node[align=center, below left] {\footnotesize $O_A$} -- (0,6) node[above left] {30};
								\draw (0,0) -- (6,0) node[below right] {30};
							%Consumidor B
								\draw (6,6) node[align=center, above right] {\footnotesize $O_B$} -- (0,6);
								\draw (6,6) -- (6,0);
					\end{tikzpicture}
				\end{center}
	\item Pregunta 7b:
				$$\frac{y_A}{x_A}	= \frac{2y_B}{x_B} \Longrightarrow \frac{10}{10} = \frac{2 \cdot 20}{20} \Longrightarrow 1 \neq 2$$
			La dotación inicial  no es asignación \emph{Eficiente en el Sentido de Pareto} (ESP). Por otro lado, si la dotación inicial fuese ESP debería pertenecer a la curva de contrato; es decir:
				$$y_A = \frac{60x_A}{30+x_A} \Longrightarrow 10 \neq 15 = \frac{60 \cdot 10}{30 +10}$$
			$\therefore$ Se afirma que no es ESP.
	\item Pregunta 7c:
				$$RMS^{A} = RMS^{B} = \frac{p_x}{p_y}$$
				
				$$\begin{array}{ccc}
					x_Ap_x + y_Ap_y	= 10p_x + 10p_y & {} & x_Bp_x + y_Bp_y	= 20p_x + 20p_y   \\[.5cm]
					x_{A}^{*}=\frac{5p_x+5p_y}{p_x} & {} & x_{B}^{*}=\frac{40p_x+40p_y}{3p_x} \\[.5cm]
					y_{A}^{*}=\frac{5p_x+5p_y}{p_y} & {} & y_{B}^{*}=\frac{20p_x+20p_y}{3p_y}
				\end{array}$$
			
				$$x_A + x_B = 30 \Longrightarrow x_{A}^{*} + x_{B}^{*} = 30 \Longrightarrow \frac{5p_x+5p_y}{p_x} + \frac{40p_x+40p_y}{3p_x} = 30 \Longrightarrow \left( \frac{p_x}{p_y} \right)^{*}= \frac{110}{70}$$
				
			Si evaluas con $y \Longrightarrow \left( \frac{p_y}{p_x}\right)^{*} = \frac{70}{110}$\\
			
			Reemplazando $\left( \frac{p_x}{p_y} \right)^{*}$ o $\left( \frac{p_y}{p_x}\right)^{*}$ en las funciones de demanda (FD); es decir, en $x_{A}^{*}, y_{A}^{*}, x_{B}^{*}, y_{B}^{*}$ los resultados serán:
				$$x_{A}^{*} = \frac{90}{11}, \enskip y_{A}^{*} = \frac{240}{11}, \enskip x_{B}^{*} = \frac{90}{7}, \enskip y_{B}^{*} = \frac{120}{70}$$
			
				\begin{nota}{Nota 1}
					Con las F.U Cobb-Douglass $U=aX^{\alpha}Y^{\beta}$
						$$ X^* = \frac{\alpha M}{(\alpha + \beta)P_X} \quad Y^* = \frac{\beta M}{(\alpha + \beta) P_Y} $$
					Donde $M = XP_X+YP_Y = W_XP_X + W_YP_Y$
				\end{nota}
			
				\begin{nota}{Nota 2}
					Si se conoce $X_{A}^*$ y $Y_{A}^*$ solo reemplazas en:
						\begin{gather*}
							X_{B}^* = 30 - X_{A}^* \\
							Y_{B}^* = 30 - Y_{A}^*
						\end{gather*}
					para hallas fácilmente $X_{B}^*$ y $Y_{B}^*$
				\end{nota}
	\item Pregunta 7d:
				\begin{center}
					\emph{Ley de Walras}: $ED = P_x z_x+P_y z_y=0$ \\[0.4cm]
					$z_x$: ED en el mercado del bien $x$\\[0.4cm]
					$z_y$: ED en el mercado del bien $y$
				\end{center}
			Entonces:
				\begin{gather*}
					z_x = x - \overline{x} = x^* - \overline{x}  = \left(x_{A}^* + x_{B}^* \right) - \left(\overline{x}_A + \overline{x}_B \right) = \left( x_{A}^* - \overline{x}_A\right) + \left( x_{B}^* - \overline{x}_B\right) = z_{A}^{x} + z_{B}^{x} \\[0.4cm]
					z_y = y - \overline{y} = y^* - \overline{y}  = \left(y_{A}^* + y_{B}^* \right) - \left(\overline{y}_A + \overline{y}_B \right) = \left( y_{A}^* - \overline{y}_A\right) + \left( y_{B}^* - \overline{y}_B\right) = z_{A}^{y} + z_{B}^{y}
				\end{gather*}
			Haciendo la suma respectiva:
				\begin{center}
					\begin{tikzpicture}
						\matrix (m) [matrix of math nodes,]
							{
								z_{1}^{A} & = & x_{1}^{*A} - w_{1}^{A} = \frac{90}{11} - 10 & = & -\frac{20}{11}\\
								z_{1}^{B} & = & x_{1}^{*B} - w_{1}^{B} = \frac{240}{11} - 20 & = & \frac{20}{11}\\
								z_{2}^{A} & = & x_{2}^{*A} - w_{2}^{A} = \frac{90}{7} - 10 & = & \frac{20}{7} \\
								z_{2}^{B} & = & x_{2}^{*B} - w_{2}^{B} = \frac{120}{7} - 20 & = &-  \frac{20}{7} \\
								{}        & {} &              {}                           & {} &      0       \\
							};	
						\draw[thick, ->] (2.8,1.5) -- (2.8,0.25) node [right, very thick, scale=0.7mm] {+}-- (2.8,-1);
						\draw[thick] (1.8,-1.05) -- (2.6,-1.05);
					\end{tikzpicture}
				\end{center}
				\vspace{-0.5cm}
			$\therefore$ Se cumple la \emph{Ley de Walras}
				\begin{nota}{Nota 3}
					Todo EW es un OP, pero no todo OP es un EW
				\end{nota}
			

\end{itemize}
