%====================================================================================
\section{Actividades}
%====================================================================================
\begin{frame}{Actividad 1}
	Considere una economía der intercambio con dos consumidores y dos bienes en la cual las preferencias son $u_{1}\left( x_{11},x_{12}\right) = x_{11}^{3}x_{12}, u_{2}\left( x_{21},x_{22}\right) = x_{21}x_{22}$ y la dotación agregada es $\overline{w} = (16,16)$. Se tiene las siguientes asignaciones de consumo:
			\bigskip
		\begin{enumerate}
			\item $\left( x_{11},x_{12}\right) = (10,4)$ y $\left( x_{21},x_{22}\right) = (8,8 )$
			\item $\left( x_{11},x_{12}\right) = (10,4)$ y $\left( x_{21},x_{22}\right) = (8,12)$
			\item $\left( x_{11},x_{12}\right) = (12,8)$ y $\left( x_{21},x_{22}\right) = (4,6 )$
			\item $\left( x_{11},x_{12}\right) = (12,4)$ y $\left( x_{21},x_{22}\right) = (3,10)$
		\end{enumerate}
			\bigskip
	Determine qué asignación son Pareto Óptimas. Para las que no son, describa qué tipo de intercambio daría lugar a una mejora paretiana.
\end{frame}
%------------------------------------------------
\begin{frame}{Actividad 1}
	\begin{block}{\textbf{Óptimo de pareto}}
		\begin{enumerate}
			\item \textbf{Condición de factibilidad}
			\item \textbf{Condición de optimización}
		\end{enumerate}
	\end{block}
	
	\textcolor{red}{\textbf{Condición de factibilidad}}
		$$\sum x = \sum w$$
	La condición de factibilidad para este ejercicio queda de la siguiente manera:
		\begin{gather*}
			x_{1}^{A} + x_{1}^{B} = w_{1}^{A} + w_{1}^{B}\\
			x_{2}^{A} + x_{2}^{B} = w_{2}^{A} + w_{2}^{B}
		\end{gather*}
\end{frame}
%------------------------------------------------
\begin{frame}{Actividad 1}
	\textbf{Datos: $w_{1}^{A} + w_{1}^{B} = 16$ , $w_{2}^{A} + w_{2}^{B}=16$}
		$$\begin{array}{ccc}
			x_{11} + x_{21}=16 & , & x_{12} + x_{22}=16\\[0.2cm]
			10 + 8 \neq 16 & , & 4 + 8  \neq 16\\
			10 + 8 \neq 16 & , & 4 + 12 = 16\\
			12 + 4 = 16    & , & 8 + 6  \neq 16\\
			12 + 3 \neq 16 & , & 4 + 10 \neq 16
		\end{array}$$
	La condición de factibilidad debe cumplir simultáneamente y no de manera parcial como en el ejercicio. En adición, se observa situaciones de Excesos de Demanda (ED) y Exceso de Oferta (EO):
		$$\begin{array}{ccc}
			18 > 16 \text{ ED} & , & 12 < 16 \text{ EO} \\
			18 > 16 \text{ ED} & , & \text{Factible}    \\
			\text{Factible}    & , & 14 < 16 \text{ EO} \\
			15 > 16 \text{ ED} & , & 14 < 16 \text{ EO}
		\end{array}$$
\end{frame}
%------------------------------------------------
\begin{frame}{Actividad 2}
	Sea una economía de intercambio en la que hay 2 bienes $(x, y)$ en cantidades $\overline{x}=10$ e $\overline{y}=10$ y dos consumidores $(A, B)$ con sus funciones de utilidad $u_{A}=x_{A}^{2}y_{A}$,  $u_{B}=x_{B}y_{B}^{2}$
		\begin{enumerate}
			\item Considerando los siguientes estados, determine los que son factibles\\
				\begin{table}
					\centering
						\begin{tabular}{|c|c|c|}
								\hline
							Estados & $\left( x_{A},y_{A}\right)$ & $\left( x_{B},y_{B}\right)$ \\
								\hline
							1 & (1,6) & (9,4) \\
							2 & (5,5) & (5,5) \\
							3 & (5,2) & (5,8) \\
							4 & (1,8) & (9,4) \\
								\hline
						\end{tabular}
				\end{table}
			\item Calcular los niveles de utlidad para cada consumidor en cada estado y comparar cada par de estados, explicitando si existe superioridad de uno de ellos.
			\item Determine todos los estados óptimos en sentido de Pareto de esta economía e indique si alguno de los 4 estados anteriores es una de ellos.
		\end{enumerate}
\end{frame}
%------------------------------------------------
\begin{frame}{Actividad 2}
	\textbf{Dotaciones: $\overline{x}=10$ , $\overline{y}=10$ \qquad FU: $u_{A}=x_{A}^{2}y_{A}$ , $u_{B}=x_{B}y_{B}^{2}$}
	\begin{enumerate}
		\item Por la condición de factibilidad, el estado 1, 2 y 3 cumplen.
		\item Multiplicando $xy$, obtenemos los siguiente resultados
				$$\begin{array}{ccc}
					u_{1}^{A} = 6   & < & u_{1}^{B}=144\\
					u_{2}^{A} = 125 & = & u_{2}^{B}=125\\
					u_{3}^{A} = 50  & < & u_{3}^{B}=320\\
					u_{4}^{A} = 8   & < & u_{4}^{B}=144
				\end{array}$$
		\item \textcolor{red}{\textbf{Condición de optimalidad}}
				\begin{gather*}
						TMS^{A} = TMS^{B} \rightarrow RMS^{A} = RMS^{B} \rightarrow \frac{2y_{A}}{x_{A}} = \frac{y_{B}}{2x_{B}}
				\end{gather*}
			  Si reemplazas los datos de las cestas factibles, solo cumple el estado 3
				  \begin{alertblock}{\textbf{Nota}}
				  	\textbf{Todo OP cumple la CF, pero no toda CF cumple OP}
				  \end{alertblock}
	\end{enumerate}
\end{frame}
