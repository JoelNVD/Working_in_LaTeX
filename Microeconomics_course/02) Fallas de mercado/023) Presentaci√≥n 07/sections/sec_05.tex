%====================================================================================
\section[Señalización]{Señalización}
%====================================================================================
\begin{frame}{Señalización}
	Una forma de prevenir la Selección Adversa es la Señalización. Ejemplo:La Educación como señal Spence (En el mercado de trabajo existen dos tipos de trabajadores: de alta capacidad y de baja capacidad.
		\begin{itemize}
			\item El producto marginal de un trabajador de alta capacidad es a $a_H$
			\item El producto marginal de un trabajador de baja capacidad es a $a_L$
			\item $a_L < a_H$
		\end{itemize}
	Una fracción $\theta$ de trabajadores es de alta capacidad. Por tanto $1-\theta$ es la proporción de trabajadores de baja capacidad
\end{frame}
%------------------------------------------------
\begin{frame}{Señalización}
	Cada trabajador es pagado según su prodctividad marginal. Si las empresas conocieran el tipo de cada trabajador
		\begin{itemize}
			\item Pagarían $w^H = a^H$ a cada trabajador de alta capacidad
			\item Pagarían $w^L = a^L$ a cada trabajador de baja capacidad
		\end{itemize}
	Si las empresas no pueden distinguir el tipo al que pertenece cada trabajador (en el momento de la contratación), y tampoco pueden observar su productividad individual, entonces pagan un salario igual a la productividad marginal esperada de cada trabajador:
		$$w_p = (1-h)a_L + ha_H$$
\end{frame}
%------------------------------------------------
\begin{frame}{Señalización}
	Cuando esto ocurre decimos que el mercado de trabajo se encuentra en una situación de Equilibrio Aunador o Unificador ( Pooling Equilibrium)\\[0.3cm]
	El equilibrio aunador es ineficiente y, además implica que los buenos trabajadores están subsidiando implícitamente el salario de los menos buenos.\\[0.3cm]
	Con el fin de seleccionar los trabajadores con niveles de capacidad adecuados, las empresas pueden exigir un cierto nivel de educación que actuaría como señal en un contexto donde la información sobre los candidatos es escasa.
\end{frame}
%------------------------------------------------
\begin{frame}{Señalización}
	La señal (educación) debe ser costosa de tal forma que solamente los trabajadores más capacitados puedan conseguirla. Sean:
		\begin{itemize}
			\item $c_H =$ los costos de la educación para un trabajador de alta capacidad
			\item $c_L=$ los costos de la educación para un trabajador de baja capacidad
			\item $c_L > c_H$: los costos de la educación no están al alcance de los trabajadores menos hábiles.
		\end{itemize}
	Supuesto: la educación no tiene efecto sobre la productividad de los trabajadores pero indica, en un mundo de información incompleta, la capacidad de los trabajadores a esforzarse
\end{frame}
%------------------------------------------------
\begin{frame}{Señalización}
	Los trabajadores de alta capacidad adquieren e H unidades de educación si 
		\begin{align}
			&w_H - w_L = a_H - a_L > c_He_H \label{eq9}\\
			&w_H - w_L = a_H - a_L < c_Le_L \label{eq10}
		\end{align}
	(\ref{eq9}) significa que adquirir $e_H$ unidades de educación beneficia a los trabajadores con mayores habilidades\\
	(\ref{eq10}) significa que adquirir $e_H$ unidades de educación perjudica a trabajadores con menore habilidades
\end{frame}
%------------------------------------------------
\begin{frame}{Señalización}
		$$a_H - a_L > c_He_H \qquad a_H - a_L < c_Le_L$$
	Juntos requieren:
		$$\frac{a_H-a_L}{c_L} < e_H < \frac{a_H-a_L}{c_H}$$
	Trabajadores con capacidad baja elijen no adquirir educación y trabajadores con capacidad alta elijen adquirir educación\\[0.3cm]
	
	Se obtiene así un equilibrio \underline{separador}: los diferentes tipos de individuos emprenden diferentes acciones).
\end{frame}