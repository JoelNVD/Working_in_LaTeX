%====================================================================================
\section[Akerlof]{El mercado de autos usados (artículo de Akerlof}
%====================================================================================

\begin{frame}{El artículo de Akerlof}
	Fue el primero que introdujo el concepto de información asimétrica\\[0.3cm]
	Su argumento básico es que en muchos mercados el comprador emplea alguna estadística del mercado para medir el valor de una clase de bienes\\[0.3cm]
	Entonces , el comprador mira el promedio del mercado completo mientras el vendedor tiene más conocimiento privado de un aspecto específico\\[0.3cm]
	Según Akerlof esta información asimétrica le da al vendedor un incentivo para vender bienes por debajo de la calidad media del mercado\\[0.3cm]
	
\end{frame}
%------------------------------------------------
\begin{frame}{El artículo de Akerlof}
	Entonces , la calidad promedio de los bienes en el mercado se reducirá con el tamaño del mercado\\[0.3cm]
	Tales diferencias en retornos sociales y privados puede ser mitigado por un número de diferentes instituciones del mercado\\[0.3cm]
	Akerlof empieza asumiendo un modelo del mercado de automóviles , donde hay cuatro tipos de autos:
		\begin{itemize}
			\item Hay autos nuevos y autos viejos ,
			\item En cada uno , pueden ser autos buenos o malos
			\item Los autos malos son popularmente conocidos como ``limones'' en EEUU.
		\end{itemize}

\end{frame}
%------------------------------------------------
\begin{frame}{El artículo de Akerlof}
	Existen dos tipos de agentes: Compradores y Vendedores
		\begin{enumerate}
			\item Compradores
				\begin{itemize}
					\item Función de utilidad: $U_v = M +1.5qn$
						\begin{itemize}
							\item $M$ = Consumo de otros bienes
							\item $q$ = Calidad del auto
							\item $n$ = $\left\lbrace 0,1\right\rbrace $ 1 compra auto, 0 no compra
						\end{itemize}
					\item Restricción presupuestaria: $y_v = M + pn$
						\begin{itemize}
							\item $y_c$ = Ingreso
							\item $p$ = Precio del auto
						\end{itemize}
				\end{itemize}
			La calidad del auto $(q)$ es información privada del vendedor. El comprador aprecia $``q''$ como la calidad media del mercado, mientras que el vendedor distingue la de cada auto
		\end{enumerate}
\end{frame}
%------------------------------------------------
\begin{frame}{El artículo de Akerlof}
	Dada la incertidumbre de $``q''$ el comprador maximiza su utilidad esperada:
		\begin{itemize}
			\item $E(U_c)=M+1.5E(q)n \quad E(q) = \mu =$ Calidad media observada
		\end{itemize}
	Sustituyendo $M$ de la $r.p .$ $\longrightarrow M =y_c - pn$ 
		\begin{itemize}
			\item $E(U_c) = y_c - pn + 1.5\mu n \longrightarrow $  factor común a ``$n$''
			\item $E[U_c] = y_c +\left[ 1.5\mu - p\right]n$
		\end{itemize}
	Decisión:
		\begin{itemize}
			\item $n=1$ compra si percibe la calidad del auto por encima del precio; es decir, si $1.5\mu \geq p$
			\item $n=0$ no compra si percibe la calidad del auto por encima del precio; es decir, si $1.5\mu \leq p$
		\end{itemize}
\end{frame}
%------------------------------------------------
\begin{frame}{El artículo de Akerlof}
	\begin{enumerate}[2]
		\item Vendedores
			\begin{itemize}
				\item Función de utilidad:$U_v = M +qn$
				\item Restricción presupuestaria: $y_v = M + pn$
				\item $n=1$ mantener el auto, $n=0$ vender del auto
			\end{itemize}
		Por tanto:
			\begin{itemize}
				\item Para el vendedor la $U_{MG} = q$
				\item Para el comprador la $U_{MG} = 1.5q$
			\end{itemize}
		Esto es lo que garantiza potenciales ganancias en el intercambio.
	\end{enumerate}
\end{frame}
%------------------------------------------------
\begin{frame}{El artículo de Akerlof}
	El vendedor conoce $q$
		\begin{itemize}
			\item Sustituyendo $M$ de la $r.p.$ donde $M=y_v - pn$
			\item $U_v = y_v - pn +qn \longrightarrow $factor común a ``$n$''
			\item $U_v = y_v + (q-p)n$
		\end{itemize}
	Decisión
		\begin{itemize}
			\item $n = 0$ vende si $p\geq p$ (Precios $\geq$ Calidad)
			\item $n = $ vende si $p\leq p$ (Precios $\leq$ Calidad)
		\end{itemize}
	¿Cómo se forma $\mu$
\end{frame}
%------------------------------------------------
\begin{frame}{El artículo de Akerlof}
	Akerlof partió del supuesto simplificador que la calidad de los autos se distribuye de manera uniforme en el rango $[0,2]$, por tanto $f(q)= 1/2$\\[0.3cm]
	
	La oferta de autos es $S=(1/2)pN$ porque:
		\begin{itemize}
			\item la probabilidad de vender es $(1/2)p$
			\item Número de autos usados $()N$
		\end{itemize}
	Es decir, la oferta es el número de autos usados por la probabilidad de venta. Tenemos todos los datos necesarios para calcular el equilibrio de esta economía
		\begin{itemize}
			\item Demanda: $(3/2) \mu \geq p$
			\item Oferta:$(1/2)pN$
			\item Calidad media: $(1/2)p$
		\end{itemize}
	Con lo que no existe ningún precio que cumpla las condiciones
		\begin{itemize}
			\item Demanda: $(3/2) \mu \geq p \rightarrow \mu=(1/2)p \rightarrow (3/2)(1/2)p\geq p \rightarrow (3/4)p\geq p \rightarrow $ \textbf{CONTRADICCIÓN}
		\end{itemize}
\end{frame}
%------------------------------------------------
\begin{frame}{El artículo de Akerlof}
Existe el colapso total del mercado y la única diferencia de este modelo con uno convencional de competencia perfecta es la asimetría de la información. Este modelo es ineficiente en el sentido de Pareto debido a la estructura de la información.\\[0.3cm]
El colapso total del mercado es consecuencia en parte de la parametrizacion del modelo. Es posible determinar una valoración marginal del comprador para que se produzca el intercambio.
\end{frame}
%------------------------------------------------
\begin{frame}{El artículo de Akerlof}
	\begin{itemize}
		\item Utilidad del comprador $\longrightarrow U_c = M+\beta qn$
		\item Regla de compra $\longrightarrow \beta \mu \geq p$
		\item La calidad de los autos se distribuye uniformemente $\mu = (1/2)p$
		\item Sustituimos $\mu$ en la regla de compra $\longrightarrow \beta(1/2)p \geq p$
		\item Por tanto $\beta \geq 2$ y lograría un precio bajo y como consecuencia el intercambio.
		\item Si la valoración del auto es alta, el comprador se arriesgará a pesar de la alta presencia de limones.
	\end{itemize}
\end{frame}
%------------------------------------------------
\begin{frame}{El artículo de Akerlof}
	El intercambio dejaría de existir produciendo el colapso del mercado a pesar de habría vendedores dispuesto a intercambiar el auto al precio ``$p$'' y compradores a adquirirlo.\\[0.3cm]
	Pero la información asimétrica impide que se realicen intercambios beneficiosos para ambas partes.
\end{frame}
%------------------------------------------------
\begin{frame}{Ejemplo}
	Seguros:
		\begin{itemize}
			\item Las personas mayores de 65 años casi no pueden comprar un seguro médico, incluso si están dispuestos a pagar un alto precio. Las compañías de seguros saben que con un precio alto, solo aquellos que tienen más probabilidades de aprovechar el seguro comprarán la póliza. Por lo tanto, las pólizas rara vez se venden en este mercado en particular.
		\end{itemize}
\end{frame}
%------------------------------------------------
\begin{frame}{Ejemplo}
	Costo de deshonestidad
		\begin{itemize}
			\item La presencia de personas que venden productos inferiores tiende a expulsar el negocio legítimo. No solo los consumidores son engañados, sino que también aumentan las preocupaciones morales y legales.
			
			\item La experiencia para decir el verdadero valor de los bienes no distinguibles se dirige fácilmente al arbitraje en lugar del propósito de producción real porque el primero es más rentable en un mundo lleno de limones.
		\end{itemize}
\end{frame}
%------------------------------------------------
\begin{frame}{Ejemplo}
	Mercado crediticio en países en desarrollo
		\begin{itemize}
			\item Los empresarios tienen que recurrir a ``agencias gestoras'', personas y empresas con reputación e influencia comunitaria, para financiar a una empresa recién creada.
			\item El mercado de crédito rural está dominado por préstamos con tasas exorbitantes de prestamistas locales en lugar de aquellos con tasas oficiales de bancos formales, ya que solo los primeros tienen buen acceso a la información del prestatario. Cualquiera que intente arbitraje tiende a perder.
		\end{itemize}
\end{frame}