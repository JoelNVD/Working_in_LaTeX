%====================================================================================
\section{Introducción}
%====================================================================================

\begin{frame}{Introducción}
Una externalidad es una relación entre agentes económicos que está fuera del sistema de precios.
	\begin{itemize}
		\item Contaminación de una fábrica.
		\item Ruidos molestos de una casa vecina.
	\end{itemize}
Las externalidades no están bajo el control del agente afectado
	\begin{itemize}
		\item No se aplican algunos supuestos del sistema competitivo.
		\item El equilibrio competitivo no es Pareto eficiente
				\begin{itemize}
					\item Gran cantidad de externalidades ``malas''
					\item Pequeña cantidad de externalidades ``buenas''
				\end{itemize}
	\end{itemize}
Las externalidades tienen una importancia práctica
	\begin{itemize}
		\item Calentamiento global
		\item Daño a la capa de ozono
	\end{itemize}
\end{frame}
%------------------------------------------------
\begin{frame}{Definición de externalidades}
Una externalidad está presente siempre que el bienestar de algún agente económico es “directamente” afectado por la acción de otro agente en la economía.\\
	\begin{center}
		\begin{tikzpicture}[transform canvas={scale=0.6}]
	% Formato de CAJA
		\draw[->] (0.5,0.5) node[align=center, below left] {\footnotesize $O_A$} -- (0.5,4.5) node[align=center, above] {\footnotesize $K^{A}$};
		\draw[->] (0.5,0.5) -- (8.5,0.5) node[align=center, right] {\footnotesize $L^{A}$};
	
		% Curvas de indiferencia1
			% Agente B
				\draw  [blue] (0.6,4) ..controls (1.4,1.4) and (1.74,1) .. (6,0.6);
				\draw  [blue] (1,4.3) ..controls (1.7,2.1) and (2,1.6) .. (6.3,1);
				\draw  [blue] (1.6,4.6) ..controls (2.1,3.2) and (1.74,2.2) .. (6.7,1.4);
			
			% Flechas
				\node[draw, single arrow,
						minimum height=30mm, minimum width=1mm,
						single arrow head extend=1.5mm,
						anchor=west, blue, fill=blue, scale=0.5, rotate=50] at (2.7,2.7) {};
	
	% Formato de CAJA rotado
		\draw[->] (18,4) node[align=center, above right] {\footnotesize $O_B$} -- (10,4) node[align=center, left] {\footnotesize $L^{B}$};
		\draw[->] (18,4) -- (18,0) node[align=center, below] {\footnotesize $K^{B}$};
	
		% Curvas de indiferencia1
			% Agente B
				\draw [red] (12,3.9) .. controls (16.76,3.5) and (17.1,3.1) .. (17.9,0.5);
				\draw [red] (11.7,3.5) .. controls (16.5,2.9) and (16.8,2.4) .. (17.3,0.2);
				\draw [red] (11.3,3.1) .. controls (16.76,2.3) and (16.4,1.3) .. (16.6,-0.1);
			
			% Flechas
				\node[draw, single arrow,
						minimum height=30mm, minimum width=1mm,
						single arrow head extend=1.5mm,
						anchor=west, red, fill=red, scale=0.5, rotate=-130] at (15.7,1.8) {};
\end{tikzpicture}
	\end{center}
\end{frame}
%------------------------------------------------
\begin{frame}{Definición de externalidades}
Externalidad pecuniaria: un efecto externo que trabaja mediante los precios
	\begin{itemize}
		\item Un aumento en el precio del pollo afecta la rentabilidad de un restaurante
	\end{itemize}
Las externalidades pecuniarias no crean una ineficiencia. No son ``fallas de mercado''\\

Con externalidades no pecuniarias (o técnicas) las acciones de los agentes no son independientes y no son determinadas sólo por los precios.
	\begin{itemize}
		\item Surge interdependencia estratégica.
		\item Es una fuente de ineficiencia
	\end{itemize}
\end{frame}
%------------------------------------------------