%====================================================================================
\section{Introducción}
%====================================================================================

\begin{frame}{Introducción}
	\begin{center}
		\begin{tikzpicture}[transform canvas={scale=0.6}]
	% Formato de CAJA
		\draw[->] (0.5,0.5) node[align=center, below left] {\footnotesize $O_A$} -- (0.5,4.5) node[align=center, above] {\footnotesize $K^{A}$};
		\draw[->] (0.5,0.5) -- (8.5,0.5) node[align=center, right] {\footnotesize $L^{A}$};
	
		% Curvas de indiferencia1
			% Agente B
				\draw  [blue] (0.6,4) ..controls (1.4,1.4) and (1.74,1) .. (6,0.6);
				\draw  [blue] (1,4.3) ..controls (1.7,2.1) and (2,1.6) .. (6.3,1);
				\draw  [blue] (1.6,4.6) ..controls (2.1,3.2) and (1.74,2.2) .. (6.7,1.4);
			
			% Flechas
				\node[draw, single arrow,
						minimum height=30mm, minimum width=1mm,
						single arrow head extend=1.5mm,
						anchor=west, blue, fill=blue, scale=0.5, rotate=50] at (2.7,2.7) {};
	
	% Formato de CAJA rotado
		\draw[->] (18,4) node[align=center, above right] {\footnotesize $O_B$} -- (10,4) node[align=center, left] {\footnotesize $L^{B}$};
		\draw[->] (18,4) -- (18,0) node[align=center, below] {\footnotesize $K^{B}$};
	
		% Curvas de indiferencia1
			% Agente B
				\draw [red] (12,3.9) .. controls (16.76,3.5) and (17.1,3.1) .. (17.9,0.5);
				\draw [red] (11.7,3.5) .. controls (16.5,2.9) and (16.8,2.4) .. (17.3,0.2);
				\draw [red] (11.3,3.1) .. controls (16.76,2.3) and (16.4,1.3) .. (16.6,-0.1);
			
			% Flechas
				\node[draw, single arrow,
						minimum height=30mm, minimum width=1mm,
						single arrow head extend=1.5mm,
						anchor=west, red, fill=red, scale=0.5, rotate=-130] at (15.7,1.8) {};
\end{tikzpicture}
	\end{center}
\end{frame}
%------------------------------------------------
\begin{frame}{Introducción}
	\begin{itemize}
		\item Bien privado
			\begin{itemize}
				\item Rivalidad en el consumo = el consumo por parte de un agente económico, reduce las posibilidades de consumo de otros agentes (usualmente a nada).
				\item Posibilidad de exclusión = es necesario pagar para consumir el bien.
			\end{itemize}
		\item Bien público
			\begin{itemize}
				\item No presentar Rivalidad en el consumo = cada consumidor puede consumir toda la cantidad disponible del bien.
				\item Ser  no excluibles =  todos los consumidores pueden consumir el bien.
			\end{itemize}
	\end{itemize}
\end{frame}
%------------------------------------------------
\begin{frame}{Introducción}
bienes públicos puros son no rivales y no  excluibles
	\begin{itemize}
		\item La no rivalidad implica que el costo marginal de proveer el bien es nulo.
		\item Si es así, no es necesario racionar el consumo del bien: Si se excluye a alguien de consumir un bien no rival, sólo se perjudica a esa persona, pero no se beneficia a otros consumidores.
		\item En consecuencia, la exclusión es ineficiente
	\end{itemize}

Algunos bienes públicos, siendo no rivales, están sujetos a congestión.
	\begin{itemize}
		\item Antes de la congestión, el CMg es nulo.
		\item Al producirse la congestión, se producen externalidades negativas.
		\item Por tanto, aplicar la exclusión (cobrar por el uso) previene la congestión.
	\end{itemize}
\end{frame}
%------------------------------------------------
\begin{frame}{Introducción}
	\begin{alertblock}{Entonces}
		\begin{itemize}
			\item Todos los bienes públicos son no rivales.
			\item Algunos bienes públicos son no excluibles.
		\end{itemize}
	\end{alertblock}
\end{frame}