%====================================================================================
\section[Relevancia]{Aspectos relevantes en el análisis de economías con Bienes Públicos}
%====================================================================================

\begin{frame}{Aspectos relevantes en el análisis de economías con Bienes Públicos}
	\begin{enumerate}
		\item Los bs. Públicos pueden ser suministrados por empresas públicas (defensa nacional) o privadas (programas de radio). 
			\begin{itemize}
				\item Entonces: La calificación del bien tiene que ver con su naturaleza, no con el tipo de empresa que lo suministra.
			\end{itemize}
		\item La cantidad consumida del bien público se asocia a la cantidad disponible, no a la efectivamente consumida.
			\begin{itemize}
				\item Por tanto: los bs. Públicos son consumidos en cantidades iguales por todos los agentes.
			\end{itemize}
	\end{enumerate}
\end{frame}
%------------------------------------------------
\begin{frame}{Aspectos relevantes en el análisis de economías con Bienes Públicos}
	\begin{enumerate}[3]
		\item Los bs. Públicos descritos usualmente son bienes públicos puros. Otro tipo son los que están sujetos a congestión (por ejemplo: una obra de teatro, carreteras, parques públicos, playas, etc.
		\item Aunque algunos bs. Públicos son disfrutados por todos por igual (sin aplicar “quien no paga, no consume”), en otros puede aplicarse la exclusión (peajes, entradas a espectáculos, cuotas de un club, etc.).
		\item Algunos ``bienes'' tienen características ``públicas'', pero como son nocivos constituyen males públicos, como:
			\begin{itemize}
				\item Lluvia ácida
				\item Destrucción de bosques amazónicos
				\item Agujero de la capa de ozono
			\end{itemize}
	\end{enumerate}
\end{frame}
%------------------------------------------------
\begin{frame}{En suma:}
	\begin{itemize}
		\item Menos problemático para economías competitivas: bs. Públicos más parecidos a los privados (hay posibilidad de excluir a los que no pagan).
		\item Más problemático: bs. Públicos puros (como los consumidores no están dispuestos a pagar y no se les puede excluir, el mecanismo de precios  deja de actuar como asignador de recursos).
	\end{itemize}
\end{frame}