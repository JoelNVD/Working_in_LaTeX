%====================================================================================
% Preamble
%------------------------------------------------------------------------------------
\documentclass[10pt,a4paper]{article}

% Apartado de texto
\usepackage[utf8]{inputenc}
\usepackage[spanish]{babel}
\usepackage[T1]{fontenc}

% Apartado matemático
\usepackage{amsmath}
\usepackage{amsfonts}
\usepackage{amssymb}
\usepackage{mathtools}
\usepackage{mathrsfs}
\usepackage[libertine]{newtxmath}

% Apartadode no sangría
\usepackage{parskip}

% Aparatod posición del texto
\usepackage[left=2cm,right=2cm,top=2cm,bottom=2cm]{geometry}
\usepackage{fancyhdr}

\usepackage{graphicx}
\usepackage{xcolor}
\definecolor{cardinal}{rgb}{0.77, 0.12, 0.23}

% Apartado de columnas y filas: juntarlas o separarlas
\usepackage{multicol, multirow}

% Apartado differents label en listado
\usepackage[shortlabels]{enumitem}

% Apartado hyperres
\usepackage[hidelinks]{hyperref}

\usepackage{tcolorbox}
\newtcolorbox{nota}[1]{
	colback=cardinal!5!white,
	colframe=cardinal!75!black,
	fonttitle=\bfseries,
	title=#1
}

% Apartado dibujos
\usepackage{tikz, pgfplots}
\pgfplotsset{compat=1.18}
\usetikzlibrary{positioning,calc,arrows}
\usetikzlibrary{shapes.arrows}
\usetikzlibrary{patterns}
\usetikzlibrary{babel} % Para que recono > y <, en inglés no se pone, pero en spañol sí
\usetikzlibrary{shadings,shadows}
\usetikzlibrary{matrix}
\usetikzlibrary{intersections}

% Apartado cambio como por punto
\decimalpoint

% \highlight[<colour>]{<stuff>}
\newcommand{\highlight}[2][yellow]{\mathchoice%
	{\colorbox{#1}{$\displaystyle#2$}}%
	{\colorbox{#1}{$\textstyle#2$}}%
	{\colorbox{#1}{$\scriptstyle#2$}}%
	{\colorbox{#1}{$\scriptscriptstyle#2$}}}%

% Definir oprerador
\DeclareMathOperator*{\M}{Max}

%====================================================================================

%====================================================================================
% Body
%====================================================================================
% Title Page
%-----------
\textwidth=450pt \textheight=620pt \oddsidemargin=0in
\topmargin=-10pt
\pagestyle{fancy} \rhead{\scriptsize{\textbf{Código del Curso:} xxXXXXX} \\
	\textbf{Fecha:} XX/XX/2021 \& 2021-I \hspace{0.04cm}}
\lhead{\scriptsize{\textbf{Profesor:} José A. Valderrama}  \\
	\textbf{Curso:} Teoría Microeconómica II}
\newcommand{\re}[1]{\smallskip\textsf{\textbf{Respuesta}} \begin{sf}\\ #1 \end{sf} \bigskip}
\newcommand{\ay}[1]{ \scriptsize{\textsl{Hint: #1}}\normalsize{}}
\newcommand{\pr}[2]{\frac{\partial #1}{\partial #2}}

%------------------------------------------------------------------------------------
% Title
%---------
\begin{document}
	\begin{center}
		{\Large {\textbf{Fallas de mercado}}}		
	\end{center}
% EJERCICIOS---------------------------------------------------------------------------------
\textbf{\LARGE Externalidades}
	\begin{enumerate}
		\item Una industria refinadora competitiva arroja una unidad de desperdicio en la atmósfera por cada unidad de producto refinado. La función inversa de demanda del producto refinado es $p=30-q$. La curva inversa de oferta para el refinamiento es $CM=3+q$. La curva de costo externo marginal es $CEM=1.5q$,  donde CEM es el costo externo marginal cuando la industria arroja q unidades de desperdicio. ¿Cuál es el precio y cantidad de equilibrio para el producto refinado?\\
		
			\textbf{\large Solución}\\
				\vspace{-0.7cm}
\begin{multicols}{2}
	\begin{tikzpicture}
		\begin{axis}[scale=0.9,
					 xmin=0, xmax=20,
					 ymin=0, ymax=12,
					 title style={at={(0.5,0.2)},anchor=north,yshift=5cm},
					 title = Consumidor $A$,
					 axis lines = left,
					 xtick={2,4,6,8,10,12,14,16,18},
					 ytick={2,4,6,8,10},
					 grid=both,
					 grid style={line width=.1pt, draw=gray!20},
					 clip = false,
					]
			
			\node [above]  at (current axis.above origin) {$x_{2}^{A}$};
			\node [right]  at (current axis.right of origin) {$x_{1}^{A}$};
			
			\draw[fill=black] (60,80) circle (1.5) node[above right] {$w^{A}$};
		\end{axis}
	\end{tikzpicture}

	\begin{tikzpicture}
		\begin{axis}[scale=0.9,
					 xmin=0, xmax=20,
					 ymin=0, ymax=12,
					 title style={at={(0.5,0.2)},anchor=north,yshift=5cm},
					 title = Consumidor $B$,
					 axis lines = left,
					 xtick={2,4,6,8,10,12,14,16,18},
					 ytick={2,4,6,8,10},
					 grid=both,
					 grid style={line width=.1pt, draw=gray!20},
					 clip = false,
					]
			
			\node [above]  at (current axis.above origin) {$x_{2}^{B}$};
			\node [right]  at (current axis.right of origin) {$x_{1}^{B}$};
			
			\draw[fill=black] (120,40) circle (1.5) node[above right] {$w^{B}$};
		\end{axis}
	\end{tikzpicture}
\end{multicols}

Otras posibilidades pasan por redistribuir unidades entre ambos consumidores. Por ejemplo, si $A$ quiere consumir más de 6 unidades de $x_1$ habrá que quitarle a $B$.\\

Podrían proponerse otras posibles repartos, si representamos una asiganción como $\left(x_{1}^{A}, x_{2}^{A}, x_{1}^{B}, x_{2}^{B}\right)$ y posibles asignaciones como $\left( 4,8,14,4\right)$, $\left(6,6,12,6 \right)$ o $\left( 10,5,8,7\right)$; sin embargo la asignación $\left( 9,5,11,6\right)$ no es una asignación válida por lo siguiente:

$$\begin{array}{ccc}
	x_{1}^{A} + x_{2}^{A} = w_{1}^{A} + w_{2}^{A} & {} & x_{1}^{B} + x_{2}^{B} = w_{1}^{B} + w_{2}^{B} \\
	 4 + 14    = 6 + 12  & {} & 8 + 4    = 8 + 4 \\
	 6 + 12    = 6 + 12  & {} & 6 + 6    = 8 + 4 \\
	 9 + 11 \neq 6 + 12  & {} & 5 + 6 \neq 8 + 4 
\end{array}$$

		
		\item De la pregunta anterior, ¿cuánto debe ser la oferta de mercado en el óptimo social?\\
		
			\textbf{\large Solución}\\
				La pregunta anterior se resolvió de la forma general, pero se puede resolver de forma práctica igualando \emph{TMS}; si y solo si, las funciones de utilidad son tipo \emph{Cobb-Douglas} o transformaciones monotónicas de esta. Veamos.

	\begin{enumerate}[a)]
		\item \colorbox{yellow}{$TMS_A = TMS_B$}
				Teniendo en cuenta:
					$$TMS = \frac{UMg_x}{UMg_y} = \frac{\partial U/\partial x}{\partial U/\partial y}$$
				entonces,
					$$\frac{y_A}{x_A} = \frac{y_B}{x_B} \rightarrow \frac{y_A}{x_A} = \frac{3-y_A}{2-x_A} \rightarrow y_A = \frac{3x_A}{2} \text{ y } y_B = \frac{3x_B}{2}$$
		\item $$
				\begin{array}{c|c}
					U_A = x_A\left( \frac{3x_A}{2}\right)  & U_B = x_B\left( \frac{3x_B}{2}\right) \\[0.3cm]
					x_A = \frac{\sqrt{6}}{3}U_A^{1/2} & x_B = \frac{\sqrt{6}}{3}U_B^{1/2}
				\end{array}
			  $$
			  	$x_A + x_B = 2 \longrightarrow U_A^{1/2}+U_B^{1/2} = \sqrt{6} \longrightarrow \therefore U_B = \left( \sqrt{6} - U_A^{1/2}\right)^{2} $
			  	\begin{center}
			  		\begin{tikzpicture}[scale = 0.7, samples = 100]
			  			\draw[purple, domain = 0:6] plot({\x},{(sqrt(6)-(\x)^(0.5))^(2)});
			  			\draw[<->] (0,7) node [above] {$U_B$} -- (0,0) -- (7,0) node [right] {$U_A$};
			  			\draw[black, fill=black] (0,6) circle[radius=0.07] node [left] {6};
			  			\draw[black, fill=black] (6,0) circle[radius=0.07] node [below] {6};
			  		\end{tikzpicture}
			  	\end{center}
			  	
	\end{enumerate}
		
		\item Suponga que donde usted vive hay una empresa contaminante cuyo producto es vendido a un precio de 30 soles. Asumimos que la función de costo marginal es: $CMG = 5 + 2Q$, por último el costo marginal externo es: $CME = 0.5Q$. Se pide la cantidad que debe producir la empresa para alcanzar la externalidad óptima.\\
		
			\textbf{\large Solución}\\
				\begin{enumerate}[a)]
	\item Se requiere que las asignaciones factoriales sean factibles y no derrochadoras
			\begin{align*}
				RMT^A & = RMS^B\\
				\frac{PMgL^{A_x}}{PMgK^{A_x}} =\frac{K_{A_x}}{L_{A_x}} & = \frac{K_{B_y}}{2L_{B_y}} =\frac{PMgL^{B_y}}{PMgK^{B_y}}
			\end{align*}
		  Reemplazando $L_{A_x} + L_{B_y} = \overline{L}$ y $K_{A_x} + K_{B_y} = \overline{K}$, para hallar la curva de contrato
			$$\therefore \frac{K_{A_x}}{L_{A_x}} = \frac{\overline{K} - K_{A_x}}{2\overline{L} - L_{A_x}}$$
	\item Una asignación igualitaria de los factores entre las dos empresas significa que:
			\begin{gather*}
				L_{A_x} = \frac{\overline{L}}{2};\enskip K_{A_x}=\frac{\overline{K}}{2}\\
				L_{B_y} = \frac{\overline{L}}{2};\enskip K_{B_y}=\frac{\overline{K}}{2}
			\end{gather*}
		  Resulta evidente que con esta asignación de factores:
		  	$$\frac{K_{A_x}}{L_{A_x}} = \frac{K_{B_y}}{L_{B_y}}$$
		  por lo que no se verifica la condición $RMT^A = RMS^B$ de eficiencia
\end{enumerate}
	\end{enumerate}

\textbf{\LARGE Bienes públicos}
	\begin{enumerate}
		\item [4.] Considere  dos consumidores con las siguientes funciones de demanda por un bien público: $P_{1}=10 - \frac{G}{10}$, $P_{2}= 20 - \frac{G}{10}$, donde $P_{i}$ es el precio que i está deseando pagar por la cantidad $G$. ¿Cuál es el nivel óptimo de consumo del bien público, si su costo marginal es igual a 25?\\
		
			\textbf{\large Solución}\\
				La curva de contrato está dada por la siguiente relación:
	\begin{eqnarray*}
		RMS_{A} & = & RMS_{B}\\[0.3cm]
		\frac{x_{2}^{A}}{x_{1}^{A}} & = & \frac{x_{2}^{B}}{x_{1}^{B}}\\[0.3cm]
		\frac{x_{2}^{A}}{x_{1}^{A}} & = & \frac{12 - x_{2}^{A}}{15 - x_{1}^{A}}\\[0.3cm]
		\therefore x_{2}^{A} & = & \frac{12}{15}x_{1}^{A}
	\end{eqnarray*}
La curva de contrato será una \textbf{LÍNEA RECTA}
		
		\item [5.] Sean dos estudiantes: Eva y Teresa, que comparten una habitación. Ambas tienen la misma función de utilidad respecto de los cuadros (bien $X$) y de las cervezas (bien $Y$).\\ 
		La función de utilidad de las 2 estudiantes es la siguiente:
			$$U_{i} = X^{\frac{1}{3}}_{i}Y^{\frac{2}{3}}_{i}$$
		Donde i representa a cada estudiante. Cada estudiante tiene un ingreso de 3000 para gastar. El precio de $X$ es 100 y el precio de $Y$ es 0.2. Se pide hallar el consumo de cuadros y cervezas, si cada persona vive sola.\\
			\textbf{\large Solución}\\
				Analizando al agente $A$, se debe cumplir la siguiente relación:
	\begin{itemize}
		\item $RMS_{A}\left(x_{1}^{A},\enskip x_{2}^{A} \right) = \frac{p_1}{p_2}\Longrightarrow \frac{x_{2}^{A}}{2x_{1}^{A}} = \frac{1}{1}$
		\item $p_{1}x_{1}^{A}+p_{2}x_{2}^{A} = p_{1}w_{1}^{A} + p_{2}w_{2}^{A} \Longrightarrow x_{1}^{A} + x_{2}^{A} = 16 + 4$
	\end{itemize}
Entonces 
	\begin{itemize}
		\item $x_{2}^{A}=2x_{1}^{A} \Longrightarrow  x_{1}^{A} + 2x_{1}^{A} = 3x_{1}^{A} = 16 + 4 = 20 \Longrightarrow x_{1}^{*A} = \frac{20}{3}$
		\item $x_{2}^{A}=2x_{1}^{A} = 2x_{1}^{*A} = \frac{20}{3} \Longrightarrow x_{2}^{*A} = \frac{40}{3}$
	\end{itemize}
La decisión optima de $A$:\\
	$$\therefore \left(x_{1}^{*A},\enskip x_{2}^{*A} \right) = \left(\frac{20}{3}, \enskip \frac{40}{3}\right)$$
Dada la dotación incial, para consumir lo anterior el condumiro $A$ tendrá que vender $\frac{28}{3}$ del bien 1 (una demanda neta de $\frac{20}{3} - 16$, negativa), y comprar $\frac{28}{3}$ del bien 2 (demanda neta $\frac{40}{3}-4$)\\

Analizando al agente $B$
			$$	\left.
					\begin{array}{l}
						\frac{2x_{2}^{B}}{x_{1}^{B}} = \frac{1}{1} \\ [.5cm]
						x_{1}^{B} + x_{2}^{B} = 4 + 10
					\end{array}
				\right\} \Longrightarrow \therefore \left(x_{1}^{*B},\enskip x_{2}^{*B} \right) = \left(\frac{28}{3}, \enskip \frac{14}{3}\right) $$
Comparando la deciión de $A$ y $B$. El consumidor $A$ necesita vender bien 1 y el $B$ necesita comprar, pero no en la misma cantidad. La demanda neta de bien 1 por parte de $B$ es de $\frac{14}{3}$, frente a los $\frac{28}{3}$ que $A$ quiere vender. Lo mismo, en sentido contrario ocurre en el bien 2.
		
		\item [6.] De la pregunta anterior, si ambas viven juntas. ¿Cuál es el nivel óptimo del bien público (bien $X$)?\\
			\textbf{\large Solución}\\
				La condición de óptimo:
		$$\sum RMS = \frac{P_{X}}{P_{Y}}$$
Teniendo en cuenta que:
	\begin{align*}
		Y_{1} + Y_{2} &= Y\\
		X_{1} = X_{2} &= X
	\end{align*}
Resolviendo:
	\begin{align*}
		\frac{Y_{1}}{2X_{1}} + \frac{Y_{2}}{2X_{2}} &= \frac{100}{0.2}\\
		\frac{Y}{2X} &= 500\\
		\frac{Y}{X} &= 1000
	\end{align*}
Hallando el nivel óptimo de X, usando la restricción monetaria conjunta:
	\begin{align*}
		P_{X}X + P_{Y}Y_{1} + P_{Y}Y_{2} &= M_{1} + M_{2}\\
		1000X + 0.2Y &= 2 \times 3000\\
		1000X + 0.2\left(1000X\right) &= 6000\\
		1200X &= 6000\\
		X^* &= 5
	\end{align*}
Por tanto el nivel óptimo del bien público es: $X^* = 5$.
	\end{enumerate}

\textbf{\LARGE Selección Adversa}
	\begin{enumerate}
		\item [7.] Supongamos que un empresario (neutral ante el riesgo) quiere contratar a un trabajador, pero no conoce todas las características de dicho trabajador. Lo que desconoce es la productividad que el esfuerzo del trabajador tiene en el proceso de producción para el que desea contratarle. Sabe, sin embargo, que el trabajador es neutral ante el riesgo y que puede ser de dos tipos: o bien su productividad es alta, con lo que su esfuerzo es igual a $e^{2}$, o bien es baja y su esfuerzo es igual a $e^{2}$
		Al primer tipo de trabajador le llamaremos  ``$B$'' y al segundo ``$M$'', ya que éste último sufre mayor desutilidad que el primero. La función de utilidad del trabajador es por tanto $U^{B}(w, e)=w-e^{2}$ o $U^{M}(w, e)=w-2e^{2}$. La probabilidad de que el trabajador sea de tipo $B$ es $q$ (y por tanto con probabilidad $(1-q)$ el trabajador es de tipo $M$). La utilidad de reserva de ambos tipos de trabajador s $\underline{U}=0 .$ El empresario, por su parte, valora el esfuerzo del trabajador en $\pi=ke$, donde $k$ es una constante suficientemente grande (de tal modo que el empresario está interesado en contratar al trabajador sea cual sea su tipo). Por cada unidad de esfuerzo del trabajador el empresario obtiene, or tanto, $k$ unidades de beneficio.
			\begin{enumerate}[a)]
				\item Formular el problema que resuelve el principal con información simétrica. Calcular los contratos óptimos. ¿Cuáles son los esfuerzos que pide y los salarios que paga?
				\item Formular el programa que debería resolver el principal en información asimétrica y resuelvalo. ¿Cuáles son los contratos que el principal ofrece a los agentes? Comparar los casos de información simétrica y asimétrica.
			\end{enumerate}
				\textbf{\large Solución}\\
					\begin{enumerate}[a)]
	\item Dadas las tecnología Leontief de las empresas, no es posible resolver el problema de forma analítica. Si los resolvemos aplicando directamente el concepto de eficiencia productiva ,ayudándonos para ello de las representaciones gráficas de esta economía, la cual se hace en la siguiente figura:
		\begin{center}
			\begin{tikzpicture}
				% Curva de contrato
					\draw[purple] (0,0) -- (6,4);
					\draw[purple] (4.5,0) -- (6,4);
					
				% Relleno
					\fill [pattern=crosshatch dots,pattern color=purple!20!white] (0,0) -- (6,4) -- (4.5,0) -- (0,0);
					
				% Caja
					\draw[<->] (0,4.5) node[align=center, above, scale=0.8] {$L_1$} -- (0,0) node[below left] {\footnotesize $O_1$} -- (6.5,0) node[align=center, right, scale=0.8] {$K_1$};
					\draw[<->] (-0.5,4) node[align=center, left, scale=0.8] {$K_2$} -- (6,4) node[align=center, above right, scale=0.8] {$O_2$} -- (6,-0.5) node[align=center, below, scale=0.8] {$L_2$};
					
				% Union de puntos
					\draw[dashed, black, fill=black] (4.5,0) circle[radius=0.05] node [below, scale=0.7] {$A$} -- (4.5,3) circle[radius=0.05] node [above, scale=0.7] {$B$};
					
				% Curvas de indiferencia
					% Agente A
						\draw[blue] (3,2.5) -- (3,2) -- (3.5,2);
						\draw[blue] (2,2.5) -- (2,1.33) -- (3.17,1.33); 
					% Agente B
						\draw[red] (4.75,2) -- (5.25,2) -- (5.25,1.5);
						\draw[red] (3.82,1.33) -- (5,1.33) -- (5,0.16);
						
				% Texto
					\draw[purple] (3,0.5) node[align=center, above, scale=0.8] {Conjunto Paretiano};
			\end{tikzpicture}
		\end{center}
	Es evidente que la asignaciones eficientes en la producción son la dadas por los puntos del conjunto cerrado $O_1AO_2$. Es decir,
		\begin{gather}
			\left\lbrace \left( q_1, K_1, L_1\right) \right\rbrace = \left\lbrace 0 \leq K_1 \leq 15 \text{ y } L_1 \leq 0.5K_1\right\rbrace \cup \left\lbrace 15 \leq K_1 \leq 20 \text{ y }  0.5K_1\geq L_1 \geq 2K_1 - 30\right\rbrace \label{eq1}
		\end{gather}
	y donde el subconjunto $\{0 \leq K_1 \leq 15$ y $L_1 \leq 0.5K_1 \}$ es el representado gráficamente por el triángulo $O_1AB$, en tanto que el sobconjunto $\{15 \leq K_1 \leq 20$ y $ 0.5K_1\geq L_1 \geq 2K_1 - 30 \}$ es el dado por el triangulo $ABO_2$.
	
	\item Para determinar el $CPP$, lo que hacemos es tomar algunos puntos del conjunto de contrato (de la producción) definido en (\ref{eq1}) y representarlo en el espacio de productos $\left\lbrace q_1,q_2\right\rbrace$. Por ejemplo, a lo largo de la recta $L_1 = 0.5 K_1$, la correspondencia es la siguiente:
		\begin{center}
			\begin{tabular}{ccc}
				\hline
					$\left( K_1, L_1\right)$ & {} & $\left( q_1, q_2\right)$ \\
				\hline
					(0,0) 	 & {} & (0,10)	\\
					(6,3) 	 & {} & (6,7)	\\
					(10,5)   & {} & (10,5)	\\
					(15,7.5) & {} & (15,2.5) \\
				\hline
			\end{tabular}
		\end{center}
	y es evidente que la FPP es\footnote{Tómense los punto (0,10) y (15,2.5) y recuérdese la ecuación de una recta que pasa por dos puntos.}
			$$q_2 = 10 - 0.5q_1, \text{ si } 0 \leq q_1 \leq 15$$
	Análogamente, si del $CCP$ definido en (\ref{eq1}) tomamos algunos puntos a lo largo de la recta $L_1 = 2K_1 - 30$ y los proyectamos en el espacio de productos $\left\lbrace q_1,q_2\right\rbrace $, se llega a
		\begin{center}
			\begin{tabular}{ccc}
				\hline
					$\left( K_2, L_2\right)$ & {} & $\left( q_1, q_2\right)$ \\
				\hline
					(15,0) 	 & {} & (0,10)	\\
					(18,6) 	 & {} & (12,4)	\\
					(19,8)   & {} & (16,2)	\\
					(20,10)  & {} & (20,0)  \\
				\hline
			\end{tabular}
		\end{center}
	pudiéndose inferir que
			$$q_2 = 10 - 0.5q_1, \text{ si } 0 \leq q_1 \leq 20$$
	En definitiva, la $FPP$ de esta economía es el conjunto de producciones
			$$\left\lbrace \left(q_1, q_2 \right) \in \mathbb{R}_{+}^{2} \mid q_2 \left( q_1\right) = 10 -0.5q_1, 0 \leq q_1 \leq 20 \right\rbrace $$
	\item A partir de la tecnología de la empresa 1, es evidente que las demandas de factores de esta empresa son las dadas por
			$$\left( K_1, L_1\right) = \left( \frac{2q_1}{2r + w}, \frac{q_1}{2r + w}\right) $$
	ya que los inputs $K$ y $L$ pueden ser interpretados como un ``input compuesto'' cuyp coste es $2r+w$. Análogamente, las demandas de inputs de la empresa 2 son
			$$\left( K_2, L_2\right) = \left( \frac{q_2}{r + 2w}, \frac{2q_2}{r + 2w}\right) $$
	Luego, el par de precios $(r, w)$ tales que
			$$\frac{2q_1}{2r + w}+  \frac{q_1}{2r + w} \leq 20$$
	y
			$$\frac{q_2}{r + 2w} + \frac{2q_2}{r + 2w} \leq 10$$
	define el equilibrio walrasiano de esta economía de inputs complementarios.
\end{enumerate}
		
		\item [8.] Una compañía ofrece el servicio de vuelo entre dos ciudades, siendo el costo por pasajero $C$. Hay dos tipos de clientes de esta compañía: los ejecutivos que se desplazan por razones de negocio y los turistas. La proporción de ejecutivos que demandan este destino es $\alpha$ y están dispuestos a pagar $P_A$ por el viaje. La proporción de turistas es $(1-\alpha)$ y están dispuestos a pagar $P_B$ por el mismo viaje. La utilidad que deriva cada ejecutivo del viaje es $U_A$ y la de cada tursitas $U_B$ con ($U_A > U_B$)Todos los pasajeros tienen una utilidad de reserva $U$. El problema al que se enfrenta la compañía es determinar la política de precios y plazas para los ejecutivos y los turistas, sin conocer a priori quienes son los unos y los otros. Escriba el problema de optimización del principal (sin resolverlo), con selección adversa.\\
		
			\textbf{\large Solución}\\
				\vspace{-0.3cm}
\begin{center}
	\begingroup
		\setlength{\tabcolsep}{10pt}
		\renewcommand{\arraystretch}{1.5} 
			\begin{tabular}{ccc}
					\hline
				\multicolumn{3}{c}{Datos} \\
					\hline
				2 consumidores: & {} & $A$ y $B$ \\
				2 bienes: & {} &$x$ e $y$ \\
				F.U: & {} &$u^{A} = \ln(x_A) + 2y_A$ y $u^{B} = \ln(x_B) + 2y_B$ \\
				Dotaciones: & {} &$w^A = (1, 4)$ y $w^B = (6, 3)$ \\
					\hline
			\end{tabular}
	\endgroup
\end{center}
La curva de contrato:
	$$RMS^A = RMS^B \Longrightarrow x_A = x_B$$
además
	$$x_A + x_B = \overline{x} \Longrightarrow x_A + x_A = \overline{x} \Longrightarrow x_A = \frac{\overline{x}}{2} = x_B$$
	\begin{center}
		\begin{tikzpicture}[scale=0.8]
				% Formato de CAJA
					\draw[->] (0,0) node[align=center, below left] {\footnotesize $O_A$} -- (0,8) node[align=center, above] {\footnotesize $x_{2}^{A}$};
				\draw[->] (0,0) -- (8,0) node[align=center, right] {\footnotesize $x_{1}^{A}$};
				
					\draw[->] (7,7) node[align=center, above right] {\footnotesize $O_B$} -- (-1,7) node[align=center, left] {\footnotesize $x_{2}^{B}$};
				\draw[->] (7,7) -- (7,-1) node[align=center, below] {\footnotesize $x_{2}^{B}$};
						
			% Curvas de indiferencia
				\draw [blue] (1.4,6.9) .. controls (2.84,5.7) and (3.8,5.2) .. (5.6,4.8);
				\draw [red] (1.4,6.15) .. controls (3.03,5.8) and (3.99,5.38) .. (5.6,4.05);
				
				\draw [blue] (1.4,5) .. controls (2.84,3.8) and (3.8,3.3) .. (5.6,2.9);
				\draw [red] (1.4,4.25) .. controls (3.03,3.9) and (3.99,3.48) .. (5.6,2.15);
				
				\draw [blue] (1.4,3.1) .. controls (2.84,1.9) and (3.8,1.4) .. (5.6,1);
				\draw [red] (1.4,2.35) .. controls (3.03,2) and (3.99,1.58) .. (5.6,0.25);
			% Curvas de contrato
				\draw [purple, very thick] (3.5,0) node [below, scale= 0.6mm] {$\frac{\overline{x}}{2}$} -- (3.5,7);
				
			% Flechas
				\node[draw, single arrow,
					minimum height=28mm, minimum width=1mm,
					single arrow head extend=1.5mm,
					anchor=west, blue, fill=blue, scale=0.5, rotate=33] at (1.5,5.1) {};
				\node[draw, single arrow,
					minimum height=28mm, minimum width=1mm,
					single arrow head extend=1.5mm,
					anchor=west, red, fill=red, rotate=-147, scale=0.5] at (5.5,2.1) {};
		\end{tikzpicture}
	\end{center}

Probamos si las dotaciones iniciales son ESP.
	$$x_{A} = x_{B} = \frac{\overline{x}}{2} \Longrightarrow	1 \neq 6 \neq \frac{7}{2}$$
Se comprueba que la dotación inicial
	\begin{center}
		\begin{tikzpicture}[scale=0.8]
			% Formato de CAJA
				\draw[->] (0,0) node[align=center, below left] {\footnotesize $O_A$} -- (0,8) node[align=center, above] {\footnotesize $x_{2}^{A}$};
				\draw[->] (0,0) -- (8,0) node[align=center, right] {\footnotesize $x_{1}^{A}$};
			
				\draw[->] (7,7) node[align=center, above right] {\footnotesize $O_B$} -- (-1,7) node[align=center, left] {\footnotesize $x_{2}^{B}$};
				\draw[->] (7,7) -- (7,-1) node[align=center, below] {\footnotesize $x_{2}^{B}$};
			
			% Curvas de indiferencia
				\draw [blue] (1.4,6.9) .. controls (2.84,5.7) and (3.8,5.2) .. (5.6,4.8);
				\draw [red] (1.4,6.15) .. controls (3.03,5.8) and (3.99,5.38) .. (5.6,4.05);
				
				\draw [blue] (1.4,5) .. controls (2.84,3.8) and (3.8,3.3) .. (5.6,2.9);
				\draw [red] (1.4,4.25) .. controls (3.03,3.9) and (3.99,3.48) .. (5.6,2.15);
				
				\draw [blue] (1.4,3.1) .. controls (2.84,1.9) and (3.8,1.4) .. (5.6,1);
				\draw [red] (1.4,2.35) .. controls (3.03,2) and (3.99,1.58) .. (5.6,0.25);
				
			% Curvas de contrato
				\draw [purple, very thick] (3.5,0) node [below, scale= 0.6mm] {$\frac{\overline{x}}{2}$} -- (3.5,7);
			
			% Flechas
				\node[draw, single arrow,
						minimum height=28mm, minimum width=1mm,
						single arrow head extend=1.5mm,
						anchor=west, blue, fill=blue, scale=0.5, rotate=33] at (1.5,5.1) {};
				\node[draw, single arrow,
						minimum height=28mm, minimum width=1mm,
						single arrow head extend=1.5mm,
						anchor=west, red, fill=red, rotate=-147, scale=0.5] at (5.5,2.1) {};
			
			% Intersección
				\draw[dashed] (1,7) node[above] {\footnotesize $x_{1}^{B}=6$} -- (1,0) node[below] {\footnotesize $x_{1}^{A}=1$};
				\draw[dashed] (0,4) node[left] {\footnotesize $x_{2}^{A}=4$} -- (7,4)node[right] {\footnotesize $x_{2}^{B}=3$};
			
			% Punto
				\draw[black, fill=black] (1,4) circle[radius=0.08] node[align=center, above left] {\footnotesize $w$};
		\end{tikzpicture}
	\end{center}

Con la inclusión de precios, la situación es la siguiente:
	$$RMS^A = RMS^B = \frac{p_x}{p_y}$$
	$$RMS^A=\frac{1}{2x_A}=\frac{p_x}{p_y} \quad , \quad RMS^B=\frac{1}{2x_B}=\frac{p_x}{p_y} \Longrightarrow x_A = x_B = \frac{p_y}{2p_x}$$
	
	$$
		\begin{array}{ccc}
			p_xx_A + p_yy_A = p_x + 4p_y & {} &p_xx_B + p_yy_B = 6p_x + 3p_y\\[0.4cm]
			p_x\frac{p_y}{2p_x} + p_yy_A = p_x + 4p_y & {} &p_x\frac{p_y}{2p_x} + p_yy_B = 6p_x + 3p_y\\[0.4cm]
			\frac{p_y}{2} + p_yy_A = p_x + 4p_y & {} & \frac{p_y}{2} + p_yy_B = 6p_x + 3p_y \\[0.4cm]
			y_{A}^* = \frac{2p_x + 7p_y}{2p_y} & {} & y_{B}^* = \frac{12p_x + 5p_y}{2p_y}
		\end{array}
	$$
	
	$$y_{A}^* + y_{B}^* = 4 + 3 = 7 \Longrightarrow \frac{2p_x + 7p_y}{2p_y} + \frac{12p_x + 5p_y}{2p_y} = 7  \Longrightarrow  \frac{p_x}{p_y} = \frac{1}{7}$$
	
	\begin{gather*}
		\therefore y_{A}^* = \frac{51}{14}  \Longrightarrow x_{B}^* = \frac{7}{2}\\[0.4cm]
		\therefore y_{b}^* = \frac{47}{14}  \Longrightarrow x_{B}^* = \frac{7}{2}
	\end{gather*}

	\begin{center}
		\begin{tikzpicture}[scale=0.8]
			% Formato de CAJA
				\draw[->] (0,0) node[align=center, below left] {\footnotesize $O_A$} -- (0,8) node[align=center, above] {\footnotesize $x_{2}^{A}$};
				\draw[->] (0,0) -- (8,0) node[align=center, right] {\footnotesize $x_{1}^{A}$};
				
				\draw[->] (7,7) node[align=center, above right] {\footnotesize $O_B$} -- (-1,7) node[align=center, left] {\footnotesize $x_{2}^{B}$};
				\draw[->] (7,7) -- (7,-1) node[align=center, below] {\footnotesize $x_{2}^{B}$};
			
			% Curvas de indiferencia
				\draw [blue] (1.4,6.9) .. controls (2.84,5.7) and (3.8,5.2) .. (5.6,4.8);
				\draw [red] (1.4,6.15) .. controls (3.03,5.8) and (3.99,5.38) .. (5.6,4.05);
				
				\draw [blue] (1.4,5) .. controls (2.84,3.8) and (3.8,3.3) .. (5.6,2.9);
				\draw [red] (1.4,4.25) .. controls (3.03,3.9) and (3.99,3.48) .. (5.6,2.15);
				
				\draw [blue] (1.4,3.1) .. controls (2.84,1.9) and (3.8,1.4) .. (5.6,1);
				\draw [red] (1.4,2.35) .. controls (3.03,2) and (3.99,1.58) .. (5.6,0.25);
			
			% Curvas de contrato
				\draw [purple, very thick] (3.5,0) -- (3.5,7);
			
			% Intersección
				\draw[dashed] (1,7) node[above] {\footnotesize $x_{1}^{B}=6$} -- (1,0) node[below] {\footnotesize $x_{1}^{A}=1$};
				\draw[dashed] (0,4) node[left] {\footnotesize $x_{2}^{A}=4$} -- (7,4)node[right] {\footnotesize $x_{2}^{B}=3$};
			
			% Punto
				\draw[black, fill=black] (1,4) circle[radius=0.08] node[align=center, above left] {\footnotesize $w$};
				\draw[black, fill=black] (3.5,3.58) circle[radius=0.08] node[align=center, below left] {\footnotesize $EGW$};
			
			% Recta presupuestaria
				\draw (0,5.26) -- (7,1.9) node[right] {\footnotesize $RP$};
			
			% Intersección
				\draw[dashed] (3.5,7) node[above] {\footnotesize $x_{1}^{B}=3.5$} -- (3.5,0) node[below] {\footnotesize $x_{1}^{A}=3.5$};
				\draw[dashed] (0,3.58) node[left] {\footnotesize $x_{2}^{A}=3.64$} -- (7,3.58)node[right] {\footnotesize $x_{2}^{B}=3.36$};
		\end{tikzpicture}
	\end{center}
		
		\item [9.] Hay muchos compradores de autos deportivos de color rojo. Los \emph{``snobs''} (en proporción $q$) están dispuestos a pagar hasta $50.000$ dólares por un auto de color rojo (logrando una utilidad $U_{s} > 50.000$ dólares). Por otra parte, los \emph{``menos snobs''} (en proporción $1-q$) pagarían hasta $30.000$ dólares por un auto de color rojo logrando una utilidad $U_{M}>30.000$ dólares. Se sabe que $U_S > U_{M}$. Los vendedores de autos tienen que elegir el tipo de contrato que desean ofrecer a cada comprador le autos rojos. Teniendo en cuenta que la utilidad de reserva de los compradores \emph{``snobs''} es de $U_{S}=25.000$ dólares y la de los \emph{``menos snobs''} es de $U_{M}=15.000$ dólares. ¿Cómo diseñan los vendedores sus contratos, en selección adversa? (formule el problema sin resolverlo).\\
		
			\textbf{\large Solución}\\
				\begin{itemize}
	\item Pregunta 9a:
			Para el agente $A$:
				\begin{align*}
					& \text{Max } \quad U_A = \alpha \ln(x_A) + (1-\alpha)\ln(x_A) \\[0.2cm]
					& \begin{array}{ll}
						\text{s.a: } & p_xx_A + p_yy_A = \overline{y}p_y + \overline{x}p_x \qquad (\overline{y}=1, \overline{x}=0)
					\end{array}
				\end{align*}
			
			Se forma el lagrange y se aplican las condiciones de primer orden:
			$$ \mathscr{L} =  \ln(x_A) + (1-\alpha)\ln(x_A) - \lambda\left[p_y - p_xx_A - p_yy_A  \right]$$
			
				$$\left.
					\begin{array}{l}
						\frac{\partial \mathscr{L}}{\partial x_{A}} = \frac{\alpha}{x_A} - \lambda p_x= 0\\[0.4cm]
						\frac{\partial \mathscr{L}}{\partial y_{A}} = \frac{(1-\alpha)}{y_A} - \lambda p_y=0
					\end{array}
				  \right\} \Longrightarrow 
				    \begin{array}{ccc}
						\frac{\alpha}{x_Ap_x} & = & \frac{(1-\alpha)}{y_Ap_y}  \\[0.3cm]
				  		y_A & = & \frac{(1-\alpha)}{\alpha p_y}x_Ap_x
				    \end{array}$$
			
				\begin{align*}
					p_xx_A + p_yy_A & = p_y\\[0.3cm]
					p_xx_A + p_y\left( \frac{(1-\alpha)}{\alpha p_y}x_Ap_x\right)  & = p_y\\[0.3cm]
					x_{A}^* & = \frac{\alpha p_y}{p_x}\\[0.3cm]
					y_{A}^* & = 1 - \alpha
				\end{align*}
			
			Para el agente $B$:
				\begin{align*}
					& \text{Max } \quad U_B = \text{Min} \left\lbrace x_B, y_B\right\rbrace \\[0.2cm]
					& \begin{array}{ll}
						\text{s.a: } & p_xx_A + p_yy_A = \overline{y}p_y + \overline{x}p_x  \qquad (\overline{y}=0, \overline{x}=1)
					  \end{array}
				\end{align*}
			
			Como $x$ e $y$ son complementarios perfectos
				$$x_B = y_B$$
			Y reemplazamos en la recta presupuestaria
				\begin{align*}
					p_xx_B + p_yy_B & = p_x\\[0.3cm]
					p_xx_B + p_yx_B  & = p_x\\[0.3cm]
					x_B\left( p_x + p_y\right) &= p_x\\[0.3cm]
					x_{B}^* & = \frac{p_x}{p_x+p_y} = y_{B}^*
				\end{align*}
			El criterio de la condición de factibilidad nos indica que los mercados se limpian; es decir, la demanda debe ser igual a la oferta (dotación)
				\begin{itemize}
					\item $x_a + x_B = \overline{x}_A + \overline{x}_B \Longrightarrow (1 - \alpha) + \frac{p_x}{p_x+p_y} = 0 + 1$
					\item $y_a + y_B = \overline{x}_A + \overline{x}_B \Longrightarrow \frac{\alpha p_y}{p_x} + \frac{p_x}{p_x+p_y} = 1 + 0$
				\end{itemize}
			Usando la expresión de $y$, los precios relativos son:
				$$\frac{p_y}{p_x} = \frac{1-\alpha}{\alpha}$$
	\item Pregunta 9b:
			$$\nexists \enskip p < 0 \enskip \Longrightarrow \enskip \frac{1 - \alpha}{\alpha} > 0 \enskip \Longrightarrow \enskip  \alpha \in <0,1>$$
\end{itemize}
	\end{enumerate}
	
\textbf{\LARGE Riesgo Moral}
	\begin{enumerate}
		\item [10.] Un trabajador puede ejerce dos esfuerzos, ``bueno'' y ``malo'', lo que induce la observación de fallos en el resultado con probabilidades 0.25 y 0.75 respectivamente. Su función de utilidad es $U(w,e) = 100-(10/w)-v$ , donde $w$es el salario que recibe, y $v$ toma el valor 2 si incorpora el buen esfuerzo y 0 si elige el malo. La aparición de un fallo (por ejemplo, una avería) es una variable que puede ser incluida en los términos del contrato, pero el esfuerzo no. El producto que se obtiene vale 20 si no hay errores, y no vale nada si tiene alguno. El principal es neutral ante el riesgo. Suponemos que el trabajador tiene una utilidad de reserva igual a $\underline{U} =0$. Calcule el contrato óptimo y el esfuerzo que el principal desea conseguir, en condiciones de información simétrica y de información asimétrica sobre el comportamiento del agente.\\
			
			\textbf{\large Solución}\\
				El punto $E$ indica la situación inicial de ambos consumidores. En este punto, el agente $A$ tiene una dotación de $(1,2)$; mientras que el $B$ una de $(2,1)$. La curva de contrato viene dada por la línea gruesa y el área morada señalada en el siguiente gráfico.

\begin{center}
	\begin{tikzpicture}[scale=1.7]
		% Formato de CAJA
		\draw[->] (0,0) node[align=center, below left] {\footnotesize $O_A$} -- (0,4) node[align=center, above] {\footnotesize $y_A$};
		\draw[->] (0,0) -- (4,0) node[align=center, right] {\footnotesize $x_{A}$};
		
		\draw[->] (3,3) node[align=center, above right] {\footnotesize $O_B$} -- (-1,3) node[align=center, left] {\footnotesize $x_{B}$};
		\draw[->] (3,3) -- (3,-1) node[align=center, below] {\footnotesize $y_{B}$};
		
		% 
		
		% Curvas de indiferencia
		\draw (0.5,3) -- (0.5,0.5) -- (3,0.5);
		\draw (2.5,3) -- (2.5,2.5) -- (3,2.5);
		
		\draw (2,3) -- (3,2);
		\draw (0,2.5) -- (2.5,0);
		\draw (0.7,3) -- (3,0.7);
		\draw (0,1) -- (1,0);
		
		% Curva de contrato       
		\draw [purple, very thick] (0,0)  -- (1,1);
		\draw [purple, very thick] (2,2)  -- (3,3);
		\draw [purple, very thick] (1,1)  -- (1,3) -- (2,3) -- (2,2) -- (3,2) -- (3,1) -- (1,1);
		\fill [color=purple!10](1,1)  -- (1,3) -- (2,3) -- (2,2) -- (3,2) -- (3,1) -- (1,1);
		
		% Punto
		\draw[black, fill=black] (1,2) circle[radius=0.04] node[align=center, right, scale = 0.25mm] {\footnotesize $E(1,2)$};
	\end{tikzpicture}
\end{center}
		\item [11.] Sea una relación de agencia en la que le principal contrata a un agente cuyo esfuerzo determina el resultado. Supongamos que la incertidumbre está representada  por tres estados de la naturaleza . El agente puede elegir entre dos esfuerzo. Los resultados están recogidos en la tabla siguiente
						\begin{center}
							\begingroup
								\setlength{\tabcolsep}{10pt} % Default value: 6pt
								\renewcommand{\arraystretch}{1.5} % Default value: 1
									\begin{tabular}{cc|ccc}
												&& \multicolumn{3}{c}{Estados de la naturaleza}\\
												&& $\varepsilon_1$ & $\varepsilon_2$ & $\varepsilon_3$\\
										\hline
											\multirow{2}{*}{Esfuerzos} & e=6 & 60.000 & 60.000 & 30.000\\
																	   & e=4 & 30.000 & 60.000 & 30.000
									\end{tabular}
							\endgroup
						\end{center}
		El principal y el agente creen que la probabilidad de cada uno de los estados es un tercio. Las funciones objetivos del principal y del agente son, respectivamente:
				\begin{align*}
					B(x,w) & = x-w\\
					U(w,e) & = \sqrt{w} - e^2
				\end{align*}
		donde $x=x(e,\varepsilon)$ es el resultado monetario de la relación, $y=w(x)$ representa el pago monetario que recibe el agente. Supondremos que el agente solo acepta el contrato si obtiene por lo menos una utilidad esperada de 114 (su nivel de utilidad de reserva)
			\begin{enumerate}[a)]
				\item ¿Qué puede deducirse de las funciones objetivo de los participantes?
				\item ¿Cuál será el esfuerzo y el pago en una situación de información simétrica? ¿Qué ocurriría si el principal no fuese neutral ante el riesgo?
				\item ¿Qué ocurre en una situación de información asimétrica? ¿Qué esquema de pago permite conseguir el esfuerzo $e=4$? ¿Qué esquema de pago permite conseguir el esfuerzo $e=6$? ¿Qué esfuerzo decide conseguir el principal? Analice el resultado
			\end{enumerate}
					\textbf{\large Solución}\\
						\begin{enumerate}[a)]
	\item El principal es neutral ante el riesgo y el agente averso
	\item Los contrato óptimos se derivan de 
				\begin{enumerate}[i)]
					\item el principal asume todo el riesgo
					\item la restricción de aceptación está saturada
				\end{enumerate}
			Si $e=6$, $w$ es tal que $w^{1/2} - 6^2 =114$. Luego $w=22.500$. En este caso, $E[U_P] = 50.000-22.500=27.500$. Si $e=4$, entonces $w=16.900$ y $E[U_P]=23.100$. La solución en información simétrica es $e^*=6$, $w^*=22.500$. Si el principal no fuese neutral al riego, ambos participantes se repartirían el riesgo de la relación.
	\item El contrato óptimo si $e=4$ es el mismo que antes: $w=16.900$, ya que ante un pago fijo el agente siempre escoge el esfuerzo más bajo. Para alcanzar $e=6$, el principal ofrecerá un contrato contingente al resultado. Pagará $w(60)$ si el resultado es 60.000 y $w(30)$ si el resultado es 30.000. El contrato debe verificar simultáneamente las restricciones de participación y de incentivo:
				\begin{align*}
					& \frac{2}{3}\left[w(60) \right]^{1/2}+\frac{1}{3}\left[w(30) \right]^{1/2} - 36 \geq 114\\
					& \frac{2}{3}\left[w(60) \right]^{1/2}+\frac{1}{3}\left[w(30) \right]^{1/2} - 36 \geq \frac{2}{3}\left[w(30) \right]^{1/2}+\frac{1}{3}\left[w(60) \right]^{1/2} - 16
				\end{align*}
			Ambas restricciones se darán con igualdad en la solución al problema de ``gastar lo menos posible'' de principal. Tenemos dos ecuaciones y dos incógnitas que llevan a $w(60)=28.900$ y $w(30)=12.100$. La utilidad esperada del principal es $U_P=26.700$. En información asimétrica elige también $e=6$, ya que $26.700 > 23.100$, pero con una pérdida en eficiencia medidad por la disminución de los beneficios esperados del principal (el agente siempre obtiene su utilidad de reserva)
\end{enumerate}
		\item [12.] Supongamos una relación entre un empresario (principal) y un agente en la que dos resultados son posibles cuyos valores son 50.000 y 25.000. El agente tiene que elegir entre tres posibles esfuerzos que tienen efecto sobre la distribución de los resultados. La distribución de probabilidades sobre los resultados en función de los esfuerzos viene dada por la tabla siguiente:
				\begin{center}
					\begingroup
						\setlength{\tabcolsep}{10pt} % Default value: 6pt
						\renewcommand{\arraystretch}{1.5} % Default value: 1
							\begin{tabular}{cc|cc}
									&& \multicolumn{2}{c}{Resultados}\\
											&& 25.000 & 50.000\\
								\hline
									\multirow{3}{*}{Esfuerzos} & $e^1$ & 1/4 & 3/4\\
															   & $e^2$ & 1/2 & 1/2\\
															   & $e^3$ & 3/4 & 1/4
							\end{tabular}
						\endgroup
				\end{center}
			Suponemos que el principal es neutral ante el riesgo y el agente adverso, con las preferencias descritas respectivamente por las siguientes funciones:
			$$B(x,w)=x-w \quad \text{y} \quad U(w,e)=\sqrt{w}-v(e)$$
			con $v\left(e^{1}\right)=10$, $v\left(e^{2}\right)=4$, y $v\left(e^{3}\right)=0$. El nivel de utilidad de reserva del agente es $\underline{U}=120$.
				\begin{enumerate}[a)]
					\item Escríbanse los contrato óptimos en información simétrica para cada nivel de esfuerzo y los beneficios obtenidos por el principal en cada caso. ¿Qué nivel de esfuerzo prefiere el principal?
					\item Escríbase los contrato óptimos cuando existe un problema de riesgo moral. ¿Cuál es el nivel d esfuerzo y el contrato que el princiapl elige? ¿Dónde influye el problema de riesgo moral?
				\end{enumerate}
					\textbf{\large Solución}\\
						\begin{enumerate}[a)]
	\item En información simétrica el agente recibirá un pago fijo, determinado por su restricción de participación.
			\begin{enumerate}[i)]
				\item Si $e^1$, el salario es $w=25.600$ y los beneficios del principal $U_P = 18.150$.
				\item Si $e^2$, $w=19.600$ y $U_P=17.900$.
				\item Si $e^3$, $w=15.625$ y $U_P=15.625$
			\end{enumerate}
		Por tanto, en información simétrica, el principal elige $e^1$.
	\item Si existe un problema de riesgo moral, entonces
			\begin{enumerate}[i)]
				\item cuando el principal desea conseguir el esfuerzo $e^1$, construye el mecanismo de pago contingente al resultado $\left[ w(25), w(50)\right]$ siguiendo el programa:
					\begin{align*}
						& \M \limits_{w(25), w(50)} \quad \frac{1}{4} w(25)+\frac{3}{4} w(50)\\[0.2cm]
						& \begin{array}{ll}
							\text{s.a.}  & \frac{1}{4} w(25)^{1 / 2}+\frac{3}{4} w(50)^{1 / 2}-40 \geq 120 \\[0.2cm]
										 & \frac{1}{4} w(25)^{1 / 2}+\frac{3}{4} w(50)^{1 / 2}-40 \geq \frac{1}{2} w(25)^{1 / 2}+\frac{1}{2} w(50)^{1 / 2}-20 \\[0.2cm]
										 & \frac{1}{4} w(25)^{1 / 2}+\frac{3}{4} w(50)^{1 / 2}-40 \geq \frac{3}{4} w(25)^{1 / 2}+\frac{1}{4} w(50)^{1 / 2}-5
						  \end{array}
					\end{align*}
				La solución es $w(25)=10.000, w(50)=32.400$ y $U_P=16.950$
				
				\item Si el principal quiere el esfuerzo $e^2$, entonces el mecanismo de pago $w(25)=12.100, w(50)=28.900$ y $U_P=17.00$\\
				
						De forma rigurosa, $s(25) = w(25)^{1/2}$ y $s(50)=w(50)^{1/2}$, el programa será
							\begin{align*}
								& \M \limits_{s(25), s(50)} \quad -s(25)^{2}-s(50)^{2}\\
								& \begin{array}{ll}
									\text{s.a.} & s(25)+s(50) \geq 280 \\
												& s(25)-s(50) \leq 80  \\
												& s(50)-s(25) \geq 60
								  \end{array}
							\end{align*}
						Sean $\lambda, \mu$ y $\delta$ los multiplicadores de las tres restricciones. Las condiciones de primer orden del lagrangiano se escriben: $-2s(25)+\lambda+\mu-\delta=0$ y $-2s(50)+\lambda-\mu+\delta$, lo que implica que $\lambda=s(50)+s(25)$. La primera restricción se verifica con igualdad. Como las restricciones segunda y tercera son incompatibles con igualdad, una de ellas estará no saturada, luego o biene $\mu=0$, o bien $\delta=0$. Si $\delta=0$, entonces, por las condiciones de primer orden del lagrangiano y por la expresión de $\lambda$, tenemos que $\mu=s(25)-s(50)$, pero ello es imposible pues $s(50)-s(25)\geq 60\geq0$. Luego $\mu=0$, con lo que $\delta=s(50)-s(25)>0$. Esto implica que la tercera restricción se verifica con igualdad. $s(25)$ y $s(50)$ salen, pues, del sistema formado por la primera y tercera restricciones verificadas con igualdad.
						
				\item El contrato óptimo para conseguir el esfuerzo $e^3$ es el mismo que en información simétrica, luego $w(25)=w(50)=15.625$ y $U_P=15.625$. En información asimétrica el principal elige el esfuerzo $e^2$. La información asimétrica no solo influye en que, dado $e^2$, los beneficios son menores $(17.00<17.900)$, sino que el esfuerzo elegido también es distinto ($e^2<e^1$)
			\end{enumerate}
\end{enumerate}
	\end{enumerate}
%------------------------------------------------------------------------------------
\end{document}		
%====================================================================================