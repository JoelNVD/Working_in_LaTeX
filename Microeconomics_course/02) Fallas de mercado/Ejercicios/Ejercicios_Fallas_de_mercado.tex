%====================================================================================
% Preamble
%------------------------------------------------------------------------------------
\documentclass[10pt,a4paper]{article}

% Apartado de texto
\usepackage[utf8]{inputenc}
\usepackage[spanish]{babel}
\usepackage[T1]{fontenc}

% Apartado matemático
\usepackage{amsmath}
\usepackage{amsfonts}
\usepackage{amssymb}
\usepackage{mathtools}
\usepackage{mathrsfs}
\usepackage[libertine]{newtxmath}

% Apartadode no sangría
\usepackage{parskip}

% Aparatod posición del texto
\usepackage[left=2cm,right=2cm,top=2cm,bottom=2cm]{geometry}
\usepackage{fancyhdr}

\usepackage{graphicx}
\usepackage{xcolor}
\definecolor{cardinal}{rgb}{0.77, 0.12, 0.23}

% Apartado de columnas y filas: juntarlas o separarlas
\usepackage{multicol, multirow}

% Apartado differents label en listado
\usepackage[shortlabels]{enumitem}

% Apartado hyperres
\usepackage[hidelinks]{hyperref}

\usepackage{tcolorbox}
\newtcolorbox{nota}[1]{
	colback=cardinal!5!white,
	colframe=cardinal!75!black,
	fonttitle=\bfseries,
	title=#1
}

% Apartado dibujos
\usepackage{tikz, pgfplots}
\pgfplotsset{compat=1.18}
\usetikzlibrary{positioning,calc,arrows}
\usetikzlibrary{shapes.arrows}
\usetikzlibrary{patterns}
\usetikzlibrary{babel} % Para que recono > y <, en inglés no se pone, pero en spañol sí
\usetikzlibrary{shadings,shadows}
\usetikzlibrary{matrix}
\usetikzlibrary{intersections}

% Apartado cambio como por punto
\decimalpoint

% \highlight[<colour>]{<stuff>}
\newcommand{\highlight}[2][yellow]{\mathchoice%
	{\colorbox{#1}{$\displaystyle#2$}}%
	{\colorbox{#1}{$\textstyle#2$}}%
	{\colorbox{#1}{$\scriptstyle#2$}}%
	{\colorbox{#1}{$\scriptscriptstyle#2$}}}%

% Definir oprerador
\DeclareMathOperator*{\M}{Max}

%====================================================================================

%====================================================================================
% Body
%====================================================================================
% Title Page
%-----------
\textwidth=450pt \textheight=620pt \oddsidemargin=0in
\topmargin=-10pt
\pagestyle{fancy} \rhead{\scriptsize{\textbf{Código del Curso:} xxXXXXX} \\
	\textbf{Fecha:} XX/XX/2021 \& 2021-I \hspace{0.04cm}}
\lhead{\scriptsize{\textbf{Profesor:} José A. Valderrama}  \\
	\textbf{Curso:} Teoría Microeconómica II}
\newcommand{\re}[1]{\smallskip\textsf{\textbf{Respuesta}} \begin{sf}\\ #1 \end{sf} \bigskip}
\newcommand{\ay}[1]{ \scriptsize{\textsl{Hint: #1}}\normalsize{}}
\newcommand{\pr}[2]{\frac{\partial #1}{\partial #2}}

%------------------------------------------------------------------------------------
% Title
%---------
\begin{document}
	\begin{center}
		{\Large {\textbf{Fallas de mercado}}}		
	\end{center}
% EJERCICIOS---------------------------------------------------------------------------------
\textbf{\LARGE Externalidades}
	\begin{enumerate}
		\item Una industria refinadora competitiva arroja una unidad de desperdicio en la atmósfera por cada unidad de producto refinado. La función inversa de demanda del producto refinado es $p=30-q$. La curva inversa de oferta para el refinamiento es $CM=3+q$. La curva de costo externo marginal es $CEM=1.5q$,  donde CEM es el costo externo marginal cuando la industria arroja q unidades de desperdicio. ¿Cuál es el precio y cantidad de equilibrio para el producto refinado?\\
		
			\textbf{\large Solución}\\
				Se tiene la función inversa de demanda:  
		$$P = 30-q$$
La función inversa de oferta es:
		$$CM = 3 + q$$
Por tanto la cantidad de equilibrio se da cuando:
		$$CM = P$$
Por tanto:
	\begin{align*}
		3+q &= 30-q\\
		2q &= 27\\
		q^* &= 13.5
	\end{align*}
Hallando el precio de equilibrio:
	\begin{align*}
		p &= 30-q^*\\
		p &= 30-13.5\\
		p^* &= 16.5
	\end{align*}
		
		\item De la pregunta anterior, ¿cuánto debe ser la oferta de mercado en el óptimo social?\\
		
			\textbf{\large Solución}\\
				El tamaño de la caja será de $15$ unidades del bien 1 por 12 del bien 2. Se calcula las utilidades del punto inicial y de los propuestos:

	\begin{center}
		\begingroup
			\setlength{\tabcolsep}{10pt} % Default value: 6pt
			\renewcommand{\arraystretch}{1.5} % Default value: 1
				\begin{tabular}{ccccc}
						\hline
					Consumidor $A$ & {} & Consumidor $B$ & {} & Situación \\
						\hline
					$U_A(12,3) = 36$ & {} & $U_B(3,9) = 27$ & {} & Inicial\\
					$U_A(9,5)  = 45$ & {} & $U_B(6,7) = 42$ & {} & Ganan\\
					$U_A(8,9)  = 72$ & {} & $U_B(7,3) = 21$ & {} & A gana y B pierde\\
						\hline
				\end{tabular}
		\endgroup
	\end{center}

En general cualquier punto perteneciente al área de intercambio cumple con lo siguiente:
	\begin{gather*}
		U^{A}\left(x_{1}^{A}, x_{2}^{A}\right) \geq U^{A}\left(w_{1}^{A}, w_{2}^{A}\right) \\
		U^{B}\left(x_{1}^{B}, x_{2}^{B}\right) \geq U^{B}\left(w_{1}^{B}, w_{2}^{B}\right)
	\end{gather*}

En este ejercicio sería así:
	$$x_{1}^{A} \cdot x_{2}^{A} \geq 36 \qquad x_{1}^{B} \cdot x_{2}^{B} \geq 27$$

y sabiendo que 
	$$x_{1}^{B} = 36 - x_{1}^{A} \qquad x_{2}^{B} = 27 - x_{2}^{A}$$
	\vspace{-0.8cm}
gráficamente
\begin{center}
	\begin{tikzpicture}[scale=1.1]
		% Formación de la caja
			% Consumidor A
				\draw[->] (0.5,0.5) node[align=center, below left] {\footnotesize $O_A$} -- (0.5,4.5) node[align=center, above] {\footnotesize $x_{1}^{A}$};
				\draw[->] (0.5,0.5) -- (8.5,0.5) node[align=center, right] {\footnotesize $x_{2}^{A}$};
			
			%Consumidor B
				\draw[->] (8,4) node[align=center, above right] {\footnotesize $O_B$} -- (0,4) node[align=center, left] {\footnotesize $x_{1}^{B}$};
				\draw[->] (8,4) -- (8,0) node[align=center, below] {\footnotesize $x_{2}^{B}$};
		
		% Curvas de indiferencia
			\begin{axis}[
						hide axis,
						xmin=0, xmax=10, 
						ymin=0, ymax=10,
						ytick=\empty,
						]
				% Área sombreada
						\fill [pattern=crosshatch dots,pattern color=green!60!white] (axis cs:4.07,6.462) to [bend right=34] coordinate[pos=0.5] (l_i) (axis cs:8.43,1.57) to (axis cs:8.42,1.56) to [bend right=34] coordinate[pos=0] (l_i) (axis cs:4,6.461);
				
				% Curvas de indiferencia
					% Agente A
						\draw [blue] (axis cs:4,6.9) to [bend right=40] coordinate[pos=1] (l_i) (axis cs:9,1.5);
					% Agente B
						\draw [red] (axis cs:8.5,1) to [bend right=40] coordinate[pos=0.5] (l_i) (axis cs:3.6,6.5);
				
			\end{axis}
		
		% Punto
			\draw[black, fill=black] (5.775,0.89) circle[radius=0.05] node [above right] {$w$};
			\draw[black, fill=black] (4.775,1.95) circle[radius=0.05] node [above] {\footnotesize $(9,5)$};
			\draw[black, fill=black] (4.4,3.4) circle[radius=0.05] node [above] {\footnotesize (8,9)};
			
			% Intersección
				\draw [dashed] (5.77,0.5) node [below]{\footnotesize $x_{1}^{A}=12$} -- (5.77,4)node [above] {\footnotesize $x_{1}^{B}=3$};
				\draw [dashed] (0.5,0.89) node [left] {\footnotesize $3=x_{2}^{A}$}-- (8,0.89) node [right] {\footnotesize $x_{2}^{B}=9$};
	
		% Etiqueta de función de utilidad
			\node [above, blue] at (3.1,1.5) {$36=u_{A}$};
			\node [below, red]  at (5.5,3.1) {$u_{B}=27$};
	\end{tikzpicture}
\end{center}
		
		\item Suponga que donde usted vive hay una empresa contaminante cuyo producto es vendido a un precio de 30 soles. Asumimos que la función de costo marginal es: $CMG = 5 + 2Q$, por último el costo marginal externo es: $CME = 0.5Q$. Se pide la cantidad que debe producir la empresa para alcanzar la externalidad óptima.\\
		
			\textbf{\large Solución}\\
				En el óptimo social, se cumple lo siguiente:
	$$CM + CEM = P$$
Entonces:
	\begin{align*}
		CM + CEM &= P\\
		\left(5+2q\right) + \left(0.5q\right) &= 30\\
		5 + 2.5q &= 30\\
		2.5q &= 25\\
		q^* &= 10
	\end{align*}
	\end{enumerate}

\textbf{\LARGE Bienes públicos}
	\begin{enumerate}
		\item [4.] Considere  dos consumidores con las siguientes funciones de demanda por un bien público: $P_{1}=10 - \frac{G}{10}$, $P_{2}= 20 - \frac{G}{10}$, donde $P_{i}$ es el precio que i está deseando pagar por la cantidad $G$. ¿Cuál es el nivel óptimo de consumo del bien público, si su costo marginal es igual a 25?\\
		
			\textbf{\large Solución}\\
				\begin{enumerate}[a)]
	\item Dado que las funciones son tipo \emph{Cobb-Douglas}, se igualan \emph{TMS}
			$$\frac{2\frac{3}{4}x_{1}^{-1/4}y_{1}^{1/4}}{2\frac{1}{4}x_{1}^{3/4}y_{1}^{-3/4}} = \frac{\frac{3}{4}x_{2}^{-1/4}y_{2}^{1/4}}{\frac{1}{4}x_{1}^{3/4}y_{2}^{-3/4}} \rightarrow \frac{y_1}{x_1} = \frac{y_2}{x_2} \rightarrow \frac{y_1}{x_1} = \frac{100- y_1}{100-x_1} \rightarrow y_1 = x_1 \text{ y } y_2 = x_2$$
		  Reemplazando en las F.U.
			$$
			\begin{array}{c|c}
				U_1 = 2x_{1}^{3/4}\left( x_1\right)^{1/4} & U_2 = x_{2}^{3/4}\left( x_{2}\right)^{1/4}\\[0.3cm]
				x_1=\frac{U_1}{2} & x_2 = U_2
			\end{array}
			$$
		  Reemplazando en la dotación de $x$ (o de $y$)
			$$x_1 + x_2 = 100 \longrightarrow \frac{U_1}{2} + U_2 = 100 \longrightarrow \therefore U_2 = 100 - \frac{U_1}{2}$$
		  Gráficamente
			\begin{center}
				\begin{tikzpicture}[samples = 100, scale=0.8]
					\draw[purple, domain = 0:10] plot({\x},{10-\x/2});
					\draw[<->] (0,10.5) node [above] {$U_2$} -- (0,5) -- (10.5,5) node [right] {$U_1$};
					\draw (4,9) node[right, purple] {$U_2 = 100 - \frac{U_1}{2}$};
				\end{tikzpicture}
			\end{center}
	\item Utilizando la siguiente expresión que se desprende de una \emph{Cobb-Douglas}\\
			$$U(x,y) = x^\alpha y^\beta$$
			$$
				\begin{array}{c|c}
					\highlight{x^M=\frac{\alpha I}{(\alpha + \beta)p_x}} & \highlight{y^M=\frac{\beta I}{(\alpha + \beta)p_y}}
				\end{array}
			$$
		  Entonces
		  	$$
		  		\begin{array}{c|c}
					x_{1}^{M} = \frac{\frac{3}{4}(20p_x+40p_y)}{\left( \frac{3}{4} + \frac{1}{4}\right)p_x} & x_{2}^{M} = \frac{\frac{3}{4}(80p_x+60p_y)}{\left( \frac{3}{4} + \frac{1}{4}\right)p_x}\\[0.3cm]
					\frac{15p_x+30p_y}{p_x} & \frac{60p_x+45p_y}{p_x}
		  		\end{array}
		  	$$
		  Reemplazando en la dotación de $x$ (o de $y$)
		  	$$x_A + x_B = 100 \longrightarrow \frac{15p_x+30p_y}{p_x} + \frac{60p_x+45p_y}{p_x} = 100 \longrightarrow \frac{p_y}{p_x} = \frac{1}{3}$$
		  Con el mismo procedimiento para $y$, se obtendrán los siguientes resultados:
		  	$$
		  		\begin{array}{c|c}
		  			x_{1}^* = 25 & y_{1}^*=25\\[0.3cm]
		  			x_{1}^* + x_2 = 100 & y_{1}^* + y_2 = 100\\[0.3cm]
		  			x_{2}^* = 75 & y_{2}^*=75
		  		\end{array}
		  	$$
		  Finalmente, reemplazando las dotaciones en las F.U.
		  		\begin{center}
		  			\begingroup
			  			\setlength{\tabcolsep}{10pt} % Default value: 6pt
			  			\renewcommand{\arraystretch}{1.5} % Default value: 1
				  			\begin{tabular}{ccc}
				  					\hline
				  				{} & $U_1$ & $U_2$ \\
					  				\hline
				  				Dotación inicial & $U_1 = 2(20)^{3/4}(40)^{1/4} \approx 47.57$ & $U_2 = (80)^{3/4}(60)^{1/4} \approx 74.45$\\
				  				Pareto & $U_1 = 2(25)^{3/4}(25)^{1/4} = 50$ & $U_2 = (75)^{3/4}(75)^{1/4} = 75$\\
					  				\hline
				  			\end{tabular}
		  			\endgroup
		  		\end{center}
	  	Podemos identificar 2 puntos, $A = (47.57, 74.45)$ y $B = (50,75)$. Gráficamente, ubicamos estos puntos en la \emph{FPU}.
	  		\begin{center}
	  			\begin{tikzpicture}[samples = 100, scale=0.8]
	  				\draw[purple, domain = 0:10] plot({\x},{10-\x/2});
	  				\draw[<->] (0,10.5) node [above] {$U_2$} -- (0,5) -- (10.5,5) node [right] {$U_1$};
	  				\draw (4,9) node[right, purple] {$U_2 = 100 - \frac{U_1}{2}$};
	  				\draw[dashed] (0,6.9) node [left] {74.45}-- (4.1,6.9) -- (4.1,5) node [below] {47.57};
	  				\draw[dashed] (0,7.5) node [left] {75}-- (5,7.5) -- (5,5) node [below] {50};
	  				\draw[black, fill=black] (4.1,6.9) circle[radius=0.05] node [above right] {$A$};
	  				\draw[black, fill=black] (5,7.5) circle[radius=0.05] node [above right] {$B$};
	  			\end{tikzpicture}
	  		\end{center}
\end{enumerate}
		
		\item [5.] Sean dos estudiantes: Eva y Teresa, que comparten una habitación. Ambas tienen la misma función de utilidad respecto de los cuadros (bien $X$) y de las cervezas (bien $Y$).\\ 
		La función de utilidad de las 2 estudiantes es la siguiente:
			$$U_{i} = X^{\frac{1}{3}}_{i}Y^{\frac{2}{3}}_{i}$$
		Donde i representa a cada estudiante. Cada estudiante tiene un ingreso de 3000 para gastar. El precio de $X$ es 100 y el precio de $Y$ es 0.2. Se pide hallar el consumo de cuadros y cervezas, si cada persona vive sola.\\
			\textbf{\large Solución}\\
				Para hallar el consumo óptimo de cuadros y cervezas de cada estudiante cuando vive solo. Se aplica lo siguiente:
	$$\frac{Umg_{X}^{i}}{Umg_{Y}^{i}} = \frac{P_{X}}{P_{Y}}$$
Resolviendo:
	\begin{align*}
		\frac{\frac{1}{3}X^{\frac{-2}{3}}_{i}Y^{\frac{2}{3}}_{i} }{\frac{2}{3}X^{\frac{1}{3}}_{i}Y^{\frac{-1}{3}}_{i}} &= \frac{P_{X}}{P_{Y}}\\
		\frac{Y_{i}}{2X_{i}} &= \frac{P_{X}}{P_{Y}}
	\end{align*}
Tenemos la restricción presupuestaria:
	$$X_{i}P_{X} + Y_{i}P_{Y} = M$$
Hallando $X_{i}$:
	\begin{align*}
		X_{i}P_{X} + 2X_{i}\frac{P_{X}}{P_{Y}}P_{Y} &= M\\
		3X_{i}P_{X} &= M\\
		X_{i}^* &= \frac{M}{3P_{X}}
	\end{align*}
Hallando $Y_{i}$:    
	\begin{align*}
		Y^*_{i} &= 2X_{i}^*\frac{P_{X}}{P_{Y}}\\
		Y^*_{i} &= 2\frac{M}{3P_{X}}\frac{P_{X}}{P_{Y}}\\
		Y^*_{i} &= \frac{2M}{3P_{Y}}
	\end{align*}
Reemplazando los datos: $M=3000$, $P_{X}=100$, $P_{Y}=0.2$, se tiene los consumos óptimos.
	\begin{align*}
		X^*_{i} &= 10\\
		Y^*_{i} &= 10000
	\end{align*}
		
		\item [6.] De la pregunta anterior, si ambas viven juntas. ¿Cuál es el nivel óptimo del bien público (bien $X$)?\\
			\textbf{\large Solución}\\
				\begin{enumerate}[a)]
	\item El primer \emph{TFB} dice que un equilibrio general competitivo (\emph{EGC}) es Pareto-eficiente. Para tener un \emph{EGC}; sin embargo, es necesario que no haya presencia de bienes públicos, externalidades, información asimétrica, mercados incompletos y que la competencia perfecta esté asegurada.\\
	
		Se puede esbozar una prueba del primer \emph{TFB} de la siguiente manera
			\begin{center}
				\begingroup
					\setlength{\tabcolsep}{10pt} % Default value: 6pt
					\renewcommand{\arraystretch}{1.5} % Default value: 1
						\begin{tabular}{ccc}
							\hline
								\textbf{Optimización del consumidor} & $\rightarrow$ & \textbf{Eficiencia en el intercambio}\\
							\hline
								$TMS_{x,y}^A = p_x/p_y$ &\multirow{2}{*}{$\rightarrow$}	& \multirow{2}{*}{$TMS_{x,y}^A = TMS_{x,y}^B$}\\
								$TMS_{x,y}^B = p_x/p_y$	\\
							\hline
								\textbf{Minimización del costo} & $\rightarrow$ & \textbf{Eficiencia en al producción}\\
							\hline
								$TMST_{L,K}^x = w/r$ &\multirow{2}{*}{$\rightarrow$}	& \multirow{2}{*}{$TMST_{L,K}^x = TMST_{L,K}^y$}\\
								$TMST_{L,K}^y = w/r$ \\
							\hline
								\textbf{Maximización del beneficio} & $\rightarrow$ & \textbf{Eficiencia de alto nivel}\\
							\hline
								$p_x = CMg_x$ &\multirow{2}{*}{$\rightarrow$}	& \multirow{2}{*}{$TMT_{x,y} = \frac{CMg_x}{CMg_y} = \frac{p_x}{p_y}=TMS_{x,y}^A = TMS_{x,y}^B$}\\
								$p_y = CMg_y$\\
							\hline
								$\downarrow$ && $\downarrow$\\
							\hline
								\textbf{Equilibrio General Competitivo} & $\rightarrow$ & \textbf{Pareto-eficiente}\\
							\hline
						\end{tabular}
				\endgroup
			\end{center}
	\item La afirmación no es correcta. Según el segundo \emph{TFB}, siempre que las preferencias y la tecnología sean convexas y se cumplan todos los demás supuestos del primer \emph{TFB}, se puede obtener cualquier asignación Pareto-eficiente dada una adecuada redistribución de los recursos iniciales. A lo más, el gobierno debería encargarse de la correcta asignación inicial de recursos, pero el mercado podría llevarlo a una solución Pareto-eficiente. Sin embargo, hay que decir que este resultado es de difícil aplicación en el mundo real.
\end{enumerate}
	\end{enumerate}

\textbf{\LARGE Selección Adversa}
	\begin{enumerate}
		\item [7.] Supongamos que un empresario (neutral ante el riesgo) quiere contratar a un trabajador, pero no conoce todas las características de dicho trabajador. Lo que desconoce es la productividad que el esfuerzo del trabajador tiene en el proceso de producción para el que desea contratarle. Sabe, sin embargo, que el trabajador es neutral ante el riesgo y que puede ser de dos tipos: o bien su productividad es alta, con lo que su esfuerzo es igual a $e^{2}$, o bien es baja y su esfuerzo es igual a $e^{2}$
		Al primer tipo de trabajador le llamaremos  ``$B$'' y al segundo ``$M$'', ya que éste último sufre mayor desutilidad que el primero. La función de utilidad del trabajador es por tanto $U^{B}(w, e)=w-e^{2}$ o $U^{M}(w, e)=w-2e^{2}$. La probabilidad de que el trabajador sea de tipo $B$ es $q$ (y por tanto con probabilidad $(1-q)$ el trabajador es de tipo $M$). La utilidad de reserva de ambos tipos de trabajador s $\underline{U}=0 .$ El empresario, por su parte, valora el esfuerzo del trabajador en $\pi=ke$, donde $k$ es una constante suficientemente grande (de tal modo que el empresario está interesado en contratar al trabajador sea cual sea su tipo). Por cada unidad de esfuerzo del trabajador el empresario obtiene, or tanto, $k$ unidades de beneficio.
			\begin{enumerate}[a)]
				\item Formular el problema que resuelve el principal con información simétrica. Calcular los contratos óptimos. ¿Cuáles son los esfuerzos que pide y los salarios que paga?
				\item Formular el programa que debería resolver el principal en información asimétrica y resuelvalo. ¿Cuáles son los contratos que el principal ofrece a los agentes? Comparar los casos de información simétrica y asimétrica.
			\end{enumerate}
				\textbf{\large Solución}\\
					\begin{enumerate}[a)]
	\item El problema de principal para el agente $M$ es maximizar su beneficio neto: 
	\begin{align*}
		& \M \quad k e-w^{M}\\
		& \begin{array}{ll}
			\text{s.a. } & U^{M} \geq 0\\
			& w^{M}-2 e^{M^{2}} \geq 0
		\end{array}
	\end{align*}
	De la restricción de participación se obtiene: $w^{M}=2 e^{M^{2}}$.\\
	
	Por tanto el beneficio neto es: $k e-2e^{2}$. Derivando respecto al esfuerzo e igualando a cero: $k-4e=0$ 
	$$e^{M*}=\frac{k}{4}$$
	Reemplazando en la $R.P.$: $w^{M*} =\frac{k^{2}}{16}$\\
	
	Similarmente, el problema de principal para el agente B es:
	\begin{align*}
		& \M \quad k e-w^{B}\\
		& \begin{array}{ll}
			\text{s.a. } & U^{B} \geq 0\\
			& w^{B}-2e^{B^{2}} \geq 0
		\end{array}
	\end{align*}
	De la restricción de participación se obtiene: $w^{B}=e^{B^{2}}$.\\
	
	Por tanto el beneficio neto es: $ke-e^{2}$. Derivando respecto al esfuerzo e igualando a cero:$k-2e=0$
	$$e^{B*}=\frac{k}{2}$$
	Reemplazando en la $R.P.$: $w^{B*}=\frac{k^{2}}{4}$.
	\item El principal ofrece un menú de contratos. El problema de optimización es: 
	\begin{align*}
		& \M \quad q\left(ke^{B}-w^{B}\right)+(1-q)\left(ke^{M}-w^{M}\right)\\
		& \begin{array}{llc}
			\text{s.a. } & \quad w^{B}-e^{B^{2}} \geq 0 & (R.P.1)\\
			& \quad w^{M}-2 e^{M^{2}} \geq 0 & (R.P.2) \\
			& w^{B}-e^{B^{2}} \geq w^{M}-e^{M^{2}} & (R.I.1) \\
			& w^{M}-2 e^{M^{2}} \geq w^{B}-2 e^{B^{2}} & (R.I.2)
		\end{array}
	\end{align*}
	La restricción de participación del agente $B (R.P.1)$ y la restricción de incentivos del agente $M (R.I.2)$ están incluidas en las demás restricciones.\\
	
	El problema se reduce entonces a :
	\begin{align*}
		& q\left(ke^{B}-w^{B}\right)+(1-q)\left(ke^{M}-w^{M}\right)\\
		& \begin{array}{ll}
			\text{s.a. } & w^{B}-e^{B^{2}} \geq w^{M}-e^{M^{2}} \\
			& w^{M}-2 e^{M^{2}} \geq 0
		\end{array}
	\end{align*}
	La ecuación de Lagrange será:
	$$ L=q\left(k e^{B}-w^{B}\right)+(1-q)\left(k e^{M}-w^{M}\right)+\lambda\left(w^{B}-e^{B^{2}}-w^{M}+e^{M^{2}}\right)+\mu\left(w^{M}-2e^{M^{2}}\right) $$
	Las condiciones de primer orden serán:
	\begin{align} 
		&\frac{\partial L}{\partial w^{B}}=0 \Leftrightarrow -q+\lambda=0 \label{eq1}\\
		&\frac{\partial L}{\partial w^{M}}=0 \Leftrightarrow -1+q-\lambda+\mu=0 \label{eq2}\\
		&\frac{\partial L}{\partial e^{B}}=0 \Leftrightarrow qk-2 \lambda e^{B}=0 \label{eq3}\\
		&\frac{\partial L}{\partial e^{M}}=0 \Leftrightarrow (1-q) k+2 \lambda e^{M}-4 \mu e^{M} \label{eq4}
	\end{align}
	De (\ref{eq1})
	\begin{gather}
		\lambda=q>0 \label{eq5}
	\end{gather}
	luego
	\begin{align}
		&\text{(\ref{eq5}) en (\ref{eq2})}: -1+q-q+\mu=0 \Rightarrow \mu=1>0 \notag\\
		&\text{(\ref{eq5}) en (\ref{eq3})}: qk-2qe^{B}=0 \Rightarrow e^{B*}=\frac{k}{2} \label{eq6}\\
		&\text{(\ref{eq5}) y (\ref{eq6}) en (\ref{eq4})}: (1-q)k+2qe^{M}-4e^{M} \Rightarrow e^{M*}=\frac{(1-q)k}{4-2q} \notag
	\end{align}
	De la restricción de participación del agente $M$:
	$$w^{M*}=2e^{M^{2}}$$
	De la restricción de incentivos del agente $B$:
	$$w^{B*}=e^{B^{2}}+w^{M}-e^{M^{2}}=e^{B^{2}}+2 e^{M^{2}}-e^{M^{2}}=e^{B^{2}}+e^{M^{2}}$$
	El esfuerzo que se pide al agente B es el mismo que con información simétrica $(\mathrm{k} / 2)$.\\
	
	El agente $M$ realiza un esfuerzo menor $\left(\frac{(1-q)k}{4-2q}<\frac{k}{4}\right)$\\
	
	El agente $M$ obtiene su utilidad de reserva (la restricción de participación se cumple con igualdad).\\
	
	El agente $B$ obtiene una renta informacional $\left(e^{B^{2}}+e^{M^{2}}>e^{B^{2}}\right)$
\end{enumerate}
		
		\item [8.] Una compañía ofrece el servicio de vuelo entre dos ciudades, siendo el costo por pasajero $C$. Hay dos tipos de clientes de esta compañía: los ejecutivos que se desplazan por razones de negocio y los turistas. La proporción de ejecutivos que demandan este destino es $\alpha$ y están dispuestos a pagar $P_A$ por el viaje. La proporción de turistas es $(1-\alpha)$ y están dispuestos a pagar $P_B$ por el mismo viaje. La utilidad que deriva cada ejecutivo del viaje es $U_A$ y la de cada tursitas $U_B$ con ($U_A > U_B$)Todos los pasajeros tienen una utilidad de reserva $U$. El problema al que se enfrenta la compañía es determinar la política de precios y plazas para los ejecutivos y los turistas, sin conocer a priori quienes son los unos y los otros. Escriba el problema de optimización del principal (sin resolverlo), con selección adversa.\\
		
			\textbf{\large Solución}\\
				\vspace{-0.3cm}
\begin{center}
	\begingroup
		\setlength{\tabcolsep}{10pt}
		\renewcommand{\arraystretch}{1.5} 
			\begin{tabular}{ccc}
					\hline
				\multicolumn{3}{c}{Datos} \\
					\hline
				2 consumidores: & {} & $A$ y $B$ \\
				2 bienes: & {} &$x$ e $y$ \\
				F.U: & {} &$u^{A} = \ln(x_A) + 2y_A$ y $u^{B} = \ln(x_B) + 2y_B$ \\
				Dotaciones: & {} &$w^A = (1, 4)$ y $w^B = (6, 3)$ \\
					\hline
			\end{tabular}
	\endgroup
\end{center}
La curva de contrato:
	$$RMS^A = RMS^B \Longrightarrow x_A = x_B$$
además
	$$x_A + x_B = \overline{x} \Longrightarrow x_A + x_A = \overline{x} \Longrightarrow x_A = \frac{\overline{x}}{2} = x_B$$
	\begin{center}
		\begin{tikzpicture}[scale=0.8]
				% Formato de CAJA
					\draw[->] (0,0) node[align=center, below left] {\footnotesize $O_A$} -- (0,8) node[align=center, above] {\footnotesize $x_{2}^{A}$};
				\draw[->] (0,0) -- (8,0) node[align=center, right] {\footnotesize $x_{1}^{A}$};
				
					\draw[->] (7,7) node[align=center, above right] {\footnotesize $O_B$} -- (-1,7) node[align=center, left] {\footnotesize $x_{2}^{B}$};
				\draw[->] (7,7) -- (7,-1) node[align=center, below] {\footnotesize $x_{2}^{B}$};
						
			% Curvas de indiferencia
				\draw [blue] (1.4,6.9) .. controls (2.84,5.7) and (3.8,5.2) .. (5.6,4.8);
				\draw [red] (1.4,6.15) .. controls (3.03,5.8) and (3.99,5.38) .. (5.6,4.05);
				
				\draw [blue] (1.4,5) .. controls (2.84,3.8) and (3.8,3.3) .. (5.6,2.9);
				\draw [red] (1.4,4.25) .. controls (3.03,3.9) and (3.99,3.48) .. (5.6,2.15);
				
				\draw [blue] (1.4,3.1) .. controls (2.84,1.9) and (3.8,1.4) .. (5.6,1);
				\draw [red] (1.4,2.35) .. controls (3.03,2) and (3.99,1.58) .. (5.6,0.25);
			% Curvas de contrato
				\draw [purple, very thick] (3.5,0) node [below, scale= 0.6mm] {$\frac{\overline{x}}{2}$} -- (3.5,7);
				
			% Flechas
				\node[draw, single arrow,
					minimum height=28mm, minimum width=1mm,
					single arrow head extend=1.5mm,
					anchor=west, blue, fill=blue, scale=0.5, rotate=33] at (1.5,5.1) {};
				\node[draw, single arrow,
					minimum height=28mm, minimum width=1mm,
					single arrow head extend=1.5mm,
					anchor=west, red, fill=red, rotate=-147, scale=0.5] at (5.5,2.1) {};
		\end{tikzpicture}
	\end{center}

Probamos si las dotaciones iniciales son ESP.
	$$x_{A} = x_{B} = \frac{\overline{x}}{2} \Longrightarrow	1 \neq 6 \neq \frac{7}{2}$$
Se comprueba que la dotación inicial
	\begin{center}
		\begin{tikzpicture}[scale=0.8]
			% Formato de CAJA
				\draw[->] (0,0) node[align=center, below left] {\footnotesize $O_A$} -- (0,8) node[align=center, above] {\footnotesize $x_{2}^{A}$};
				\draw[->] (0,0) -- (8,0) node[align=center, right] {\footnotesize $x_{1}^{A}$};
			
				\draw[->] (7,7) node[align=center, above right] {\footnotesize $O_B$} -- (-1,7) node[align=center, left] {\footnotesize $x_{2}^{B}$};
				\draw[->] (7,7) -- (7,-1) node[align=center, below] {\footnotesize $x_{2}^{B}$};
			
			% Curvas de indiferencia
				\draw [blue] (1.4,6.9) .. controls (2.84,5.7) and (3.8,5.2) .. (5.6,4.8);
				\draw [red] (1.4,6.15) .. controls (3.03,5.8) and (3.99,5.38) .. (5.6,4.05);
				
				\draw [blue] (1.4,5) .. controls (2.84,3.8) and (3.8,3.3) .. (5.6,2.9);
				\draw [red] (1.4,4.25) .. controls (3.03,3.9) and (3.99,3.48) .. (5.6,2.15);
				
				\draw [blue] (1.4,3.1) .. controls (2.84,1.9) and (3.8,1.4) .. (5.6,1);
				\draw [red] (1.4,2.35) .. controls (3.03,2) and (3.99,1.58) .. (5.6,0.25);
				
			% Curvas de contrato
				\draw [purple, very thick] (3.5,0) node [below, scale= 0.6mm] {$\frac{\overline{x}}{2}$} -- (3.5,7);
			
			% Flechas
				\node[draw, single arrow,
						minimum height=28mm, minimum width=1mm,
						single arrow head extend=1.5mm,
						anchor=west, blue, fill=blue, scale=0.5, rotate=33] at (1.5,5.1) {};
				\node[draw, single arrow,
						minimum height=28mm, minimum width=1mm,
						single arrow head extend=1.5mm,
						anchor=west, red, fill=red, rotate=-147, scale=0.5] at (5.5,2.1) {};
			
			% Intersección
				\draw[dashed] (1,7) node[above] {\footnotesize $x_{1}^{B}=6$} -- (1,0) node[below] {\footnotesize $x_{1}^{A}=1$};
				\draw[dashed] (0,4) node[left] {\footnotesize $x_{2}^{A}=4$} -- (7,4)node[right] {\footnotesize $x_{2}^{B}=3$};
			
			% Punto
				\draw[black, fill=black] (1,4) circle[radius=0.08] node[align=center, above left] {\footnotesize $w$};
		\end{tikzpicture}
	\end{center}

Con la inclusión de precios, la situación es la siguiente:
	$$RMS^A = RMS^B = \frac{p_x}{p_y}$$
	$$RMS^A=\frac{1}{2x_A}=\frac{p_x}{p_y} \quad , \quad RMS^B=\frac{1}{2x_B}=\frac{p_x}{p_y} \Longrightarrow x_A = x_B = \frac{p_y}{2p_x}$$
	
	$$
		\begin{array}{ccc}
			p_xx_A + p_yy_A = p_x + 4p_y & {} &p_xx_B + p_yy_B = 6p_x + 3p_y\\[0.4cm]
			p_x\frac{p_y}{2p_x} + p_yy_A = p_x + 4p_y & {} &p_x\frac{p_y}{2p_x} + p_yy_B = 6p_x + 3p_y\\[0.4cm]
			\frac{p_y}{2} + p_yy_A = p_x + 4p_y & {} & \frac{p_y}{2} + p_yy_B = 6p_x + 3p_y \\[0.4cm]
			y_{A}^* = \frac{2p_x + 7p_y}{2p_y} & {} & y_{B}^* = \frac{12p_x + 5p_y}{2p_y}
		\end{array}
	$$
	
	$$y_{A}^* + y_{B}^* = 4 + 3 = 7 \Longrightarrow \frac{2p_x + 7p_y}{2p_y} + \frac{12p_x + 5p_y}{2p_y} = 7  \Longrightarrow  \frac{p_x}{p_y} = \frac{1}{7}$$
	
	\begin{gather*}
		\therefore y_{A}^* = \frac{51}{14}  \Longrightarrow x_{B}^* = \frac{7}{2}\\[0.4cm]
		\therefore y_{b}^* = \frac{47}{14}  \Longrightarrow x_{B}^* = \frac{7}{2}
	\end{gather*}

	\begin{center}
		\begin{tikzpicture}[scale=0.8]
			% Formato de CAJA
				\draw[->] (0,0) node[align=center, below left] {\footnotesize $O_A$} -- (0,8) node[align=center, above] {\footnotesize $x_{2}^{A}$};
				\draw[->] (0,0) -- (8,0) node[align=center, right] {\footnotesize $x_{1}^{A}$};
				
				\draw[->] (7,7) node[align=center, above right] {\footnotesize $O_B$} -- (-1,7) node[align=center, left] {\footnotesize $x_{2}^{B}$};
				\draw[->] (7,7) -- (7,-1) node[align=center, below] {\footnotesize $x_{2}^{B}$};
			
			% Curvas de indiferencia
				\draw [blue] (1.4,6.9) .. controls (2.84,5.7) and (3.8,5.2) .. (5.6,4.8);
				\draw [red] (1.4,6.15) .. controls (3.03,5.8) and (3.99,5.38) .. (5.6,4.05);
				
				\draw [blue] (1.4,5) .. controls (2.84,3.8) and (3.8,3.3) .. (5.6,2.9);
				\draw [red] (1.4,4.25) .. controls (3.03,3.9) and (3.99,3.48) .. (5.6,2.15);
				
				\draw [blue] (1.4,3.1) .. controls (2.84,1.9) and (3.8,1.4) .. (5.6,1);
				\draw [red] (1.4,2.35) .. controls (3.03,2) and (3.99,1.58) .. (5.6,0.25);
			
			% Curvas de contrato
				\draw [purple, very thick] (3.5,0) -- (3.5,7);
			
			% Intersección
				\draw[dashed] (1,7) node[above] {\footnotesize $x_{1}^{B}=6$} -- (1,0) node[below] {\footnotesize $x_{1}^{A}=1$};
				\draw[dashed] (0,4) node[left] {\footnotesize $x_{2}^{A}=4$} -- (7,4)node[right] {\footnotesize $x_{2}^{B}=3$};
			
			% Punto
				\draw[black, fill=black] (1,4) circle[radius=0.08] node[align=center, above left] {\footnotesize $w$};
				\draw[black, fill=black] (3.5,3.58) circle[radius=0.08] node[align=center, below left] {\footnotesize $EGW$};
			
			% Recta presupuestaria
				\draw (0,5.26) -- (7,1.9) node[right] {\footnotesize $RP$};
			
			% Intersección
				\draw[dashed] (3.5,7) node[above] {\footnotesize $x_{1}^{B}=3.5$} -- (3.5,0) node[below] {\footnotesize $x_{1}^{A}=3.5$};
				\draw[dashed] (0,3.58) node[left] {\footnotesize $x_{2}^{A}=3.64$} -- (7,3.58)node[right] {\footnotesize $x_{2}^{B}=3.36$};
		\end{tikzpicture}
	\end{center}
		
		\item [9.] Hay muchos compradores de autos deportivos de color rojo. Los \emph{``snobs''} (en proporción $q$) están dispuestos a pagar hasta $50.000$ dólares por un auto de color rojo (logrando una utilidad $U_{s} > 50.000$ dólares). Por otra parte, los \emph{``menos snobs''} (en proporción $1-q$) pagarían hasta $30.000$ dólares por un auto de color rojo logrando una utilidad $U_{M}>30.000$ dólares. Se sabe que $U_S > U_{M}$. Los vendedores de autos tienen que elegir el tipo de contrato que desean ofrecer a cada comprador le autos rojos. Teniendo en cuenta que la utilidad de reserva de los compradores \emph{``snobs''} es de $U_{S}=25.000$ dólares y la de los \emph{``menos snobs''} es de $U_{M}=15.000$ dólares. ¿Cómo diseñan los vendedores sus contratos, en selección adversa? (formule el problema sin resolverlo).\\
		
			\textbf{\large Solución}\\
				\begin{enumerate}[a)]
	\item La pendiente de la \emph{FBS} $u^2(u^1) = \sqrt{100-u^2}$ es $\frac{\partial u^2}{\partial u^1} = -\frac{1}{2\sqrt{100-u^2}}$ y, evaluada en el punto $(u^1,u^2) = (75,5)$, es (el valor absoluto) igual a $\frac{1}{10}$. Por otra parte, dada la \emph{FBS} de Bergson que ``tiene in mente'' el individuo 1 y que es del tipo $W(u^1,u^2) = \theta^1u^1 + \theta^2u^2$, su pendiente es $-\frac{\theta^1}{\theta^2}$. Entonces ha de verificarse.
		$$\frac{\theta^1}{\theta^2}=\frac{1}{10}$$
	Por otra parte, el punto de máximo bienestar debe ser tal que el problema
		\begin{equation}
			\begin{aligned}
				& \M \limits_{(u^1,u^2)} \quad \theta^1u^1 + \theta^2u^2\\
				& \begin{array}{ll}
					\text{s.a. } & u^1 +(u^2)^2 = 100
				\end{array}
			\end{aligned} \label{eq1}
		\end{equation}
	dé como resultado el reparto de bienestar $(u^1,u^2) = (75,5)$. Las CPO de este problema (\ref{eq1}) son 
		\begin{gather}
			\theta^1 - \lambda = 0 \label{eq2}\\
			\theta^2 - 2 \lambda u_2 = 0 \label{eq3}\\
			\intertext{y}
			100 - u_1 - u_{2}^2 = 0
		\end{gather}
	De las CPO (\ref{eq2}) y (\ref{eq3}) resulta $\theta^1=\frac{\theta^2}{2u^2}$, y dado que $u^2=5$, se obtiene $\theta^2=10\theta^2$. En definitivo,
		$$W(u^1, u^2) = \theta^1u^1+10\theta^1u^2, \theta^1>0$$
	es la \emph{FBS} utilitarista ponderada a o de Bergson que implícitamente está utilizando el individuo 1.
	\item Operando de manera análoga a la del apartado 1, se obtiene
		$$W(u^1, u^2) = \theta^1u^1+18\theta^1u^2, \theta^1>0$$
	como \emph{FBS} según el criterio del individuo 2.
\end{enumerate}
	\end{enumerate}
	
\textbf{\LARGE Riesgo Moral}
	\begin{enumerate}
		\item [10.] Un trabajador puede ejerce dos esfuerzos, ``bueno'' y ``malo'', lo que induce la observación de fallos en el resultado con probabilidades 0.25 y 0.75 respectivamente. Su función de utilidad es $U(w,e) = 100-(10/w)-v$ , donde $w$es el salario que recibe, y $v$ toma el valor 2 si incorpora el buen esfuerzo y 0 si elige el malo. La aparición de un fallo (por ejemplo, una avería) es una variable que puede ser incluida en los términos del contrato, pero el esfuerzo no. El producto que se obtiene vale 20 si no hay errores, y no vale nada si tiene alguno. El principal es neutral ante el riesgo. Suponemos que el trabajador tiene una utilidad de reserva igual a $\underline{U} =0$. Calcule el contrato óptimo y el esfuerzo que el principal desea conseguir, en condiciones de información simétrica y de información asimétrica sobre el comportamiento del agente.\\
			
			\textbf{\large Solución}\\
				\begin{enumerate}[a)]
	\item La fucnión de producción del bien 2, $q_2 = K_2 + L_2$, se puede expresar como
					\begin{gather}
						q_2 = (10 - K_1) + 2(10- L_1) = 30-(K_1 + 2L_2) \label{eq7}
					\end{gather}
			y dado que $q_1 = K_1 + 2L_1$, basta con reescribir (\ref{eq7}) para llegar al $CPP$ de la economía
					\begin{gather}
						\left\lbrace (q_1,q_2) \in \mathbb{R}_{+}^{2} \mid q_2(q_1) = 30 - q_1\right\rbrace \label{eq8}
					\end{gather}
			De mamenra análogo, el $CPU$ se obtiene teniendo en cuenta las funciones de utlidad de los consumidores, $u^1 = q_{1}^{1} + q_{2}^{1}$ y $u^2 = q_{2}^{2}$, y además el hecho de que la cantidad de bien 1 satisface la condición
					$$q_{1}^{1} + 0 = q_1$$
			mientras que la cantidad de bien 2 verifica
					$$q_{2}^{1} + q_{2}^{2} = q_{2}$$
			condición que teniendo en cuenta el $CPP$ dado en (\ref{eq8}) se convierte en
					$$q_{2}^{1} + q_{2}^{2} = 30 - q_{1}$$
			Sumando las utilidades, se tiene que $u^1 + u^2 = q_1 + q_2 = 30$. En definitiva, el $CPP$ de la economía descrita es
					$$\left\lbrace (u^1,u^2) \in \mathbb{R}_{+}^{2} \mid u^2(u^1) = 30 - u^1\right\rbrace$$
	\item Dada las asignaciones $\left[ \left( q_{1}^{1},q_{2}^{1} \right) ,\left( q_{1}^{2},q_{2}^{2} \right) \right] = \left[ \left( 10, 5\right) ,\left( 0, 15\right) \right]$, asignación representada por el punto $A$ de la caga de Edgeworth para el consumo de la siguiente figura:
			\begin{center}
				\begin{tikzpicture}[scale=1.065]
					% FPP
						\draw[orange] (0,3) node [black, left, scale = 0.6] {30} -- (3,0) node [black, below, scale = 0.6] {30};
					% Curva de indiferencia
						% Agente A
							\draw[blue] (0,2) node [black, above left, scale = 0.5] {$20$} -- (1,1);
							\draw[blue] (0,1.5) -- (1,0.5);
							\draw[blue] (0,1) -- (1,0) node [black, below right, scale = 0.5] {$10$};
						% Agente B
							\draw[red] (0,0.5) -- (1,0.5);
							\draw[red] (0,1) -- (1,1);
							\draw[red] (0,1.5) -- (1,1.5);
					% Caja en consumo
						\draw[<->] (0,2.5) node [black, left, scale = 0.6] {$q_{2}^{1}$}-- (0,0)  node [black, below left , scale = 0.6] {$O_1$}-- (1.5,0) node [black, below, scale = 0.6] {$q_{1}^{1}$};
						\draw[<->] (-0.5,2) node [black, left, scale = 0.6] {$q_{1}^{2}$} -- (1,2) node [black, above right, scale = 0.6] {$O_2$} -- (1,-0.5) node [black, below, scale = 0.6] {$q_{2}^{2}$};
					% Eje
						\draw[<->] (0,4) node[align=center, above] {$q_2$} -- (0,0)  -- (4,0) node[align=center, right] {$q_1$};
					% Punto
						\draw[black, fill=black] (1,0.5) circle[radius=0.05] node [right, scale=0.7] {$A$};
				\end{tikzpicture}
			\end{center}
		es claro que una posibilidad para aumentar $u^2$ es aumentar $q_{2}^{2}$, pero entonces disminuye $y^1$. Del mismo modo, para aumentar $u^1$ habría que incrementar $q_{1}^{1}$ o $q_{2}^{1}$, pero entonces disminuye $u^1$. Luego la asignación $\left[ \left( q_{1}^{1},q_{2}^{1} \right) ,\left( q_{1}^{2},q_{2}^{2} \right) \right] = \left[ \left( 10, 5\right) ,\left( 0, 15\right) \right]$ es eficiente en el consumo. y como el plan de producción $\left[ \left( q_1, K_1, L_1 \right) ,\left( q_2, K_2, L_2 \right) \right] = \left[ \left( 10, 10, 0\right), \left( 20, 0, 10\right) \right]$ que permite dichos consumos es técnicamente eficiente, se concluye que la asignación es eficiente en términos globales.\\
		
		Para descentralizar esta asignación de recursos, es necesario tener en cuenta, de acuerdo con la siguiente figura:
			\begin{center}
				\begin{tikzpicture}[scale=1.5]
					% FPP
						% Curva
							\draw[orange] (0,3) node [black, left, scale = 0.6] {30} -- (3,0) node [black, below, scale = 0.6] {30};
						% Eje
							\draw[<->] (0,4) node[align=center, above] {$q_2$} -- (0,0)  -- (4,0) node[align=center, right] {$q_1$};
					% Caja en consumo
						% Curva de indiferencia
							% Agente A
							\draw[blue] (0,1.5) -- (1,0.5);
							% Agente B
							\draw[red] (0,0.5) -- (1,0.5);
						% Ejes
							\draw[<->] (0,2.5) node [black, left, scale = 0.6] {$q_{2}^{1}$}-- (0,0)  node [black, below left , scale = 0.6] {$O_1$}-- (1.5,0) node [black, below, scale = 0.6] {$q_{1}^{1}$};
							\draw[<->] (-0.5,2) node [black, left, scale = 0.6] {$q_{1}^{2}$} -- (1,2) node [black, above right, scale = 0.6] {$O_2$} -- (1,-0.5) node [black, below, scale = 0.6] {$q_{2}^{2}$};
						% Punto
							\draw[black, fill=black] (1,0.5) circle[radius=0.035] node [right, scale=0.7] {$A$};
							\draw[black, fill=black] (0,1.5) circle[radius=0.035] node [left, scale=0.7] {$B$};
							\draw (1,0) node [black, below right, scale = 0.5] {$10$};
							\draw (0,2) node [black, above left, scale = 0.5] {$20$};
						% Recta de precios
						\draw (0.1,0.8) -- (1,0.5) -- (0.4,1.6);
					% Flechas
						\draw[->] (1.6,2.8) node [right, scale = 0.5] {Pendiente $= -1$}-- (0.05,1.45);
						\draw[->] (2,2.5) node  [right, scale = 0.5] {$\frac{p_1}{p_2} > 1$} -- (0.58,1.26);
						\draw[->] (2.4,2.2) node [right, scale = 0.5] {$\frac{p_1}{p_2} < 1$} --(0.6,0.63);
				\end{tikzpicture}
			\end{center}
		que si los precios de los bienes, $p_1$ y $p_2$, son tales que $\frac{p_1}{p_2} > 1$, no es posible descentralizar la citada asignación porque el individuo 1 consumiría únicamente el bien 2. Es por ello que los precios candidatos a descentralizar la asignación propuesta han de verificar la condición
			$$\frac{p_1}{p_2} \leq 1$$
		A la hora de calcular la dotación incial de recursos de los consumidores , $\left[W^1, W^2 \right] = \left[\left(w_{1}^{1},w_{2}^{1} \right), \left(w_{1}^{2},w_{2}^{2} \right) \right] $, en la que debes estas ``situada'' la economía para alcanzar la asignaciones propuestas, es necesario tener en cuenta que la utilidad del consumidor 1 debe cumplir la condición $u^{1}(W^1) = u^1(10,5)$. Con ello, las condiciones a satisfacer por $\left[W^1, W^2 \right]$ son:
			\begin{gather}
				10p_1 + 5p_2 = p_1w_{1}^{1} + p_2w_{2}^{1} \label{eq13}\\
				15p_2 = p_1w_{1}^{2}+p_2w_{2}^{2} \label{eq14} \\
				w_{1}^{1} + w_{1}^{2} = 10 \label{eq15} \\
				w_{2}^{1} + w_{2}^{2} = 20 \label{eq16} \\
				\frac{p_1}{p_2} < 1 \label{eq17}
			\end{gather}
		y, además, $\left[W^1, W^2 \right]$ ha de estar situada dentro del triángulo $ABC$ (ya que si está por encima, entonces $u^2(W^2) < u^2(0,5)$ y si está por debajo entonces $u^1(W^1) < u^1(10,5))$.\\
		
		Si fijamos $p_1 = 1$ y $p_2 = 2$, la condición (\ref{eq13}) se convierte en
			\begin{gather}
				10 + 10 = w_{1}^{1} + 2w_{2}^{1} \label{eq18}
			\end{gather}
		y la condición (\ref{eq14}) en
			\begin{gather}
				30 = w_{1}^{2} + 2w_{2}^{2} = (10 - w_{1}^{1}) + 2(20 - w_{2}^{1}) = 50 -w_{1}^{1} - 2w_{2}^{1} \label{eq19}
			\end{gather}
		una vez tenidas en cuenta (\ref{eq15}) y (\ref{eq16}). Finalmente, es vlaro que las condiciones (\ref{eq18}) y (\ref{eq19}) son idénticas, ya que ambas colapsan en $20 = w_{1}^{1} + 2w_{2}^{1}$, uqe es la condición de recursos inciales. En definitiva, con el sistema de precios
			$$(p_1,p_2) = (1,2)$$
		y el conjunto de rentas
			$$(m^1, m^2) = (20,30)$$
		se consigue descentralizar la asignación propuesta.
\end{enumerate}
		\item [11.] Sea una relación de agencia en la que le principal contrata a un agente cuyo esfuerzo determina el resultado. Supongamos que la incertidumbre está representada  por tres estados de la naturaleza . El agente puede elegir entre dos esfuerzo. Los resultados están recogidos en la tabla siguiente
						\begin{center}
							\begingroup
								\setlength{\tabcolsep}{10pt} % Default value: 6pt
								\renewcommand{\arraystretch}{1.5} % Default value: 1
									\begin{tabular}{cc|ccc}
												&& \multicolumn{3}{c}{Estados de la naturaleza}\\
												&& $\varepsilon_1$ & $\varepsilon_2$ & $\varepsilon_3$\\
										\hline
											\multirow{2}{*}{Esfuerzos} & e=6 & 60.000 & 60.000 & 30.000\\
																	   & e=4 & 30.000 & 60.000 & 30.000
									\end{tabular}
							\endgroup
						\end{center}
		El principal y el agente creen que la probabilidad de cada uno de los estados es un tercio. Las funciones objetivos del principal y del agente son, respectivamente:
				\begin{align*}
					B(x,w) & = x-w\\
					U(w,e) & = \sqrt{w} - e^2
				\end{align*}
		donde $x=x(e,\varepsilon)$ es el resultado monetario de la relación, $y=w(x)$ representa el pago monetario que recibe el agente. Supondremos que el agente solo acepta el contrato si obtiene por lo menos una utilidad esperada de 114 (su nivel de utilidad de reserva)
			\begin{enumerate}[a)]
				\item ¿Qué puede deducirse de las funciones objetivo de los participantes?
				\item ¿Cuál será el esfuerzo y el pago en una situación de información simétrica? ¿Qué ocurriría si el principal no fuese neutral ante el riesgo?
				\item ¿Qué ocurre en una situación de información asimétrica? ¿Qué esquema de pago permite conseguir el esfuerzo $e=4$? ¿Qué esquema de pago permite conseguir el esfuerzo $e=6$? ¿Qué esfuerzo decide conseguir el principal? Analice el resultado
			\end{enumerate}
					\textbf{\large Solución}\\
						\begin{enumerate}[a)]
	\item El principal es neutral ante el riesgo y el agente averso
	\item Los contrato óptimos se derivan de 
				\begin{enumerate}[i)]
					\item el principal asume todo el riesgo
					\item la restricción de aceptación está saturada
				\end{enumerate}
			Si $e=6$, $w$ es tal que $w^{1/2} - 6^2 =114$. Luego $w=22.500$. En este caso, $E[U_P] = 50.000-22.500=27.500$. Si $e=4$, entonces $w=16.900$ y $E[U_P]=23.100$. La solución en información simétrica es $e^*=6$, $w^*=22.500$. Si el principal no fuese neutral al riego, ambos participantes se repartirían el riesgo de la relación.
	\item El contrato óptimo si $e=4$ es el mismo que antes: $w=16.900$, ya que ante un pago fijo el agente siempre escoge el esfuerzo más bajo. Para alcanzar $e=6$, el principal ofrecerá un contrato contingente al resultado. Pagará $w(60)$ si el resultado es 60.000 y $w(30)$ si el resultado es 30.000. El contrato debe verificar simultáneamente las restricciones de participación y de incentivo:
				\begin{align*}
					& \frac{2}{3}\left[w(60) \right]^{1/2}+\frac{1}{3}\left[w(30) \right]^{1/2} - 36 \geq 114\\
					& \frac{2}{3}\left[w(60) \right]^{1/2}+\frac{1}{3}\left[w(30) \right]^{1/2} - 36 \geq \frac{2}{3}\left[w(30) \right]^{1/2}+\frac{1}{3}\left[w(60) \right]^{1/2} - 16
				\end{align*}
			Ambas restricciones se darán con igualdad en la solución al problema de ``gastar lo menos posible'' de principal. Tenemos dos ecuaciones y dos incógnitas que llevan a $w(60)=28.900$ y $w(30)=12.100$. La utilidad esperada del principal es $U_P=26.700$. En información asimétrica elige también $e=6$, ya que $26.700 > 23.100$, pero con una pérdida en eficiencia medidad por la disminución de los beneficios esperados del principal (el agente siempre obtiene su utilidad de reserva)
\end{enumerate}
		\item [12.] Supongamos una relación entre un empresario (principal) y un agente en la que dos resultados son posibles cuyos valores son 50.000 y 25.000. El agente tiene que elegir entre tres posibles esfuerzos que tienen efecto sobre la distribución de los resultados. La distribución de probabilidades sobre los resultados en función de los esfuerzos viene dada por la tabla siguiente:
				\begin{center}
					\begingroup
						\setlength{\tabcolsep}{10pt} % Default value: 6pt
						\renewcommand{\arraystretch}{1.5} % Default value: 1
							\begin{tabular}{cc|cc}
									&& \multicolumn{2}{c}{Resultados}\\
											&& 25.000 & 50.000\\
								\hline
									\multirow{3}{*}{Esfuerzos} & $e^1$ & 1/4 & 3/4\\
															   & $e^2$ & 1/2 & 1/2\\
															   & $e^3$ & 3/4 & 1/4
							\end{tabular}
						\endgroup
				\end{center}
			Suponemos que el principal es neutral ante el riesgo y el agente adverso, con las preferencias descritas respectivamente por las siguientes funciones:
			$$B(x,w)=x-w \quad \text{y} \quad U(w,e)=\sqrt{w}-v(e)$$
			con $v\left(e^{1}\right)=10$, $v\left(e^{2}\right)=4$, y $v\left(e^{3}\right)=0$. El nivel de utilidad de reserva del agente es $\underline{U}=120$.
				\begin{enumerate}[a)]
					\item Escríbanse los contrato óptimos en información simétrica para cada nivel de esfuerzo y los beneficios obtenidos por el principal en cada caso. ¿Qué nivel de esfuerzo prefiere el principal?
					\item Escríbase los contrato óptimos cuando existe un problema de riesgo moral. ¿Cuál es el nivel d esfuerzo y el contrato que el princiapl elige? ¿Dónde influye el problema de riesgo moral?
				\end{enumerate}
					\textbf{\large Solución}\\
						\begin{enumerate}[a)]
	\item En información simétrica el agente recibirá un pago fijo, determinado por su restricción de participación.
			\begin{enumerate}[i)]
				\item Si $e^1$, el salario es $w=25.600$ y los beneficios del principal $U_P = 18.150$.
				\item Si $e^2$, $w=19.600$ y $U_P=17.900$.
				\item Si $e^3$, $w=15.625$ y $U_P=15.625$
			\end{enumerate}
		Por tanto, en información simétrica, el principal elige $e^1$.
	\item Si existe un problema de riesgo moral, entonces
			\begin{enumerate}[i)]
				\item cuando el principal desea conseguir el esfuerzo $e^1$, construye el mecanismo de pago contingente al resultado $\left[ w(25), w(50)\right]$ siguiendo el programa:
					\begin{align*}
						& \M \limits_{w(25), w(50)} \quad \frac{1}{4} w(25)+\frac{3}{4} w(50)\\[0.2cm]
						& \begin{array}{ll}
							\text{s.a.}  & \frac{1}{4} w(25)^{1 / 2}+\frac{3}{4} w(50)^{1 / 2}-40 \geq 120 \\[0.2cm]
										 & \frac{1}{4} w(25)^{1 / 2}+\frac{3}{4} w(50)^{1 / 2}-40 \geq \frac{1}{2} w(25)^{1 / 2}+\frac{1}{2} w(50)^{1 / 2}-20 \\[0.2cm]
										 & \frac{1}{4} w(25)^{1 / 2}+\frac{3}{4} w(50)^{1 / 2}-40 \geq \frac{3}{4} w(25)^{1 / 2}+\frac{1}{4} w(50)^{1 / 2}-5
						  \end{array}
					\end{align*}
				La solución es $w(25)=10.000, w(50)=32.400$ y $U_P=16.950$
				
				\item Si el principal quiere el esfuerzo $e^2$, entonces el mecanismo de pago $w(25)=12.100, w(50)=28.900$ y $U_P=17.00$\\
				
						De forma rigurosa, $s(25) = w(25)^{1/2}$ y $s(50)=w(50)^{1/2}$, el programa será
							\begin{align*}
								& \M \limits_{s(25), s(50)} \quad -s(25)^{2}-s(50)^{2}\\
								& \begin{array}{ll}
									\text{s.a.} & s(25)+s(50) \geq 280 \\
												& s(25)-s(50) \leq 80  \\
												& s(50)-s(25) \geq 60
								  \end{array}
							\end{align*}
						Sean $\lambda, \mu$ y $\delta$ los multiplicadores de las tres restricciones. Las condiciones de primer orden del lagrangiano se escriben: $-2s(25)+\lambda+\mu-\delta=0$ y $-2s(50)+\lambda-\mu+\delta$, lo que implica que $\lambda=s(50)+s(25)$. La primera restricción se verifica con igualdad. Como las restricciones segunda y tercera son incompatibles con igualdad, una de ellas estará no saturada, luego o biene $\mu=0$, o bien $\delta=0$. Si $\delta=0$, entonces, por las condiciones de primer orden del lagrangiano y por la expresión de $\lambda$, tenemos que $\mu=s(25)-s(50)$, pero ello es imposible pues $s(50)-s(25)\geq 60\geq0$. Luego $\mu=0$, con lo que $\delta=s(50)-s(25)>0$. Esto implica que la tercera restricción se verifica con igualdad. $s(25)$ y $s(50)$ salen, pues, del sistema formado por la primera y tercera restricciones verificadas con igualdad.
						
				\item El contrato óptimo para conseguir el esfuerzo $e^3$ es el mismo que en información simétrica, luego $w(25)=w(50)=15.625$ y $U_P=15.625$. En información asimétrica el principal elige el esfuerzo $e^2$. La información asimétrica no solo influye en que, dado $e^2$, los beneficios son menores $(17.00<17.900)$, sino que el esfuerzo elegido también es distinto ($e^2<e^1$)
			\end{enumerate}
\end{enumerate}
	\end{enumerate}
%------------------------------------------------------------------------------------
\end{document}		
%====================================================================================