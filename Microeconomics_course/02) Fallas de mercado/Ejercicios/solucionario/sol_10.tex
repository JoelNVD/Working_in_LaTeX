Si el principal desea conseguir el esfuerzo malo pagará un salario fijo al agente (ello es cierto tanto en información simétrica como asimétrica) de $w=1/10$. El beneficio del principal es $U_P=(1/4)20-(1/10) = 49/10$. Si el principal quiere que el agente realice el esfuerzo bueno y este esfuerzo es contractual le ofrecerá un salario fijo $w=10/98$. El beneficio del principal es $U_P=(3/4)20-(10/98)=1470/98$. Cuando la información sobre el esfuerzo es asimétrica, entonces el principal debe ofrecer un contrato $(w^E,w^F)$ en función del ``éxito'' o ``fracaso''. El contrato debe satisfacer la condición de incentivos, y es fácil ver que las dos restricciones tienen que verificarse con igualdad. El sistema de dos ecuaciones con dos incógnitas puede reescribirse como:
	$$\frac{3}{w^E}+ \frac{1}{w^F}=\frac{392}{10} \quad \text{y} \quad \frac{1}{w^F} - \frac{1}{w^E}=\frac{4}{10}$$ 
cuya solución es $w^E=10/97$ y $w^F=10/101$. El beneficio esperado por el principal es $U_P = (3/4)[20-10/97]-(1/4)(10/101)$, superior al beneficio con el esfuerzo malo.