\begin{enumerate}[a)]
	\item El principal es neutral ante el riesgo y el agente averso
	\item Los contrato óptimos se derivan de 
				\begin{enumerate}[i)]
					\item el principal asume todo el riesgo
					\item la restricción de aceptación está saturada
				\end{enumerate}
			Si $e=6$, $w$ es tal que $w^{1/2} - 6^2 =114$. Luego $w=22.500$. En este caso, $E[U_P] = 50.000-22.500=27.500$. Si $e=4$, entonces $w=16.900$ y $E[U_P]=23.100$. La solución en información simétrica es $e^*=6$, $w^*=22.500$. Si el principal no fuese neutral al riego, ambos participantes se repartirían el riesgo de la relación.
	\item El contrato óptimo si $e=4$ es el mismo que antes: $w=16.900$, ya que ante un pago fijo el agente siempre escoge el esfuerzo más bajo. Para alcanzar $e=6$, el principal ofrecerá un contrato contingente al resultado. Pagará $w(60)$ si el resultado es 60.000 y $w(30)$ si el resultado es 30.000. El contrato debe verificar simultáneamente las restricciones de participación y de incentivo:
				\begin{align*}
					& \frac{2}{3}\left[w(60) \right]^{1/2}+\frac{1}{3}\left[w(30) \right]^{1/2} - 36 \geq 114\\
					& \frac{2}{3}\left[w(60) \right]^{1/2}+\frac{1}{3}\left[w(30) \right]^{1/2} - 36 \geq \frac{2}{3}\left[w(30) \right]^{1/2}+\frac{1}{3}\left[w(60) \right]^{1/2} - 16
				\end{align*}
			Ambas restricciones se darán con igualdad en la solución al problema de ``gastar lo menos posible'' de principal. Tenemos dos ecuaciones y dos incógnitas que llevan a $w(60)=28.900$ y $w(30)=12.100$. La utilidad esperada del principal es $U_P=26.700$. En información asimétrica elige también $e=6$, ya que $26.700 > 23.100$, pero con una pérdida en eficiencia medidad por la disminución de los beneficios esperados del principal (el agente siempre obtiene su utilidad de reserva)
\end{enumerate}