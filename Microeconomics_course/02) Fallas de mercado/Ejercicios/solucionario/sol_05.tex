Para hallar el consumo óptimo de cuadros y cervezas de cada estudiante cuando vive solo. Se aplica lo siguiente:
	$$\frac{Umg_{X}^{i}}{Umg_{Y}^{i}} = \frac{P_{X}}{P_{Y}}$$
Resolviendo:
	\begin{align*}
		\frac{\frac{1}{3}X^{\frac{-2}{3}}_{i}Y^{\frac{2}{3}}_{i} }{\frac{2}{3}X^{\frac{1}{3}}_{i}Y^{\frac{-1}{3}}_{i}} &= \frac{P_{X}}{P_{Y}}\\
		\frac{Y_{i}}{2X_{i}} &= \frac{P_{X}}{P_{Y}}
	\end{align*}
Tenemos la restricción presupuestaria:
	$$X_{i}P_{X} + Y_{i}P_{Y} = M$$
Hallando $X_{i}$:
	\begin{align*}
		X_{i}P_{X} + 2X_{i}\frac{P_{X}}{P_{Y}}P_{Y} &= M\\
		3X_{i}P_{X} &= M\\
		X_{i}^* &= \frac{M}{3P_{X}}
	\end{align*}
Hallando $Y_{i}$:    
	\begin{align*}
		Y^*_{i} &= 2X_{i}^*\frac{P_{X}}{P_{Y}}\\
		Y^*_{i} &= 2\frac{M}{3P_{X}}\frac{P_{X}}{P_{Y}}\\
		Y^*_{i} &= \frac{2M}{3P_{Y}}
	\end{align*}
Reemplazando los datos: $M=3000$, $P_{X}=100$, $P_{Y}=0.2$, se tiene los consumos óptimos.
	\begin{align*}
		X^*_{i} &= 10\\
		Y^*_{i} &= 10000
	\end{align*}