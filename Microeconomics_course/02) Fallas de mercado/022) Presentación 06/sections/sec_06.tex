%====================================================================================
\section[Modelo]{Un modelo simplificado de economía con bienes públicos}
%====================================================================================

%------------------------------------------------
\begin{frame}{Un modelo simplificado de economía con bienes públicos}
	\begin{itemize}
		\item Caracterización de la economía
		\item Provisión eficiente del bien público.
		\item Provisión voluntaria del bien público.
		\item Equilibrio de Lindahl.
	\end{itemize}
\end{frame}
%------------------------------------------------
\begin{frame}{Caracterización de la economía}
	\begin{itemize}
		\item Se tienen una economía con dos agentes ($A$ y $B$) y dos bienes ($X$= bien privado, $G$ = bien público).
		\item Cada agente tiene dotaciones iniciales de dinero: $(W^A, W^B)$
		\item Cada agente puede aportar una cierta cantidad de dinero para financiar el bien público: $(g^2, g^B)$
		\item El bien X es numerario. Por tanto: $X^i = W^i - g^i$
		\item $r$ es la cantidad de bien privado usada como insumo para producir el bien público.
	\end{itemize}
\end{frame}
%------------------------------------------------
\begin{frame}{Provisión eficiente de un bien público}
	\begin{itemize}
		\item La condición de óptimo es: 
				$$TMT = \sum TMS (\text{ Regla Bowen – Lindhal – Samuelson})$$
		\item En el caso particular de dos bienes, se asume que el $CMg$ de una unidad de bien público es una unidad de bien privado:
				$$TMS^A + TMS^B = 1$$
	\end{itemize}
\end{frame}
%------------------------------------------------
\begin{frame}{Provisión voluntaria del bien público}
	\begin{itemize}
		\item La economía con bienes públicos es una economía de propiedad privada en la que cada consumidor participa en la producción del bien público financiando una parte de su costo, que decide en función de sus preferencias.
		\item El bien público se obtiene a partir de la suscripción voluntaria de los diferentes consumidores, cada uno de los cuales contribuye con una cantidad $z_i$ de sus recursos (que la empresa obtiene sin costo).
		\item Los consumidores disfrutarán del bien público que es ofrecido ``gratuitamente'' por la empresa que lo produce.
		\item Resolviendo el problema de optimización, se tiene que cada consumidor iguala su $TMS$ a la $TMT$.
		\item Ineficiencia: cada consumidor no tiene en cuenta el beneficio que de su contribución se deriva para los demás consumidores.
		\item La eficiencia requiere una coordinación que no se da en el contexto competitivo, de donde resulta una provisión insuficiente de bienes públicos.
	\end{itemize}
\end{frame}
%------------------------------------------------
\begin{frame}{Equilibrio de Lindhal}
	\begin{itemize}
		\item Se trata de crear nuevos precios y mercados para que todos los efectos externos sean internalizados.
		\item El bien público puede considerarse un bien distinto para cada consumidor, por tanto, aunque las cantidades sean las mismas, los precios son personalizados.
		\item Resultado: equilibrio de Lindhal (eficiente).
	\end{itemize}
\end{frame}